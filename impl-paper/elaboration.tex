\newcommand{\ttinterp}[1]{\mathcal{E}\interp{#1}}

\section{Elaborating \Idris{}}

An \Idris{} program consists of a series of declarations --- data types, functions,
type classes and instances. In this section, we describe how these high level declarations
are translated into a \TT{} program consisting of inductive families and pattern matching
function definitions. We will need to work at the \remph{declaration} level, and at
the \remph{expression} level, defining the following meta-operations
which together constitute an algorithm for elaborating \Idris{}
programs to \TT{}.

\begin{itemize}
\item $\ttinterp{\cdot}$, which builds a \TT{} expression from an \Idris{} expression
\item $\MO{Elab}$, which processes a top level \Idris{} declaration by generating
one or more \TT{} declarations.
\end{itemize}


\subsection{The Development Calculus \TTdev}

We build \TT{} expressions by using high level \Idris{} expressions to
direct a tactic based theorem prover, which builds the \TT{} expressions
step by step, by refinement. In order to build expressions in this way,
the type theory needs to support
\remph{incomplete} terms, and a method for term construction. 
To achieve this, we extend \TT{} with \remph{holes},
calling the extended calculus \TTdev{}.
Holes stand for the parts of programs which have not yet been
instantiated; this largely follows the \Oleg{} development
calculus~\cite{McBride1999}.

The basic idea is to extend the syntax for binders with a \remph{hole}
binding and a \remph{guess} binding. 
These extensions are given in Figure \ref{ttdev}.
The \remph{guess} binding is
similar to a $\LET$ binding, but without any computational force,
i.e. there are no reduction rules for guess bindings. 
Using binders to represent holes is useful in a dependently typed setting since
one value may determine another. Attaching a guess to a binder ensures that
instantiating one such value also instantiates all of its dependencies. The
typing rules for binders ensure that no $?$ bindings leak into types.

\FFIG{
\AR{
\vb ::= \ldots 
 \:\mid\: \hole{\vx}{\vt} \:\:(\mbox{hole binding}) \:\:
 \:\mid\: \guess{\vx}{\vt}{\vt} \:\:(\mbox{guess})
\medskip\\
\Rule{
\Gamma;\hole{\vx}{\vS}\proves\ve\Hab\vT
}
{
\Gamma\proves\hole{\vx}{\vS}\SC\ve\Hab\vT
}
\hspace*{0.1cm}\vx\not\in\vT
\hspace*{0.1in}\mathsf{Hole}
\hg
\Rule{
\Gamma;\guess{\vx}{\vS}{\ve_1}\proves\ve_2\Hab\vT
}
{
\Gamma\proves\guess{\vx}{\vS}{\ve_1}\SC\ve_2\Hab\vT
}
\hspace*{0.1cm}\vx\not\in\vT
\hspace*{0.1in}\mathsf{Guess}
}
}
{\TTdev{} extensions}
{ttdev}


\subsection{Proof State}

A proof state is a tuple, $(\vC, \Delta, \ve, \vQ)$, containing:

\begin{itemize}
\item A global context, $\vC$, containing pattern matching definitions and their types
\item A local context, $\Delta$, containing pattern bindings
\item A proof term, $\ve$, in \TTdev{}
\item A hole queue, $\vQ$
%\item \remph{Deferred} definitions, $\vD$, for introducing global metavariables
\end{itemize}

The \remph{hole queue} is a list of names of hole and guess binders 
$\langle\vx_1,\vx_2,\ldots,\vx_n\rangle$
in the proof term ---
we ensure that each bound name is unique. Holes essentially refer to \remph{sub goals}
in the proof.
When this queue is empty, the proof term is complete.
Creating a \TT{} expression from an \Idris{} expresson involves creating
a new proof state, with an empty proof term, and using the high level definition
to direct the building of a final proof state, with a complete proof term.

In the implementation, the proof state is captured in an elaboration monad,
\texttt{Elab}, which includes various operations for querying and updating
the proof state, manipulating terms, generating fresh names, etc. However, we will
describe \Idris{} elaboration in terms of meta-operations on the proof state,
in order to capture the essence of the elaboration process without being distracted
by implementation details. These meta-operations include: 

\begin{itemize}
\item \demph{Queries} which retrieve values from the proof state, without modifying
the state. For example, we can:
\begin{itemize}
\item Get the type of the current sub goal
\item Retrieve the local context $\Gamma$ at the current sub goal
\item Type check or normalise a term relative to $\Gamma$
\end{itemize}
\item \demph{Unification}, which unifies two terms (potentially solving sub goals) 
relative to $\Gamma$
\item \demph{Tactics} which update the proof term. Tactics operate on the sub term
at the binder specified by the head of the hole queue $\vQ$.
\item \demph{Focussing} on a specific sub goal, which brings a different sub goal to the
head of the hole queue.
%\item \demph{Deferring} a sub goal, which adds a new definition to the global context
%$\vC$ which solves the sub goal.
\end{itemize}

Elaboration of an \Idris{} expression involves creating a new proof state, running
a series of tactics to build a complete proof term, then retrieving and \remph{rechecking}
the final proof term, which must be a \TT{} program (i.e. does not contain any of the
\TTdev{} extensions). We call a sub-term which contains no hole or guess bindings 
\demph{pure}. Although a pure term does not containg hole or guess bindings, it may
neverthless \remph{refer} to hole- or guess-bound variables.

We initialise a proof state with the $\MO{NewProof}$ operation. Given a global
context $\vC$, $\MO{NewProof}\:\vt$ sets up the proof state as:

\DM{
(\vC, \cdot, \hole{\vx}{\vt}\SC\vx, \langle\vx\rangle)
}

The local context is initially empty, and the initial hole queue is the $\vx$ standing for
the entire expression. We can reset the proof term with the $\MO{NewTerm}$ operation.
In an existing proof state $(\vC, \Delta, \ve, \vQ)$,
$\MO{NewTerm}\:\vt$ discards the proof term and hole queue, and
updates the proof state to:

\DM{
(\vC, \Delta, \hole{\vx}{\vt}\SC\vx, \langle\vx\rangle)
}


This allows us in particular to use pattern bindings from the left hand side of a pattern
matching definition in the term on the right hand side.

\subsection{System State}

The system state is a tuple, $(\vC,\vA,\vI)$, containing:

\begin{itemize}
\item A global context, $\vC$, containing pattern matching definitions and their types
\item Implicit arguments, $\vA$, recording which arguments are implicit for each global name
\item Type class instances, $\vI$, containing dictionaries for type classes
\end{itemize}

In the implementation, the system state is captured in a monad, \texttt{Idris}, and
includes additional information such as syntax overloadings,
command line options, and optimisations, which do not concern us here. Elaboration
of expressions requires access to the system state in particular in order to expand
implicit arguments and resolve type classes. 

For each global name, $\vA$ records whether its arguments are explicit, implicit,
or type class constraints.  For example, recall the declaration
of \texttt{vAdd}:

\begin{SaveVerbatim}{vAddImpT}

vAdd : Num a => Vect a n -> Vect a n -> Vect a n

\end{SaveVerbatim}
\useverb{vAddImpT} 

\noindent
Written in full, and giving each argument an explicit name, we get the
type declaration:

\begin{SaveVerbatim}{vAddImpT}

vAdd : (a : _) -> (n : _) -> (c : Num a) -> 
       (xs : Vect a n) -> (ys : Vect a n) -> Vect a n

\end{SaveVerbatim}
\useverb{vAddImpT} 

\noindent
For \tFN{vAdd}, we record that \texttt{a} and \texttt{n} are implicit, 
\texttt{c} is a constraint, and \texttt{xs} and \texttt{ys} are explicit. When
the elaborator encounters an application of \tFN{vAdd}, it knows that unless these arguments
are given explicitly, the application must be expanded.

\newcommand{\Check}{\MO{Check}_\Gamma}
\newcommand{\Eval}{\MO{Normalise}_\Gamma}
\newcommand{\Unify}{\MO{Unify}_\Gamma}

\subsection{Tactics}

% Meta-operations Check, Normalise, Unify 
In order to build \TT{} expressions from \Idris{} programs, we define a collection
of meta-operations for querying and modifying the proof state. Meta-operations
may have side-effects including failure, or updating the proof state. We have the following
primitive meta-operations:

\begin{itemize}
\item $\MO{Focus}\:\vn$, which moves $\vn$ to the head of the hole queue.
\item $\Check\:\ve$, which type checks an expression $\ve$ relative to a context
$\Gamma$, returning its type.
$\MO{Check}$ will fail
if the expression is not well-typed.
\item $\Eval\:\ve$, which evaluates a well-typed expression $\ve$ relative to a context 
$\Gamma$, returning its normal form.
\item 
$\Unify\:\ve_1\:\ve_2$, 
which unifies $\ve_1$ and $\ve_2$ by finding the values that holes must be instantiated
with for $\ve_1$ and $\ve_2$ to be convertible relative to $\Gamma$
(i.e. for $\Gamma\proves\ve_1\converts\ve_2$ to hold). $\MO{Unify}$ will fail
if it cannot find such values. If successful, $\MO{Unify}$ will update the proof state.
\end{itemize}

\remph{Tactics} are specifically meta-operations which operate on the sub-term given
by the hole at the head of the hole queue in the proof state. They take the following form:

\DM{
\PA{\A\A}{
\MO{Tactic}_\Gamma & \:\vec{\VV{args}} & \:\vt & \MoRet{\vt'}
}
}

A tactic takes a sequence of zero or more arguments $\vec{\VV{args}}$ followed
by the sub-term $\vt$ on which it is operating. It runs relative to a context
$\Gamma$ which contains all the bindings and pattern bindings in scope at that
point in the term. The sub-term $\vt$ will either be a hole binding
$\hole{\vx}{\vT}\SC\ve$ or a guess binding $\guess{\vx}{\vT}{\vv}\SC\ve$. The
tactic returns a new term $\vt'$ which can take any form, provided it is
well-typed, with a type convertible to the type of $\vt$. 
Tactics may also have the side effect of updating the proof state,
therefore we will describe tactics in a pseudo-code with $\RW{do}$ notation.

We define a set of primitive tactics: $\MO{Create}$, $\MO{Fill}$ and $\MO{Solve}$
which are used to create and destroy holes; and $\MO{Lambda}$, $\MO{Pi}$, $\MO{Let}$
and $\MO{Attack}$ which are used to create binders.

\subsubsection{Creating and destroying holes}

The $\MO{Claim}$ tactic, given a name and a type, adds a new hole binding in
the scope of the current goal $\vx$, adding the new binding to the hole queue, but
keeping $\vx$ at the head:

\DM{
\PA{\A\A}{
\MO{Claim}_\Gamma & (\vy \Hab\vS) & (\hole{\vx}{\vT}\SC\ve) & 
   \MoRet{\RW{return}\:\hole{\vx}{\vT}\SC\hole{\vy}{\vS}\SC\ve} \\
}
}

The $\MO{Fill}$ tactic, given a value $\vv$, attempts to solve the current goal
with $\vv$, creating a guess binding in its place. $\MO{Fill}$ attempts to
solve other holes by unifying the expected type of $\vx$ with the type of $\vv$:

\DM{
\PA{\A\A}{
\MO{Fill}_\Gamma & \vv & (\hole{\vx}{\vT}\SC\ve) & 
   \MoRet{\RW{do}\:\AR{
   \vT' \leftarrow \Check\:\vv\\
   \Unify\:\vT\:\vT'\\
   \RW{return}\:\guess{\vx}{\vT}{\vv'}\SC\ve}
   } \\
}
}

\noindent
For example, consider the following proof term:

\DM{
\AR{
\hole{\vA}{\Set}\SC\hole{\vk}{\Nat}\SC
\hole{\vx}{\vA}\SC\hole{\vxs}{\Vect\:\vA\:\vk}\SC
\\
\hole{\vys}{\Vect\:\vA\:(\suc\:\vk)}\SC\vys
}
}

\noindent
If $\vx$ is in focus (i.e., at the head of the hole queue) and we attempt to
$\MO{Fill}$ it with an $\TC{Int}$ value $42$, we have:

\begin{itemize}
\item $\Check\:42\:\mq\:\TC{Int}$
\item Unifying $\TC{Int}$ with $\vA$ (the type of $\vx$) is only possible if
$\vA\:=\:\TC{Int}$, so we solve $\vA$.
\end{itemize}

\noindent
Therefore the resulting proof term is:

\DM{
\AR{
\hole{\vk}{\Nat}\SC
\guess{\vx}{\TC{Int}}{42}\SC\hole{\vxs}{\Vect\:\TC{Int}\:\vk}\SC
\\
\hole{\vys}{\Vect\:\TC{Int}\:(\suc\:\vk)}\SC\vys
}
}

The $\MO{Solve}$ tactic operates on a guess binding. If the guess is \remph{pure}, i.e., it
is a \TT{} term containing no hole or guess bindings, then the value attached to
the guess is substituted into its scope:

\DM{
\PA{\A}{
\MO{Solve}_\Gamma & (\guess{\vx}{\vT}{\vv}\SC\ve) &
   \MoRet{\RW{return}\:\ve[\vv/\vx]\hg\mbox{(if $\MO{Pure}\:\vv$)}}
}
}

In each of these tactics, if any step fails, or the term in focus does not take
the correct form (e.g. is not a guess in the case of $\MO{Solve}$ or not a hole
in the case of $\MO{Claim}$ and $\MO{Fill}$, the entire tactic fails. We can
handle failure using the $\MO{Try}$ tactic combinator:

\DM{
\PA{\A\A\A}{
\MO{Try}_\Gamma & \VV{t1} & \VV{t2} & \vt &
   \MoRet{\AR{\VV{t1}_\Gamma\:\vt\hg\mbox{(if $\VV{t1}$ succeeds)} \\
              \VV{t2}_\Gamma\:\vt\hg\mbox{(otherwise)}}}
}
}

\subsubsection{Creating binders}

We also define primitive tactics for constructing binders. We can create a $\lambda$
binding if the goal normalises to a function type:

\DM{
\PA{\A\A}{
\MO{Lambda}_\Gamma & \vn & (\hole{\vx}{\vT}\SC\vx) &
 \MoRet{\RW{do}\:\AR{
   \all{\vy}{\vS}\SC\vT'\:\leftarrow\:\Eval\:\vT \\
   \RW{return}\:\lam{\vn}{\vS}\SC\hole{\vx}{\vT'[\vn/\vy]}\SC\vx
   }
   }
}
}

\noindent
We can create a $\forall$ binding if the goal is a $\TC{Set}$:

\DM{
\PA{\A\A}{
\MO{Pi}_\Gamma & (\vn\Hab\vS) & (\hole{\vx}{\Set}\SC\vx) &
 \MoRet{\RW{do}\:\AR{
   \Set\:\leftarrow\:\Check\:\vS\\
   \RW{return}\:\all{\vn}{\vS}\SC\hole{\vx}{\Set}\SC\vx
   }
   }
}
}

\noindent
To create a $\LET$ binding, we give a type and a value.

\DM{
\PA{\A\A}{
\MO{Let}_\Gamma & (\vn\Hab\vS\defq\vv) & (\hole{\vx}{\vT}\SC\vx) &
 \MoRet{\RW{do}\:\AR{
   \Set\:\leftarrow\:\Check\:\vS\\
   \vS'\:\leftarrow\:\Check\:\vv\\
   \Unify\:\vS\:\vS'\\
   \RW{return}\:\LET\:\vn\Hab\vS\defq\vv\SC\hole{\vx}{\vT}\SC\vx
   }
   }
}
}

Each of these tactics require the term in focus to be of the form $\hole{\vx}{\vT}\SC\vx$.
This is important, because if the scope of the binding was an arbitrary expression $\ve$,
the binder would be scoped across this whole expression rather than the subexpression
$\vx$ as intended.
The $\MO{attack}$ tactic ensures that a hole
is in the appropriate form, creating a new hole $\vh$ which is placed at the head
of the queue:

\DM{
\PA{\A}{
\MO{Attack}_\Gamma & (\hole{\vx}{\vT}\SC\ve) &
 \MoRet{\RW{return}\:\guess{\vx}{\vT}{(\hole{\vh}{\vT}\SC\vh)}\SC\ve}
}
}

Finally, we can convert a hole binding to a pattern binding by giving the 
pattern variable a name. This solves a hole
by adding the pattern binding to the proof state, and updating the proof term
with the pattern variable directly:

\DM{
\PA{\A\A}{
\MO{Pat}_\Gamma & \vn & (\hole{\vx}{\vT}\SC\ve) &
  \MoRet{\RW{do}\:\AR{
    \MO{PatBind}\:(\vx\Hab\vT)\\
    \RW{return}\:\ve[\vn/\vx]
  }}
}
}

The $\MO{PatBind}$ operation simply updates the proof state with the given pattern
binding. Once we have created bindings from the left hand side of a pattern
matching definition, for example, we can retain these bindings for use when
building the right hand side.

\subsubsection{Example}

Tactics are executed by a higher level meta-operation $\MO{RunTac}$, which
locates the appropriate sub-term, applies the tactic with the context
local to this sub-term, and
replaces the sub-term with the term returned by the
tactic. It then updates the hole queue in the proof state, and updates holes which have
been solved by unification. If the tactic creates new holes, these are automatically
added to the \remph{head} of the hole queue.
For example, consider the following simple
\TT{} definition for the identify function:

\DM{
\AR{
\FN{id}\Hab\all{\vA}{\Set}\SC\all{\va}{\vA}\SC\vA \\
\FN{id}\:=\:\lam{\vA}{\Set}\SC\lam{\va}{\vA}\SC\va  
}
}

We can build $\FN{id}$ either as a complete term, or by applying a sequence of tactics.
To achieve this, we create a proof state initialised with the type of $\FN{id}$ and
apply a series of $\MO{Lambda}$ and $\MO{Fill}$ operations using $\MO{RunTac}$:

\DM{
\AR{
\MO{MkId}\:\mq\:\RW{do}\:
 \AR{
   \MO{NewProof}\:\all{\vA}{\Set}\SC\all{\va}{\vA}\SC\vA \\
   \MO{RunTac}\;\MO{Attack} \\
   \MO{RunTac}\;(\MO{Lambda}\;\vA) \\
   \MO{RunTac}\;\MO{Attack} \\
   \MO{RunTac}\;(\MO{Lambda}\;\va) \\
   \MO{RunTac}\;(\MO{Fill}\;\va)\\ 
   \MO{RunTac}\;\MO{Solve}\\
   \MO{RunTac}\;\MO{Solve}\\
   \MO{RunTac}\;\MO{Solve}\\
 }
}
}

To aid readability, we will elide $\MO{RunTac}$, and use a semi-colon to indicate
sequencing. Using this convention, we can build $\FN{id}$'s type and definition as shown
in Figure \ref{idelab}. Note that $\MO{Term}$ retrieves the proof term from the current proof
state. Both $\MO{MkIdType}$ and $\MO{MkId}$ finish by returning a completed \TT{} term.
Note in particular that each $\MO{Attack}$ and each $\MO{Fill}$, which create new guesses,
are closed with a $\MO{Solve}$.

Setting up elaboration in this way, with a proof state captured in a monad,
and a primitive collection of tactics,
makes it easy to derive more complex tactics for elaborating higher level language constructs,
in much the same way as the \texttt{Ltac} language in Coq. As a result, elaboration of
a language construct (or a program such as $\FN{id}$) bears a strong resemblance to
a Coq proof script.

\FFIG{
\AR{
\MO{MkIdType}\:\mq\:\RW{do}\;
 \AR{
   \MO{NewProof}\:\Set\\
   \MO{Attack} ; \MO{Pi}\:(\vA\Hab\Set) ;
   \MO{Attack} ; \MO{Pi}\:(\va\Hab\vA) \\
   \MO{Fill}\;\vA \\
   \MO{Solve} ; \MO{Solve} ; \MO{Solve} \\
   \MO{Term}
 }
 \medskip\\
\MO{MkId}\:\mq\:\RW{do}\;
 \AR{
   \vt\:\leftarrow\:\MO{MkIdType}; \MO{NewProof}\:\vt\\
   \MO{Attack} ; \MO{Lambda}\;\vA ; 
   \MO{Attack} ; \MO{Lambda}\;\va \\
   \MO{Fill}\;\va \\
   \MO{Solve} ; \MO{Solve} ; \MO{Solve}\\
   \MO{Term}
 }
}
}
{Building $\FN{id}$ with tactics}
{idelab}

%--- give unify in full, esp. as it solves sub goals? Maybe...

% Unify' G x t             = Success (x, t) if ?x : t in G
% Unify' G t x             = Success (x, t) if ?x : t in G
% Unify' G (b x. e) (b' x'. e')   = Unify' G b b'; Unify' G;b e e'[x/x']
% Unify' G ((\x.e) x) e'   = Unify' G e e' 
% Unify' G e ((\x.e') x)   = Unify' G e e' 
% Unify' G (f es) (f' es') = vs <- Unify' G f f'; Injective f
                             
% Unify' G x y             = Success () if G |- x == y
% Unify' G . .             = Failure

% Unify' G (\x : t . e) (\x : t' . e') = Unify' G t t'; Unify' G e e'
% ...


%\DM{
%}

\subsection{Elaborating Expressions}


\IdrisM{}, a subset of \Idris{} not including syntactic sugar (e.g. pairs, do notation, etc).

Implicit and type class arguments? Expanded at the application site (we need to know
it's the global name after all and we do that by type).

\subsection{Elaborating Data Types}

\subsection{Elaborating Pattern Matching}

