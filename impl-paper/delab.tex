\section{Reversing Elaboration}

\label{sect:delab}

As well as translating from \Idris{} to \TT{}, so that programs can be type
checked and evaluated, it is valuable to define the reverse transformation. This
serves two principal purposes:

\begin{itemize}
\item To assist the user, it is preferable that the results of evaluation, and any
error messages produced by the elaborator, are presented in \Idris{} syntax
rather than \TT{}.
\item For correctness, we would like to ensure as far as possible that the 
result of elaboration is equivalent to the original program. Informally, we can
achieve this by checking that reversing the elaboration process yields the original
program (with implicit arguments expanded).
\end{itemize}

\noindent
In this section, we describe the process for reversing elaboration and the required
properties of the elaboration process as a whole. Fortunately, translating from
\TT{} to \Idris{} is significantly easier than \Idris{} to \TT{}, because it is
primarily \emph{erasing} information.

\subsection{From \TT{} to \Idris{}}

We define a meta-operation $\uninterp{\vt}$, which converts a \TT{} expression
$\vt$ to an \Idris{} expression which would elaborate to $\vt$:

\DM{
\AR{
\begin{array}{rl}
\uninterp{\vx} &\mq\:\vx\\
\uninterp{\vc} &\mq\:\vc\\
\uninterp{\vx\;\ta} &\mq\:\vx\;(\vec{\MO{Impl}}\:\vx\:\ta) \\
\uninterp{\vf\:\va} &\mq\:\uninterp{\vf}\:\uninterp{\va}\\
\uninterp{\lam{\vx}{\vT}\SC\ve} &\mq \:\ilam{\vx}\uninterp{\ve}  \\
\uninterp{\all{\vx}{\vT}\SC\ve} &\mq \:\piexp{\vx}{\uninterp{\vT}}\uninterp{\ve}  \\
\uninterp{\LET\:\vx\defq\vt\Hab\vT\SC\ve} &\mq  
\ilet{\vx}{\uninterp{\vt}}\uninterp{\ve}\\
\end{array}
\medskip\\
\begin{array}{rll}
\MO{Impl}\:\vx\:\va_i\:\mq
&
\iarg{\vn}{\ttinterp{\va_i}}&
\mbox{(if the $i$th argument to $\vx$ is implicit argument $\vn$)}\\
& \carg{\ttinterp{\va_i}} &
\mbox{(if the $i$th argument to $\vx$ is a constraint argument)}\\
& \ttinterp{\va_i} & \mbox{(otherwise)}
\end{array}
}
}

\noindent
This is mostly a straightforward
traversal of the \TT{} expression, translating directly to an \Idris{} equivalent.
The interesting case is for applications of named functions,
$\uninterp{\vx\:\ta}$, where the arguments are translated to either implicit,
constraint or explicit arguments according to the definition of $\vx$.
Since only type declarations are allowed to have implicit or constraint arguments,
and $\uninterp{\cdot}$ translates \emph{expressions},
all function types are assumed to take explicit arguments.

We also define an operation $\MO{Unelab}$, which translates \TT{} declarations
to corresponding \Idris{} declarations. This generates data declarations and
pattern matching definitions only --- it makes no attempt to reconstruct 
\texttt{class}
or \texttt{instance} declarations, or rebuild \texttt{case} expressions.
First, we define the reverse elaboration of type declarations, which must reconstruct
which arguments are implicit or constraint arguments:

\DM{
\AR{
\begin{array}{l}
\MO{UnelabType}\:(\vx\Hab\vt)\:\mq\:\vx\Hab\MO{UnelabTyDecl}\:0\:\vt
\end{array}
\medskip\\
\begin{array}{ll}
\MO{UnelabTyDecl}\:\vi\:(\all{\vx}{\vT}\SC\ve)
\mq & \piimp{\vx}{\uninterp{\vT}}(\MO{UnelabTyDecl}\:(\vi+1)\:\ve) \\
 & \hg\mbox{(if the $i$th argument to $\vx$ is an implicit argument)} \\
 & \piconst{\uninterp{\vT}}(\MO{UnelabTyDecl}\:(\vi+1)\:\ve) \\
 & \hg\mbox{(if the $i$th argument to $\vx$ is a constraint argument)} \\
 & \piexp{\vx}{\uninterp{\vT}}(\MO{UnelabTyDecl}\:(\vi+1)\:\ve) \\
 & \hg\mbox{(otherwise)} \\
\end{array}
}
}

\noindent
Using this, we define $\MO{Unelab}$ for top level declarations. For pattern
matching clauses, we reverse elaborate as follows, discarding the explicit
pattern variable bindings and applying $\MO{UnelabType}$ to reconstruct
the type declaration:

\DM{
\AR{
\MO{Unelab}\:(\vx\Hab\vt)\:\mq\:\MO{UnelabType}\:(\vx\Hab\vt)\\
\MO{Unelab}\:(\pat{\tx}{\tU}\SC\FN{f}\:\tts\:=\:\ve)
\:\mq\:\uninterp{\FN{f}\:\tts}\:=\:\uninterp{\ve}
}
}

\noindent
For data type declarations, we unelaborate as follows, applying $\MO{UnelabType}$
for each of the top level type declarations:

\DM{
\AR{
\MO{Unelab}\:(\Data\;\TC{T}\:(\tx\Hab\ttt)\Hab\vT\;\Where\;\vec{\VV{cons}})
\:\\
\hg\hg\mq\:\idata\:\MO{UnelabType}\:(\TC{T}\Hab\all{\tx}{\ttt}\SC\vT)\:\iwhere\:
(\vec{\MO{UnelabType}}\:\vec{\VV{cons}})
}
}


\subsection{Elaboration Properties}

\todo[inline]{
What are the properties of elaboration?
Need to define unelaboration, and say that elaborating then unelaborating
an expression yields the original expression.
(Works for expressions but not declarations)
}

Properties:
\begin{itemize}
\item Elaboration produces a well-typed term
\item $\ve \MO{Matches} \uninterp{\ttinterp{\ve}}$
\end{itemize}


