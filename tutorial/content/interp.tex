\section{Example: The Well-Typed Interpreter}
\label{sect:interp}

In this section, we'll use the features we've seen so far to write a larger
example, an interpreter for a simple functional programming language, with
variables, function application, binary operators and an \texttt{if...then...else}
construct. We will use the dependent type system to ensure that any programs
which can be represented are well-typed. First, let us define the types in the
language. We have integers, booleans, and functions, represented by \texttt{Ty}:

\begin{code}
data Ty = TyInt | TyBool | TyFun Ty Ty
\end{code}

\noindent
We can write a function to translate these representations to a concrete \Idris{}
type --- remember that types are first class, so can be calculated just like
any other value:

\begin{code}
interpTy : Ty -> Type
interpTy TyInt       = Int
interpTy TyBool      = Bool
interpTy (TyFun A T) = interpTy A -> interpTy T
\end{code}

\noindent
We're going to define a representation of our language in such a way that only
well-typed programs can be represented. We'll index the representations of
expressions by their type and the types of local variables (the context), which
we'll be using regularly as an implicit argument, so we define everything
in a \texttt{using} block:

\begin{code}
using (G:Vect n Ty)
\end{code}

\noindent
The full representation of expressions is given in Figure \ref{exprty}. They are
indexed by the types of the local variables, and the type of the expression itself:

\begin{code}
data Expr : Vect n Ty -> Ty -> Type
\end{code}

\noindent
Since expressions are indexed by their type, we can read the typing rules of
the language from the definitions of the constructors. Let us look at each
constructor in turn.

\begin{code}[caption={Expression representation},label=exprty]
data HasType : (i : Fin n) -> Vect n Ty -> Ty -> Type where
    stop : HasType fZ (t :: G) t
    pop  : HasType k G t -> HasType (fS k) (u :: G) t

data Expr : Vect n Ty -> Ty -> Type where
    Var : HasType i G t -> Expr G t
    Val : (x : Int) -> Expr G TyInt
    Lam : Expr (a :: G) t -> Expr G (TyFun a t)
    App : Expr G (TyFun a t) -> Expr G a -> Expr G t
    Op  : (interpTy a -> interpTy b -> interpTy c) -> Expr G a -> Expr G b -> 
          Expr G c
    If  : Expr G TyBool -> Expr G a -> Expr G a -> Expr G a
\end{code}

\noindent
We use a nameless representation for variables --- they are \emph{de Bruijn indexed}.
Variables are represented by a proof of their membership in the context, 
\texttt{HasType i G T}, which is a proof that variable \texttt{i} in context 
\texttt{G} has type \texttt{T}. This is defined as follows:

\begin{code}
data HasType : (i : Fin n) -> Vect n Ty -> Ty -> Type where
    stop : HasType fZ (t :: G) t
    pop  : HasType k G t -> HasType (fS k) (u :: G) t
\end{code}

\noindent
We can treat \remph{stop} as a proof that the most recently defined variable is well-typed,
and \remph{pop n} as a proof that, if the \texttt{n}th most recently defined variable is
well-typed, so is the \texttt{n+1}th. In practice, this means we use \texttt{stop} to
refer to the most recently defined variable, \texttt{pop stop} to refer to the next, and so on,
via the \texttt{Var} constructor:

\begin{code}
Var : HasType i G t -> Expr G t
\end{code}

\noindent
So, in an expression \texttt{$\backslash$x. $\backslash$y. x y}, 
the variable \texttt{x} would have a de Bruijn index
of 1, represented as \texttt{pop stop}, and \texttt{y 0}, represented
as \texttt{stop}. We find these by counting the number of lambdas between the
definition and the use.

\noindent
A value carries a concrete representation of an integer: 

\begin{code}
Val : (x : Int) -> Expr G TyInt
\end{code}

\noindent
A lambda creates a function. In the scope of a function of type \texttt{a -> t}, there is
a new local variable of type \texttt{a}, which is expressed by the context index:

\begin{code}
Lam : Expr (a :: G) t -> Expr G (TyFun a t)
\end{code}

\noindent
Function application produces a value of type \texttt{t} given a function from 
\texttt{a} to \texttt{t}
and a value of type \texttt{a}:

\begin{code}
App : Expr G (TyFun a t) -> Expr G a -> Expr G t
\end{code}

\noindent
We allow arbitrary binary operators, where the type of the operator
informs what the types of the arguments must be:

\begin{code}
Op  : (interpTy a -> interpTy b -> interpTy c) -> Expr G a -> Expr G b -> 
      Expr G c
\end{code}

\noindent
Finally, if expressions make a choice given a boolean. Each branch must
have the same type:

\begin{code}
If  : Expr G TyBool -> Expr G a -> Expr G a -> Expr G a
\end{code}

\noindent
When we evaluate an \texttt{Expr}, we'll need to know the values in scope, as well as
their types. \texttt{Env} is an environment, indexed over the types in scope.
Since an environment is just another form of list, albeit with a strongly specified connection
to the vector of local variable types, we use the usual \texttt{::} and \texttt{Nil}
constructors so that we can use the usual list syntax. Given a proof that a variable
is defined in the context, we can then produce a value from the environment:

\begin{code}
data Env : Vect n Ty -> Type where
    Nil  : Env Nil
    (::) : interpTy a -> Env G -> Env (a :: G)

lookup : HasType i G t -> Env G -> interpTy t
lookup stop    (x :: xs) = x
lookup (pop k) (x :: xs) = lookup k xs
\end{code}

\begin{code}[caption={Intepreter definition},label=interpdef]
interp : Env G -> Expr G t -> interpTy t
interp env (Var i)     = lookup i env
interp env (Val x)     = x
interp env (Lam sc)    = \x => interp (x :: env) sc
interp env (App f s)   = interp env f (interp env s)
interp env (Op op x y) = op (interp env x) (interp env y)
interp env (If x t e)  = if interp env x then interp env t 
                                         else interp env e
\end{code}

\noindent
Given this, an interpreter (Figure \ref{interpdef})
is a function which translates an \texttt{Expr} into a
concrete \Idris{} value with respect to a specific environment:

\begin{code}
interp : Env G -> Expr G t -> interpTy t
\end{code}

\noindent
To translate a variable, we simply look it up in the environment:

\begin{code}
interp env (Var i) = lookup i env
\end{code}

\noindent
To translate a value, we just return the concrete representation of the value:

\begin{code}
interp env (Val x) = x
\end{code}

\noindent
Lambdas are more interesting. In this case, we construct a function which
interprets the scope of the lambda with a new value in the environment. So, a
function in the object language is translated to an \Idris{} function:

\begin{code}
interp env (Lam sc) = \x => interp (x :: env) sc
\end{code}

\noindent
For an application, we interpret the function and its argument and apply it
directly. We know that interpreting \texttt{f} must produce a function, because of its
type:

\begin{code}
interp env (App f s) = interp env f (interp env s)
\end{code}

\noindent
Operators and interpreters are, again, direct translations into the equivalent
\Idris{} constructs. For operators, we apply the function to its operands directly,
and for \texttt{If}, we apply the \Idris{} \texttt{if...then...else} construct
directly.

\begin{code}
interp env (Op op x y) = op (interp env x) (interp env y)
interp env (If x t e)  = if interp env x then interp env t 
                                         else interp env e
\end{code}

\noindent
We can make some simple test functions. Firstly, adding two inputs
\texttt{$\backslash$x. $\backslash$y. y + x} is written as follows:

\begin{code}
add : Expr G (TyFun TyInt (TyFun TyInt TyInt))
add = Lam (Lam (Op (+) (Var stop) (Var (pop stop))))
\end{code}

\noindent
More interestingly, we can write a factorial function. First, we write a \emph{lazy}
version of the \texttt{App} constructor, so that the recursive branch will only
be evaluated if necessary:

\begin{code} 
app : |(f : Expr G (TyFun a t)) -> Expr G a -> Expr G t
app = \f, a => App f a
\end{code}

\noindent
The vertical bar annotation before the \texttt{f} argument indicates that this
argument should be evaluated lazily when \texttt{app} is applied --- that is, the
argument will be evaluated only when needed.
Then \texttt{fact} 
(i.e. \texttt{$\backslash$x. if (x == 0) then 1 else (fact (x-1) * x)})
is written as follows:

\begin{code}
fact : Expr G (TyFun TyInt TyInt)
fact = Lam (If (Op (==) (Var stop) (Val 0))
               (Val 1) (Op (*) (app fact (Op (-) (Var stop) (Val 1))) 
                               (Var stop)))
\end{code}

To finish, we write a \texttt{main} program which interprets the factorial function
on user input:

\begin{code}
main : IO ()
main = do putStr "Enter a number: "
          x <- getLine
          print (interp [] fact (cast x)) 
\end{code}

\noindent
Here, \texttt{cast} is an overloaded function which converts a value from one
type to another if possible. Here, it converts a string to
an integer, giving 0 if the input is invalid. An example run of this program
at the \Idris{} interactive environment is shown in Figure \ref{factrun}.

\begin{lstlisting}[caption={Running the well-typed interpreter}, label=factrun, escapechar=^]
$ idris interp.idr  
     ____    __     _                                          
    /  _/___/ /____(_)____                                     
    / // __  / ___/ / ___/     Version ^\version{}^
  _/ // /_/ / /  / (__  )      http://www.idris-lang.org/      
 /___/\__,_/_/  /_/____/       Type :? for help                

Type checking ./interp.idr
*interp> :exec 
Enter a number: 6 
720
*interp> 
\end{lstlisting}

\subsubsection*{Aside: \texttt{cast}}

The prelude defines a type class \texttt{Cast} which allows conversion between
types:

\begin{code}
class Cast from to where
    cast : from -> to
\end{code} 

\noindent
It is a \emph{multi-parameter} type class, defining the source type and object
type of the cast. It must be possible for the type checker to infer \remph{both}
parameters at the point where the cast is applied.
There are casts defined between all of the primitive types, as far as
they make sense.

\subsection{Unit testing}

Recall the \texttt{so} data type:

\begin{code}
data so : Bool -> Type where
  oh : so True
\end{code}

\noindent
This requires its parameter to be a \texttt{True} value. One simple application of this
is to perform unit testing of functions at compile time. For example, we know that 
\texttt{4! = 24} so we can set up a test case as follows:

\begin{code}
unitTestFac : so (interp [] fact 4 == 24)
unitTestFac = oh
\end{code} 

\noindent
For the program to compile, the test case must pass. If we change \texttt{24} to \texttt{25},
say, we will get an error message like the following:

\begin{lstlisting}
Type checking ./interp.idr
interp.idr:64:Can't unify so True with so False

Specifically:
	 Can't unify True with False
\end{lstlisting} 


