\section{Miscellany}

In this section we discuss a variety of additional features: extensible syntax,
auto implicit and default 
arguments, literate programming, interfacing with external libraries through the
foriegn function interface, and the universe hierarchy.

\section{Syntax Extensions}

\Idris{} supports the implementation of Embedded Domain Specific Languages (EDSLs) in
several ways~\cite{res-dsl-padl12}. One way, as we have already seen, is through
extending \texttt{do} notation. Another important way is to allow extension of the core
syntax. In this section we describe two ways of extending the syntax: \texttt{syntax}
rules and \texttt{dsl} notation.

\subsection{\texttt{syntax} rules}

We have seen \texttt{if...then...else} expressions, but these
are not built in --- instead, we define a function in the prelude\ldots

\begin{SaveVerbatim}{boolelim}

boolElim : (x:Bool) -> |(t : a) -> |(f : a) -> a; 
boolElim True  t e = t;
boolElim False t e = e;

\end{SaveVerbatim}
\useverb{boolelim}

\noindent
\ldots and extend the core syntax with a \texttt{syntax} declaration:

\begin{SaveVerbatim}{syntaxif}

syntax if [test] then [t] else [e] = boolElim test t e;

\end{SaveVerbatim}
\useverb{syntaxif}

\noindent
The left hand side of a \texttt{syntax} declaration describes the syntax rule, and the right
hand side describes its expansion. The syntax rule itself consists of:

\begin{itemize}
\item \textbf{Keywords} --- here, \texttt{if}, \texttt{then} and \texttt{else}, which must
be valid identifiers
\item \textbf{Non-terminals} --- included in square brackets, \texttt{[test]}, \texttt{[t]}
and \texttt{[e]} here, which stand for arbitrary expressions. To avoid parsing ambiguities, 
these expressions cannot use syntax extensions at the top level (though they can be used
in parentheses).
\item \textbf{Names} --- included in braces, which stand for names which may be bound
on the right hand side.
\item \textbf{Symbols} --- included in quotations marks, e.g. \texttt{":="}. This can
also be used to include reserved words in syntax rules, such as \texttt{"let"} or \texttt{"in"}.
\end{itemize}

\noindent
The limitations on the form of a syntax rule are that it must include at least one
symbol or keyword, and there must be no repeated variables standing for non-terminals.
Rules can use previously defined rules, but may not be recursive.
The following syntax extensions would therefore be valid:

\begin{SaveVerbatim}{syntaxex}

syntax [var] ":=" [val]              = Assign var val;
syntax [test] "?" [t] ":" [e]        = if test then t else e;
syntax select [x] from [t] where [w] = SelectWhere x t w;
syntax select [x] from [t]           = Select x t;

\end{SaveVerbatim}
\useverb{syntaxex}

\noindent
Syntax macros can be further restricted to apply only in patterns (i.e., only on the left
hand side of a pattern match clause) or only in terms (i.e. everywhere but the left hand side
of a pattern match clause) by being marked as \texttt{pattern} or \texttt{term} syntax
rules. For example, we might define an interval as follows, with a static check
that the lower bound is below the upper bound using \texttt{so}:

\begin{SaveVerbatim}{interval}

data Interval : Type where
   MkInterval : (lower : Float) -> (upper : Float) -> 
                so (lower < upper) -> Interval

\end{SaveVerbatim}
\useverb{interval}

\noindent
We can define a syntax which, in patterns, always matches \texttt{oh} for the proof 
argument, and in terms requires a proof term to be provided:

\begin{SaveVerbatim}{intervalsyn}

pattern syntax "[" [x] "..." [y] "]" = MkInterval x y oh
term    syntax "[" [x] "..." [y] "]" = MkInterval x y ?bounds_lemma

\end{SaveVerbatim}
\useverb{intervalsyn} 

\noindent
In terms, the syntax \texttt{[x...y]} will generate a proof obligation
\texttt{bounds\_lemma} (possibly renamed).

Finally, syntax rules may be used to introduce alternative binding forms. For
exampe, a \texttt{for} loop binds a variable on each iteration:

\begin{SaveVerbatim}{forloop}

syntax for {x} "in" [xs] [body] = forLoop xs (\x => body)
  
main : IO ()
main = do for x in [1..10] do
              putStrLn ("Number " ++ show x)
          putStrLn "Done!"

\end{SaveVerbatim}
\useverb{forloop} 

\noindent
Note that we have used the \texttt{\{x\}} form to state that \texttt{x} represents
a bound variable, substituted on the right hand side. We have also put \texttt{"in"} in
quotation marks since it is already a reserved word.

\subsection{\texttt{dsl} notation}

The well-typed interpreter in Section \ref{sect:interp} is a simple example of
a common programming pattern with dependent types, namely: describe an
\emph{object language}
and its type system with dependent types to guarantee that only well-typed programs
can be represented, then program using that representation. Using this approach
we can, for example, write programs for serialising binary data~\cite{plpv11} or
running concurrent processes safely~\cite{cbconc-fi}.

Unfortunately, the form of object language programs makes it rather hard to program
this way in practice. Recall the factorial program in \texttt{Expr} for example:

\useverb{facttest}

\noindent
Since this is a particularly useful pattern, \Idris{} provides syntax
overloading~\cite{res-dsl-padl12} to make it easier to program in such
object languages:

\begin{SaveVerbatim}{exprdsl}

dsl expr
    lambda      = Lam
    variable    = Var
    index_first = stop
    index_next  = pop

\end{SaveVerbatim}
\useverb{exprdsl} 

\noindent
A \texttt{dsl} block describes how each syntactic construct is represented in an
object language. Here, in the \texttt{expr} language, any \Idris{} lambda is
translated to a \texttt{Lam} constructor; any variable is translated to the
\texttt{Var} constructor, using \texttt{pop} and \texttt{stop} to construct the
de Bruijn index (i.e., to count how many bindings since the variable itself was bound).
It is also possible to overload \texttt{let} in this way. We can now write \texttt{fact}
as follows:

\begin{SaveVerbatim}{factb}

fact : Expr G (TyFun TyInt TyInt)
fact = expr (\x => If (Op (==) x (Val 0))
                      (Val 1) (Op (*) (app fact (Op (-) x (Val 1))) x))

\end{SaveVerbatim}
\useverb{factb} 

\noindent
In this new version, \texttt{expr} declares that the next expression will be overloaded.
We can take this further, using idiom brackets, by declaring:

\begin{SaveVerbatim}{idiomexpr}

(<$>) : |(f : Expr G (TyFun a t)) -> Expr G a -> Expr G t
(<$>) = \f, a => App f a

pure : Expr G a -> Expr G a
pure = id

\end{SaveVerbatim}
\useverb{idiomexpr} 

\noindent
Note that there is no need for these to be part of an instance of \texttt{Applicative},
since idiom bracket notation translates directly to the names \texttt{<\$>} and
\texttt{pure}, and ad-hoc type-directed overloading is allowed. We can now say:

\begin{SaveVerbatim}{factc}

fact : Expr G (TyFun TyInt TyInt)
fact = expr (\x => If (Op (==) x (Val 0))
                      (Val 1) (Op (*) [| fact (Op (-) x (Val 1)) |] x))

\end{SaveVerbatim}
\useverb{factc} 

\noindent
With some more ad-hoc overloading and type class instances, and a new
syntax rule, we can even go as far as:

\begin{SaveVerbatim}{factd}

syntax IF [x] THEN [t] ELSE [e] = If x t e

fact : Expr G (TyFun TyInt TyInt)
fact = expr (\x => IF x == 0 THEN 1 ELSE [| fact (x - 1) |] * x)

\end{SaveVerbatim}
\useverb{factd} 





\subsection{Auto implicit arguments}

We have already seen implicit arguments, which allows arguments to be omitted when
they can be inferred by the type checker, e.g.

\useverb{vlookupimpty}

\noindent
In other situations, it may be possible to infer arguments not by type checking but
by searching the context for an appropriate value, or constructing a proof. For example,
the following definition of \texttt{head} which requires a proof that the list is
non-empty

\begin{SaveVerbatim}{safehead}

isCons : List a -> Bool
isCons [] = False
isCons (x :: xs) = True

head : (xs : List a) -> (isCons xs = True) -> a
head (x :: xs) _ = x

\end{SaveVerbatim}
\useverb{safehead} 

\noindent
If the list is statically known to be non-empty, either because its value is known or
because a proof already exists in the context, the proof can be constructed
automatically. Auto implicit arguments allow this to happen silently. We define
\texttt{head} as follows:

\begin{SaveVerbatim}{headauto}

head : (xs : List a) -> {auto p : isCons xs = True} -> a
head (x :: xs) = x

\end{SaveVerbatim}
\useverb{headauto} 

\noindent
The \texttt{auto} annotation on the implicit argument means that \Idris{} will
attempt to fill in the implicit argument using the \texttt{trivial} tactic, which
searches through the context for a proof, and tries to solve with \texttt{refl}
if a proof is not found.
Now when \texttt{head} is applied, the proof can be omitted. In the case that a proof
is not found, it can be provided explicitly as normal:

\begin{SaveVerbatim}{headapp}

head xs {p = ?headProof} 

\end{SaveVerbatim}
\useverb{headapp} 

\noindent
More generally, we can fill in implicit arguments with a default value by annotating
them with \texttt{default}. The definition above is equivalent to:

\begin{SaveVerbatim}{defimp}

head : (xs : List a) -> 
       {default proof { trivial; } p : isCons xs = True} -> a
head (x :: xs) = x

\end{SaveVerbatim}
\useverb{defimp} 

\subsection{Literate programming}

Like Haskell, \Idris{} supports \emph{literate} programming. If a file has an
extension of \texttt{.lidr} then it is assumed to be a literate file. In literate
programs, everything is assumed to be a comment unless the line begins with a
greater than sign \texttt{>}, for example:

\begin{SaveVerbatim}{litidr}

> module literate

This is a comment. The main program is below

> main : IO ()
> main = putStrLn "Hello literate world!\n"

\end{SaveVerbatim}
\useverb{litidr}

\noindent
An additional restriction is that there must be a blank line between a program
line (beginning with \texttt{>}) and a comment line (beginning with any other
character).

\subsection{Foreign function calls}

For practical programming, it is often necessary to be able to use external libraries,
particularly for interfacing with the operating system, file system, networking, etc.
\Idris{} provides a lightweight foreign function interface for achieving this,
as part of the prelude. For this, we assume a certain amount of knowledge of
C and the \texttt{gcc} compiler. First, we define a datatype which describes the external
types we can handle:

\begin{SaveVerbatim}{foreignty}

data FTy = FInt | FFloat | FChar | FString | FPtr | FUnit

\end{SaveVerbatim}
\useverb{foreignty}

\noindent
Each of these corresponds directly to a C type. Respectively: \texttt{int},
\texttt{float}, \texttt{char}, \texttt{char*}, \texttt{void*} and \texttt{void}.
There is also a translation to a concrete \Idris{} type, described by the
following function:

\begin{SaveVerbatim}{interpfty}

interpFTy : FTy -> Set
interpFTy FInt    = Int
interpFTy FFloat  = Float
interpFTy FChar   = Char
interpFTy FString = String
interpFTy FPtr    = Ptr
interpFTy FUnit   = ()

\end{SaveVerbatim}
\useverb{interpfty}

\noindent
A foreign function is described by a list of input types and a return type, which
can then be converted to an \Idris{} type:

\begin{SaveVerbatim}{ffunty}

ForeignTy : (xs:List FTy) -> (t:FTy) -> Set

\end{SaveVerbatim}
\useverb{ffunty}

\noindent
A foreign function is assumed to be impure, so \texttt{ForeignTy} builds an
\texttt{IO} type, for example:

\begin{SaveVerbatim}{ftyex}

Idris> ForeignTy [FInt, FString] FString
Int -> String -> IO String : Set

Idris> ForeignTy [FInt, FString] FUnit 
Int -> String -> IO () : Set

\end{SaveVerbatim}
\useverb{ftyex}

\noindent
We build a call to a foreign function by giving the name of the function, a list of
argument types and the return type. The built in function \texttt{mkForeign}
converts this description to a function callable by \Idris{}

\begin{SaveVerbatim}{mkForeign}

data Foreign : Set -> Set where
    FFun : String -> (xs:List FTy) -> (t:FTy) -> 
           Foreign (ForeignTy xs t)

mkForeign : Foreign x -> x

\end{SaveVerbatim}
\useverb{mkForeign}

\noindent
For example, the \texttt{putStr} function is implemented as follows, as a call to 
an external function \texttt{putStr} defined in the run-time system:

\begin{SaveVerbatim}{putStrex}

putStr : String -> IO ()
putStr x = mkForeign (FFun "putStr" [FString] FUnit) x
\end{SaveVerbatim}
\useverb{putStrex}

\subsubsection*{Include and linker directives}

Foreign function calls are translated directly to calls to C functions, with appropriate
conversion between the \Idris{} representation of a value and the C representation.
Often this will require extra libraries to be linked in, or extra header and object files.
This is made possible through the following directives:

\begin{itemize}
\item \texttt{\%lib "x"} --- include the \texttt{libx} library, equivalent to passing the
\texttt{-lx} option to \texttt{gcc}.
\item \texttt{\%include "x.h"} --- use the header file \texttt{x.h}.
\item \texttt{\%obj "x.o"} --- link with the object file \texttt{x.o}.
\end{itemize}

\subsection{Cumulativity}

Since values can appear in types and \emph{vice versa}, it is natural that types themselves
have types. For example:

\begin{SaveVerbatim}{typetypes}

*universe> :t Nat
Nat : Set
*universe> :t Vect
Vect : Set -> Nat -> Set

\end{SaveVerbatim}
\useverb{typetypes} 

\noindent
But what about the type of \texttt{Set}? If we ask \Idris{} it reports

\begin{SaveVerbatim}{setset}

*universe> :t Set
Set : Set

\end{SaveVerbatim}
\useverb{setset} 

\noindent
It \emph{appears} that \texttt{Set} is its own type. This would lead to an inconsitency
due to Girard's paradox~\cite{girard-thesis}, so internally there is a \emph{hierarchy}
of types (or \emph{universes}):

\begin{SaveVerbatim}{sethierarchy}

Set : Set 1 : Set 2 : Set 3 : ...

\end{SaveVerbatim}
\useverb{sethierarchy} 

\noindent
Universes are \emph{cumulative}, that is, if \texttt{x : Set n} we can also have that
\texttt{x : Set m}, as long as \texttt{n < m}. 
The typechecker generates such universe 
constraints and reports an error if any inconsistencies are found. Ordinarily, a
programmer does not need to worry about this, but it does prevent (contrived)
programs such as the following:

\begin{SaveVerbatim}{idid}

myid : (a : Set) -> a -> a
myid _ x = x

idid :  (a : Set) -> a -> a
idid = myid _ myid

\end{SaveVerbatim}
\useverb{idid} 

\noindent
The application of \texttt{myid} to itself leads to a cycle in the universe hierarchy
--- \texttt{myid}'s first argument is a \texttt{Set}, which cannot be at a lower level
than required if it is applied to itself.

%\subsection{Comparison}

%How does \Idris{} compare with other dependently typed languages and proof
%assistants, such as Coq, Agda and Epigram?
