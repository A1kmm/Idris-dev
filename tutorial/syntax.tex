\section{Syntax Extensions}

\Idris{} supports the implementation of Embedded Domain Specific Languages (EDSLs) in
several ways~\cite{res-dsl-padl12}. One way, as we have already seen, is through
extending \texttt{do} notation. Another important way is to allow extension of the core
syntax. In this section we describe two ways of extending the syntax: \texttt{syntax}
rules and \texttt{dsl} notation.

\subsection{\texttt{syntax} rules}

We have seen \texttt{if...then...else} expressions, but these
are not built in --- instead, we define a function in the prelude\ldots

\begin{SaveVerbatim}{boolelim}

boolElim : (x:Bool) -> |(t : a) -> |(f : a) -> a; 
boolElim True  t e = t;
boolElim False t e = e;

\end{SaveVerbatim}
\useverb{boolelim}

\noindent
\ldots and extend the core syntax with a \texttt{syntax} declaration:

\begin{SaveVerbatim}{syntaxif}

syntax if [test] then [t] else [e] = boolElim test t e;

\end{SaveVerbatim}
\useverb{syntaxif}

\noindent
The left hand side of a \texttt{syntax} declaration describes the syntax rule, and the right
hand side describes its expansion. The syntax rule itself consists of:

\begin{itemize}
\item \textbf{Keywords} --- here, \texttt{if}, \texttt{then} and \texttt{else}, which must
be valid identifiers
\item \textbf{Non-terminals} --- included in square brackets, \texttt{[test]}, \texttt{[t]}
and \texttt{[e]} here, which stand for arbitrary expressions. To avoid parsing ambiguities, 
these expressions cannot use syntax extensions at the top level (though they can be used
in parentheses).
\item \textbf{Names} --- included in braces, which stand for names which may be bound
on the right hand side.
\item \textbf{Symbols} --- included in quotations marks, e.g. \texttt{":="}. This can
also be used to include reserved words in syntax rules, such as \texttt{"let"} or \texttt{"in"}.
\end{itemize}

\noindent
The limitations on the form of a syntax rule are that it must include at least one
symbol or keyword, and there must be no repeated variables standing for non-terminals.
Rules can use previously defined rules, but may not be recursive.
The following syntax extensions would therefore be valid:

\begin{SaveVerbatim}{syntaxex}

syntax [var] ":=" [val]              = Assign var val;
syntax [test] "?" [t] ":" [e]        = if test then t else e;
syntax select [x] from [t] where [w] = SelectWhere x t w;
syntax select [x] from [t]           = Select x t;

\end{SaveVerbatim}
\useverb{syntaxex}

\noindent
Syntax macros can be further restricted to apply only in patterns (i.e., only on the left
hand side of a pattern match clause) or only in terms (i.e. everywhere but the left hand side
of a pattern match clause) by being marked as \texttt{pattern} or \texttt{term} syntax
rules. For example, we might define an interval as follows, with a static check
that the lower bound is below the upper bound using \texttt{so}:

\begin{SaveVerbatim}{interval}

data Interval : Set where
   MkInterval : (lower : Float) -> (upper : Float) -> 
                so (lower < upper) -> Interval

\end{SaveVerbatim}
\useverb{interval}

\noindent
We can define a syntax which, in patterns, always matches \texttt{oh} for the proof 
argument, and in terms requires a proof term to be provided:

\begin{SaveVerbatim}{intervalsyn}

pattern syntax "[" [x] "..." [y] "]" = MkInterval x y oh
term    syntax "[" [x] "..." [y] "]" = MkInterval x y ?bounds_lemma

\end{SaveVerbatim}
\useverb{intervalsyn} 

\noindent
In terms, the syntax \texttt{[x...y]} will generate a proof obligation
\texttt{bounds\_lemma} (possibly renamed).

Finally, syntax rules may be used to introduce alternative binding forms. For
exampe, a \texttt{for} loop binds a variable on each iteration:

\begin{SaveVerbatim}{forloop}

syntax for {x} "in" [xs] [body] = forLoop xs (\x => body)
  
main : IO ()
main = do for x in [1..10] do
              putStrLn ("Number " ++ show x)
          putStrLn "Done!"

\end{SaveVerbatim}
\useverb{forloop} 

\noindent
Note that we have used the \texttt{\{x\}} form to state that \texttt{x} represents
a bound variable, substituted on the right hand side. We have also put \texttt{"in"} in
quotation marks since it is already a reserved word.

\subsection{\texttt{dsl} notation}

The well-typed interpreter in Section \ref{sect:interp} is a simple example of
a common programming pattern with dependent types, namely: describe an
\emph{object language}
and its type system with dependent types to guarantee that only well-typed programs
can be represented, then program using that representation. Using this approach
we can, for example, write programs for serialising binary data~\cite{plpv11} or
running concurrent processes safely~\cite{cbconc-fi}.

Unfortunately, the form of object language programs makes it rather hard to program
this way in practice. Recall the factorial program in \texttt{Expr} for example:

\useverb{facttest}

\noindent
Since this is a particularly useful pattern, \Idris{} provides syntax
overloading~\cite{res-dsl-padl12} to make it easier to program in such
object languages:

\begin{SaveVerbatim}{exprdsl}

dsl expr
    lambda      = Lam
    variable    = Var
    index_first = stop
    index_next  = pop

\end{SaveVerbatim}
\useverb{exprdsl} 

\noindent
A \texttt{dsl} block describes how each syntactic construct is represented in an
object language. Here, in the \texttt{expr} language, any \Idris{} lambda is
translated to a \texttt{Lam} constructor; any variable is translated to the
\texttt{Var} constructor, using \texttt{pop} and \texttt{stop} to construct the
de Bruijn index (i.e., to count how many bindings since the variable itself was bound).
It is also possible to overload \texttt{let} in this way. We can now write \texttt{fact}
as follows:

\begin{SaveVerbatim}{factb}

fact : Expr G (TyFun TyInt TyInt)
fact = expr (\x => If (Op (==) x (Val 0))
                      (Val 1) (Op (*) (app fact (Op (-) x (Val 1))) x))

\end{SaveVerbatim}
\useverb{factb} 

\noindent
In this new version, \texttt{expr} declares that the next expression will be overloaded.
We can take this further, using idiom brackets, by declaring:

\begin{SaveVerbatim}{idiomexpr}

(<$>) : |(f : Expr G (TyFun a t)) -> Expr G a -> Expr G t
(<$>) = \f, a => App f a

pure : Expr G a -> Expr G a
pure = id

\end{SaveVerbatim}
\useverb{idiomexpr} 

\noindent
Note that there is no need for these to be part of an instance of \texttt{Applicative},
since idiom bracket notation translates directly to the names \texttt{<\$>} and
\texttt{pure}, and ad-hoc type-directed overloading is allowed. We can now say:

\begin{SaveVerbatim}{factc}

fact : Expr G (TyFun TyInt TyInt)
fact = expr (\x => If (Op (==) x (Val 0))
                      (Val 1) (Op (*) [| fact (Op (-) x (Val 1)) |] x))

\end{SaveVerbatim}
\useverb{factc} 

\noindent
With some more ad-hoc overloading and type class instances, and a new
syntax rule, we can even go as far as:

\begin{SaveVerbatim}{factd}

syntax IF [x] THEN [t] ELSE [e] = If x t e

fact : Expr G (TyFun TyInt TyInt)
fact = expr (\x => IF x == 0 THEN 1 ELSE [| fact (x - 1) |] * x)

\end{SaveVerbatim}
\useverb{factd} 



