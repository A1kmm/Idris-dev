\newcommand{\ttinterp}[1]{\mathcal{E}\interp{#1}}
\newcommand{\ttpinterp}[1]{\mathcal{P}\interp{#1}}
\newcommand{\uninterp}[1]{\mathcal{T}\interp{#1}}

\section{Elaborating \Idris{}}

\label{sect:elaboration}

An \Idris{} program consists of a series of declarations --- data types,
functions, classes and instances. In this section, we describe how these
high level declarations are translated into a \TT{} program consisting of
inductive families and pattern matching function definitions. We will need to
work at the \remph{declaration} level, and at the \remph{expression} level,
defining the following meta-operations.

\begin{itemize}
\item $\ttinterp{\cdot}$, which builds a \TT{} expression from an \Idris{} expression.
\item $\MO{Elab}$, which processes a top level \Idris{} declaration by generating
one or more \TT{} declarations.
\item $\MO{TTDecl}$, which adds a top level \TT{} declaration.
\end{itemize}


\subsection{The Development Calculus \TTdev}

\TT{} expressions are built by using high level \Idris{} expressions to
direct a tactic based theorem prover, which builds the \TT{} expressions
step by step, by refinement. In order to build expressions in this way,
the type theory needs to support
\remph{incomplete} terms, and a method for term construction. 
To achieve this, we extend \TT{} with \remph{holes},
calling the extended calculus \TTdev{}.
Holes stand for the parts of programs which have not yet been
instantiated; this largely follows the Oleg development
calculus~\cite{McBride1999}.

The basic idea is to extend the syntax for binders with a \remph{hole}
binding and a \remph{guess} binding. 
These extensions are given in Figure \ref{ttdev}.
The \remph{guess} binding is
similar to a $\LET$ binding, but without any computational force,
i.e. there are no reduction rules for guess bindings. 
Using binders to represent holes is useful in a dependently typed setting since
one value may determine another. Attaching a guess to a binder ensures that
instantiating one such variable also instantiates all of its dependencies. 
%The
%typing rules for binders ensure that no $?$ bindings leak into types.

\FFIG{
\begin{array}{c}
\vb ::= \ldots 
 \;\mid\; \hole{\vx}{\vt} \;\;(\mbox{hole binding}) \;\;
 \;\mid\; \guess{\vx}{\vt}{\vt} \;\;(\mbox{guess})
\medskip\\
\mathsf{HoleVar}\;
\Rule{(\hole{\vx}{\vS})\in\Gamma}
{\Gamma\vdash\vx\Hab\vS}
\hg
\mathsf{GuessVar}\;
\Rule{(\guess{\vx}{\vS}{\vs})\in\Gamma}
{\Gamma\vdash\vx\Hab\vS}
\\
\mathsf{Hole}\;
\Rule{
\Gamma;\hole{\vx}{\vS}\proves\ve\Hab\vT
}
{
\Gamma\proves\hole{\vx}{\vS}\SC\ve\Hab\vT
}
%\hspace*{0.1cm}\vx\not\in\vT
\hg
\mathsf{Guess}\;
\Rule{
\Gamma;\guess{\vx}{\vS}{\ve_1}\proves\ve_2\Hab\vT
}
{
\Gamma\proves\guess{\vx}{\vS}{\ve_1}\SC\ve_2\Hab\vT
}
%\hspace*{0.1cm}\vx\not\in\vT
\end{array}
}
{\TTdev{} extensions}
{ttdev}


\subsection{Proof State}

\label{sect:prfstate}

We will build expressions in \TT{} from high level \Idris{} expressions by
creating a top level hole binding with the appropriate type, and refining it
according to the structure of the \Idris{} expression. Noting the Curry-Howard
correspondence between programs and proofs, and the relationship between our
method for elaboration and tactic
based theorem proving, we refer to a development as a \emph{proof}.
A proof state is a tuple, $(\vC, \Delta, \ve, \vP, \vQ)$, containing:

\begin{itemize}
\item A global context, $\vC$, containing pattern matching definitions and their types
\item A local context, $\Delta$, containing pattern bindings
\item A proof term, $\ve$, in \TTdev{}
\item Unsolved unification problems, $\vP$
\begin{itemize}
\item These take the form $(\Gamma,\ve_1,\ve_2)$, i.e. the context in which
the problem is run, and the expressions to be unified.
\end{itemize}
\item A hole queue, $\vQ$
%\item \remph{Deferred} definitions, $\vD$, for introducing global metavariables
\end{itemize}

The \remph{hole queue} is a priority queue of names of hole and guess binders 
$\langle\vx_1,\vx_2,\ldots,\vx_n\rangle$
in the proof term ---
ensuring that each bound name is unique. Holes refer to \remph{sub goals}
in the proof. As a proof develops, unification problems arising from unsolved
holes may be introduced, and may be solved when those holes are instantiated.
When the hole queue is empty, and there are no remaining unification 
problems, the proof term is complete.

Creating a \TT{} expression from an \Idris{} expression involves creating
a new proof state, with an empty proof term, and using the high level definition
to direct the building of a final proof state, with a complete proof term.

In the implementation, the proof state is captured in an elaboration monad,
\texttt{Elab}, which is simply a state monad with exceptions. 
Exceptions are captured by the option type $\MO{Result}$ which
either indicates success, or reports an error $\MO{Err}$:

\DM{
\RW{data}\;\MO{Result}\;\vt\;=\;\MO{Success}\;\vt\;\mid\;\MO{Failure}\;\MO{Err}
}

\noindent
If any operation fails during a proof (e.g. a sub term fails to type check)
then the entire proof fails (i.e. returns $\MO{Failure}\;\ve$ with some error
message $\ve$) unless otherwise handled.

The \texttt{Elab} monad supports various operations for querying and updating
the proof state, manipulating terms, generating fresh names, etc. However, we
will describe \Idris{} elaboration in terms of meta-operations on the proof
state, in order to capture the essence of the elaboration process without being
distracted by implementation details. These meta-operations include: 

\begin{itemize}
\item \demph{Queries} which retrieve values from the proof state, without modifying
the state. For example, we can:
\begin{itemize}
\item Get the type of the current sub goal ($\MO{Type}$).
\item Get the current proof term ($\MO{Term}$).
\item Retrieve the local context $\Gamma$ at the current sub goal
($\MO{Context}$).
\item Type check ($\MO{Check}$) or normalise ($\MO{Normalise}$) a term relative
to $\Gamma$.
\end{itemize}
\item \demph{Unification}, which unifies two terms (potentially solving sub
goals) relative to $\Gamma$. This may introduce new unification problems
to the proof state.
\item \demph{Tactics} which update the proof term. Tactics operate on the sub term
at the binder specified by the head of the hole queue $\vQ$.
\item \demph{Focussing} on a specific sub goal, which brings a different sub goal to the
head of the hole queue.
%\item \demph{Deferring} a sub goal, which adds a new definition to the global context
%$\vC$ which solves the sub goal.
\end{itemize}

Elaboration of an \Idris{} expression involves creating a new proof state,
running a series of tactics to build a complete proof term, then retrieving and
\remph{rechecking} the final proof term, which must be a \TT{} program (i.e.
does not contain any of the \TTdev{} extensions). We call a sub-term which
contains no hole or guess bindings \demph{pure}. Although a pure term does not
contain hole or guess bindings, it may nevertheless \remph{refer} to hole- or
guess-bound variables.

The proof state is initialised with the $\MO{NewProof}$ operation. Given a
global context $\vC$, $\MO{NewProof}\;\vt$ sets up the proof state as:

\DM{
(\vC, \cdot, \hole{\vx}{\vt}\SC\vx, \langle\rangle, \langle\vx\rangle)
}

The local context is initially empty, and the initial hole queue is the $\vx$ standing for
the entire expression. The proof term is reset with the $\MO{NewTerm}$ operation.
In an existing proof state $(\vC, \Delta, \ve, \vP, \vQ)$,
$\MO{NewTerm}\;\vt$ discards the proof term and hole queue, and
updates the proof state to:

\DM{
(\vC, \Delta, \hole{\vx}{\vt}\SC\vx, \langle\rangle, \langle\vx\rangle)
}

Resetting the term and queue but retaining pattern bindings
allows the elaborator to use pattern bindings from the left
hand side of a pattern matching definition in the term on the right hand side.


\renewcommand{\Check}{\MO{Check}_\Gamma}
\newcommand{\Eval}{\MO{Normalise}_\Gamma}
\newcommand{\Unify}{\MO{Unify}_\Gamma}
\newcommand{\PrimUnify}{\MO{PrimUnify}_\Gamma}
\newcommand{\Subst}{\MO{Subst}}

\subsection{Tactics}

% Meta-operations Check, Normalise, Unify 
In order to build \TT{} expressions from \Idris{} programs, we define a collection
of meta-operations for querying and modifying the proof state. Meta-operations
may have side-effects including failure, or updating the proof state, and
may fail. We have the following primitive meta-operations:

\begin{itemize}
\item $\MO{Get}$ and $\MO{Put}$ which allow direct access to the proof state.
% Focus and Unfocus don't need to be primitive, we could show in terms of
% Get and Put
%\item $\MO{Focus}\;\vn$, which moves $\vn$ to the head of the hole queue.
%\item $\MO{Unfocus}$, which moves the current hole to the back of the hole queue.
\item $\Check\;\ve$, which type checks an expression $\ve$ relative to a context
$\Gamma$, returning its type.
$\MO{Check}$ will fail
if the expression is not well-typed.
\item $\MO{Convert}_\Gamma\;\ve_1\;\ve_2$, which checks that the well-typed
expressions $\ve_1$ and $\ve_2$ respect the cumulativity relation 
(i.e. $\Gamma\proves\ve_1\cumul\ve_2$), and fails otherwise.
\item $\Eval\;\ve$, which evaluates a well-typed expression $\ve$ relative to a context 
$\Gamma$, returning its normal form. $\MO{Normalise}$ reduces under binders,
but does not reduce definitions which are not total.
\item 
$\PrimUnify\;\ve_1\;\ve_2$, 
which attempts to unify $\ve_1$ and $\ve_2$ by finding the values with which
holes must be instantiated for $\ve_1$ and $\ve_2$ to be convertible relative
to $\Gamma$ (i.e. for $\Gamma\proves\ve_1\converts\ve_2$ to hold).
\item $\Subst\;\vx\;\ve$, which instantiates a hole $\vx$ directly with a term
$\ve$, typically arising from unification and removes $\vx$ from the hole
queue.
\end{itemize}

\noindent
Additionally, we have $\MO{Focus}$, $\MO{Unfocus}$, $\MO{NewHole}$,
$\MO{NextHole}$ and $\MO{RemoveHole}$
operations which manipulate and inspect
the hole queue. $\MO{Focus}\;\vn$ moves a hole to the head of the hole queue:

\DM{
\MO{Focus}\;\vn\;=\;\RW{do}\;\AR{
(\vC,\Delta,\ve,\vP,\vQ;\vn;\vQ')\gets\MO{Get}\\
\MO{Put}\;(\vC,\Delta,\ve,\vP,\vn;\vQ;\vQ')\\
}
}

\noindent
$\MO{Focus}$ refers to a hole $\vn$ by name. Hole names are either given
explicitly in proofs, or chosen (as fresh names) by tactics.
Correspondingly, $\MO{Unfocus}$ moves the currently in focus hole to the
end of the hole queue:

\DM{
\MO{Unfocus}\;=\;\RW{do}\;\AR{
(\vC,\Delta,\ve,\vP,\vn;\vQ)\gets\MO{Get}\\
\MO{Put}\;(\vC,\Delta,\ve,\vP,\vQ;\vn)\\
}
}

\noindent
$\MO{NewHole}$ declares that a new hole has been created and is to be solved
next; $\MO{NextHole}$ returns the next hole to be solved, and $\MO{RemoveHole}$
removes a hole from the head of the queue:

\DM{
\AR{
\MO{NewHole}\;\vn\;=\;\RW{do}\;\AR{
(\vC,\Delta,\ve,\vP,\vQ)\gets\MO{Get}\\
\MO{Put}\;(\vC,\Delta,\ve,\vP,\vn;\vQ)\\
}
\\
\MO{NextHole}\;=\;\RW{do}\;\AR{
(\vC,\Delta,\ve,\vP,\vn;\vQ)\gets\MO{Get}\\
\RW{return}\;\vn
}
\\
\MO{RemoveHole}\;=\;\RW{do}\;\AR{
(\vC,\Delta,\ve,\vP,\vn;\vQ)\gets\MO{Get}\\
\MO{Put}\;(\vC,\Delta,\ve,\vP,\vQ)
}
}
}

\noindent
\demph{Tactics} are meta-operations which operate on the sub-term given by the
hole at the head of the hole queue in the proof state. They take the following
form:

\DM{
\PA{\A\A}{
\MO{Tac}_\Gamma & \;\vec{\VV{args}} & \;\vt & \MoRet{\RW{do}\;\AR{
\ldots\\
\RW{return}\:\vt'}
}
}
}

A tactic $\MO{Tac}$ takes a sequence of zero or more arguments
$\vec{\VV{args}}$ followed by the sub-term $\vt$ on which it is operating. It
runs relative to a context $\Gamma$ which contains all the bindings and pattern
bindings in scope at that point in the term. The sub-term $\vt$ will either be
a hole binding $\hole{\vx}{\vT}\SC\ve$ or a guess binding
$\guess{\vx}{\vT}{\vv}\SC\ve$. The tactic returns a new term $\vt'$ which can
take any form, provided it is well-typed, with a type convertible to the type
of $\vt$.  Tactics may also have the side effect of updating the proof state,
therefore we will describe tactics in a pseudo-code with $\RW{do}$ notation.

Tactics are executed by a higher level meta-operation $\MO{Tactic}$, 
given in Figure \ref{runtac} which
locates the appropriate sub-term, applies the tactic with the context
local to this sub-term, and
replaces the sub-term with the term returned by the
tactic. Note that monadic lifting is left implicit in this Figure, so that
the details of the algorithm are not obscured.
% It then updates the hole queue in the proof state, and updates holes which have
% been solved by unification. If the tactic creates new holes, these are automatically
% added to the \remph{head} of the hole queue.

\FFIG{
\AR{
\MO{Tactic}\;\VV{tac}\;=\;\RW{do}\;\AR{
\ve\Gets\MO{Term}\\
\vh\Gets\MO{NextHole}\\
\ve'\Gets\MO{Tactic'}\;\langle\rangle\;\ve\\
\MO{SetTerm}\;\ve'
}
\medskip
\\
\PA{\A\A\A}{
\RW{where} \\
& \MO{Tactic'} & \Gamma & (\hole{\vx}{\vT}\SC\ve)
& \MoRet{\VV{tac}_\Gamma\;(\hole{\vx}{\vT}\SC\ve)\hg\mbox{if}\;\vx=\vh} \\
& \MO{Tactic'} & \Gamma & (\hole{\vx}{\vT}\SC\ve)
& \MoRet{
\RW{do}\; \AR{\vT'\Gets\;\MO{Tactic'}\;\Gamma\;\vT\\
\RW{return}\;(\hole{\vx}{\vT'}\SC \MO{Tactic'}\;(\Gamma;\hole{\vx}{\vT'})\;\ve)
\hg\RW{otherwise}}
}
\\
& \MO{Tactic'} & \Gamma & (\guess{\vx}{\vT}{\vv}\SC\ve)
& \MoRet{\VV{tac}_\Gamma\;(\guess{\vx}{\vT}{\vv}\SC\ve)\hg\mbox{if}\;\vx=\vh} \\

& \MO{Tactic'} & \Gamma & (\guess{\vx}{\vT}{\vv}\SC\ve)
& \MoRet{\RW{do}\;\AR{\vv'\Gets\;\MO{Tactic'}\;\Gamma\;\vv\\
\vT'\Gets\;\MO{Tactic'}\;\Gamma\;\vT\\
\RW{return}\;(\guess{\vx}{\vT'}{\vv'}\SC
\MO{Tactic'}\;(\Gamma;\guess{\vx}{\vT'}{\vv'}\;\ve)
\hg\RW{otherwise}}} \\

& \MO{Tactic'} & \Gamma & (\lam{\vx}{\vT}\SC\ve)
& \MoRet{
\RW{do}\; \AR{\vT'\Gets\;\MO{Tactic'}\;\Gamma\;\vT\\
\RW{return}\;(\lam{\vx}{\vT'}\SC \MO{Tactic'}\;(\Gamma;\lam{\vx}{\vT'})\;\ve)}
}
\\
& \MO{Tactic'} & \Gamma & (\all{\vx}{\vT}\SC\ve)
& \MoRet{
\RW{do}\; \AR{\vT'\Gets\;\MO{Tactic'}\;\Gamma\;\vT\\
\RW{return}\;(\all{\vx}{\vT'}\SC \MO{Tactic'}\;(\Gamma;\all{\vx}{\vT'})\;\ve)}
}
\\
& \MO{Tactic'} & \Gamma & (\LET\;\vx\defq\vv\Hab\vT\SC\ve)
& \MoRet{\RW{do}\;\AR{\vv'\Gets\;\MO{Tactic'}\;\Gamma\;\vv\\
\vT'\Gets\;\MO{Tactic'}\;\Gamma\;\vT\\
\RW{return}\;(\LET\;\vx\defq\vv'\Hab\vT'\SC
\MO{Tactic'}\;(\Gamma;\LET\;\vx\defq\vv'\Hab\vT')\;\ve)}} \\
& \MO{Tactic'} & \Gamma & (\vf\;\va) & 
\MoRet{\RW{return}\;(\MO{Tactic'}\;\Gamma\;\vf)\; (\MO{Tactic'}\;\Gamma\;\va)}\\
& \MO{Tactic'} & \Gamma & \vt & \MoRet{\RW{return}\;\vt}
}
}
}
{Applying tactics to the proof state}
{runtac}

In the remainder of this section, we will define a set of primitive tactics:
$\MO{Claim}$, $\MO{Fill}$ and $\MO{Solve}$ which are used to create and destroy
holes; and $\MO{Lambda}$, $\MO{Pi}$, $\MO{Let}$ and $\MO{Attack}$ which are
used to create binders. First, however, we must explain how unification is
applied during elaboration.

\subsubsection{Unification}

Unification is crucial to the success of elaboration --- it is unification
which finds values of implicit arguments, in particular. We do not require
unification to succeed immediately, in general, as instantiating a hole may 
affect the solution to other unification problems. Instead, 
$\MO{PrimUnify}$ may give one of the following results:

\begin{itemize}
\item $\MO{Success}\;(\tx=\te, \tp)$, containing the solved holes mapping
$\vx$ to $\ve$, and a possibly empty set of blocked unification problems $\tp$.
If $\tp$ is empty, there is a solution for $\ve_1$ and $\ve_2$. 
\item $\MO{Failure}\;\ve$, if no solution is found due to a failure to unify
sub-terms in normal form.
\end{itemize}

$\MO{Unify}$ attempts to solve the given unification problem using
$\MO{PrimUnify}$,
then attempts to further refine existing problems in the proof
state. This operation is given in Figure \ref{unifyps}. After attempting to
solve the problem with $\MO{PrimUnify}$ and updating the proof term with
any solutions found, it updates the existing problems (with $\MO{SubstProblems}$
which applies $\MO{Subst}$ across the terms in the problems), then attempts to
resolve the existing problems with $\MO{ReUnify}$.

\FFIG{
%\mbox{\textbf{[UNIFY ALGORITHM GOES HERE] }}
%Show how it uses a PrimUnify
%operation to update
%the proof state with new problems, and solve the old ones
\AR{
\Unify\;\ve_1\;\ve_2\;=\;\RW{do}\;
\AR{
(\tx=\te,\tp)\gets\PrimUnify\;\ve_1\;\ve_2\\
\vec{\MO{Subst}}\;\tx\;\te\\
\MO{SubstProblems}\;\tx\;\te\\
(\vC,\Delta,\VV{term},\tP,\vQ)\gets\MO{Get}\\
\vec{\MO{ReUnify}}\;\tP
}
\medskip
\\
\AR{
\MO{ReUnify}\;(\Gamma,\ve_1,\ve_2)\;=\;\RW{do}\;\AR{
(\tx=\te,\tp)\Gets\PrimUnify\;\ve_1\;\ve_2\\
\vec{\MO{Subst}}\;\tx\;\te\\
\RW{return}\;\tp
}
}
}
}
{The $\MO{Unify}$ meta-operation}
{unifyps}


\subsubsection{Creating and destroying holes}

The $\MO{Claim}$ tactic, given a name and a type, adds a new hole binding in
the scope of the current goal $\vx$, adding the new binding to the hole queue, but
keeping $\vx$ at the head:

\DM{
\PA{\A\A}{
\MO{Claim}_\Gamma & (\vy \Hab\vS) & (\hole{\vx}{\vT}\SC\ve) & 
   \MoRet{\RW{do}\;\AR{\MO{NewHole}\;\vy\\
          \MO{Focus}\:\vx\\
          \RW{return}\;\hole{\vy}{\vS}\SC\hole{\vx}{\vT}\SC\ve}} \\
}
}

An obvious difficulty is in ensuring that names are unique throughout a proof
term.  The implementation ensures that any hole created by the $\MO{NewHole}$
operation has a unique name by checking against existing names in scope and
modifying the name if necessary. In this paper, we will assume that all created
names are fresh.

The $\MO{Fill}$ tactic, given a value $\vv$, attempts to solve the current goal
with $\vv$, creating a guess binding in its place. $\MO{Fill}$ attempts to
solve other holes by unifying the expected type of $\vx$ with the type of $\vv$:

\DM{
\PA{\A\A}{
\MO{Fill}_\Gamma & \vv & (\hole{\vx}{\vT}\SC\ve) & 
   \MoRet{\RW{do}\;\AR{
   \vT' \leftarrow \Check\;\vv\\
   \Unify\;\vT\;\vT'\\
   \RW{return}\;\guess{\vx}{\vT}{\vv}\SC\ve}
   } \\
}
}

\noindent
For example, consider the following proof term:

\DM{
\AR{
\hole{\va}{\Type}\SC\hole{\vk}{\Nat}\SC
\hole{\vx}{\va}\SC\hole{\vxs}{\Vect\;\vk\;\va}\SC
\\
\hole{\vys}{\Vect\;(\suc\;\vk)\;\va}\SC\vys
}
}

\noindent
If $\vx$ is in focus (i.e., at the head of the hole queue) and the elaborator
attempts to
$\MO{Fill}$ it with an $\TC{Int}$ value $42$, we have:

\begin{itemize}
\item $\Check\;42\;\mq\;\TC{Int}$
\item Unifying $\TC{Int}$ with $\vA$ (the type of $\vx$) is only possible if
$\vA\;=\;\TC{Int}$, so solve $\vA$.
\end{itemize}

\noindent
Therefore the resulting proof term is:

\DM{
\AR{
\hole{\vk}{\Nat}\SC
\guess{\vx}{\TC{Int}}{42}\SC\hole{\vxs}{\Vect\;\vk\;\TC{Int}}\SC
\\
\hole{\vys}{\Vect\;(\suc\;\vk)\;\TC{Int}}\SC\vys
}
}

The $\MO{Solve}$ tactic operates on a guess binding. If the guess is \remph{pure}, i.e., it
is a \TT{} term containing no hole or guess bindings, then the value attached to
the guess is substituted into its scope, and the corresponding hole removed
from the hole queue:

\DM{
\PA{\A}{
\MO{Solve}_\Gamma & (\guess{\vx}{\vT}{\vv}\SC\ve) &
   \MoRet{\RW{do}\;\AR{\MO{RemoveHole}\\
   \RW{return}\;\ve[\vv/\vx]\hg\mbox{(if $\MO{Pure}\;\vv$)}}}
}
}

The two step process, with $\MO{Fill}$ followed by $\MO{Solve}$, allows the elaborator
to work safely with incomplete terms, since an incomplete guess binding has no 
computational force. Once a term is complete in a guess binding, it may be substituted into the 
scope of the binding safely.
In each of these tactics, if any step fails, or the term in focus does not take
the correct form (e.g. is not a guess in the case of $\MO{Solve}$ or not a hole
in the case of $\MO{Claim}$ and $\MO{Fill}$) the entire tactic fails. We can
handle failure using the $\MO{Try}$ tactic combinator:

\DM{
\PA{\A\A\A}{
\MO{Try}_\Gamma & \VV{t1} & \VV{t2} & \vt &
   \MoRet{\RW{case}\;\AR{\VV{t1}_\Gamma\;\vt\;\RW{of}\\
             \MO{Success}\;\vt'\cq\MO{Success}\;\vt'\\
             \MO{Failure}\;\_\cq\VV{t2}_\Gamma\;\vt\\
          }}
}
}

We have a primitive tactic $\MO{Fail}$, which may be invoked by
any tactic which encounters a failure condition (for example, an unsolvable unification
problem) and is handled by $\MO{Try}$.

\subsubsection{Creating binders}

We also define primitive tactics for constructing binders. Creating a $\lambda$
binding requires that the goal normalises to a function type:

\DM{
\PA{\A\A}{
\MO{Lambda}_\Gamma & \vn & (\hole{\vx}{\vT}\SC\vx) &
 \MoRet{\RW{do}\;\AR{
   \all{\vy}{\vS}\SC\vT'\;\leftarrow\;\Eval\;\vT \\
   \RW{return}\;\lam{\vn}{\vS}\SC\hole{\vx}{\vT'[\vn/\vy]}\SC\vx
   }
   }
}
}

\noindent
Creating a $\forall$ binding requires that the goal type converts
with $\TC{Type}$:

\DM{
\PA{\A\A}{
\MO{Pi}_\Gamma & (\vn\Hab\vS) & (\hole{\vx}{\Type}\SC\vx) &
 \MoRet{\RW{do}\;\AR{
   \vV\;\leftarrow\;\Check\;\vS\\
   \MO{Convert}_\Gamma\;\vV\;\Type\\
   \RW{return}\;\all{\vn}{\vS}\SC\hole{\vx}{\Type}\SC\vx
   }
   }
}
}

\noindent
Creating a $\LET$ binding requires a type and a value.

\DM{
\PA{\A\A}{
\MO{Let}_\Gamma & (\vn\Hab\vS\defq\vv) & (\hole{\vx}{\vT}\SC\vx) &
 \MoRet{\RW{do}\;\AR{
   \vV\;\leftarrow\;\Check\;\vS\\
   \MO{Convert}_\Gamma\;\vV\;\Type\\
   \vS'\;\leftarrow\;\Check\;\vv\\
   \Unify\;\vS\;\vS'\\
   \RW{return}\;\LET\;\vn\Hab\vS\defq\vv\SC\hole{\vx}{\vT}\SC\vx
   }
   }
}
}

Each of these tactics require the term in focus to be of the form $\hole{\vx}{\vT}\SC\vx$.
This is important, because if the scope of the binding were an arbitrary expression $\ve$,
the binder would be scoped across this \emph{whole} expression rather than the subexpression
$\vx$ as intended.
The $\MO{Attack}$ tactic ensures that a hole
is in the appropriate form, creating a new hole $\vx'$ which is placed at the
head of the queue:

\DM{
\PA{\A}{
\MO{Attack}_\Gamma & (\hole{\vx}{\vT}\SC\ve) &
 \MoRet{\RW{do}\;\AR{
 \MO{NewHole}\;\vx'\\
 \RW{return}\;\guess{\vx}{\vT}{(\hole{\vx'}{\vT}\SC\vx')}\SC\ve}}
}
}

\noindent
For example, consider the following proof term:

\DM{
\hole{\vx}{\va\to\vb}\SC\vx\;\ve
}

\noindent
If we try to instantiate $\vx$ with a $\lambda$-binding $\vn$ immediately,
without applying $\MO{Attack}$ first, the $\lambda$ will scope over the
entirety of $\vx\;\ve$, and the resulting term is not well-typed:

\DM{
\AR{
\lam{\vn}{\va}\SC\hole{\vx}{\vb}\SC\vx\;\ve
}
}

\noindent
If, on the other hand, we $\MO{Attack}$ the hole then instantiate it with
a $\lambda$ binding $\vn$, we get:

\DM{
\begin{array}{ll}
\guess{\vx}{\va\to\vb}{\hole{\vx'}{\va\to\vb}\SC\vx'}\SC\vx\;\ve
& \hg\mbox{(after $\MO{Attack}$)}\\
\guess{\vx}{\va\to\vb}{\lam{\vn}{\va}\SC\hole{\vx'}{\vb}\SC\vx'}\SC\vx\;\ve
& \hg\mbox{(after $\MO{Lambda}\;\vn$)}
\end{array}
}

\noindent
Now, once $\vx'$ has been instantiated, the guess for $\vx$ can safely be
closed with $\MO{Solve}$.

\subsubsection{Pattern Binders}

Finally, we can convert a hole binding to a pattern binding by giving the 
pattern variable a name. This solves a hole
by adding the pattern binding to the proof state, and updating the proof term
with the pattern variable directly:

\DM{
\PA{\A\A}{
\MO{Pat}_\Gamma & \vn & (\hole{\vx}{\vT}\SC\ve) &
  \MoRet{\RW{do}\;\AR{
    \MO{PatBind}\;(\vn\Hab\vT)\\
    \RW{return}\;\ve[\vn/\vx]
  }}
}
}

The $\MO{PatBind}$ operation simply updates the proof state with the given pattern
binding. Once we have created bindings from the left hand side of a pattern
matching definition, for example, we can retain these bindings for use when
building the right hand side.

\subsubsection{Example}

To illustrate how tactics are applied, using $\MO{Tactic}$, to build a
complete term by refining a proof state, let us consider the following
\TT{} definition for the identify function:

\DM{
\AR{
\FN{id}\Hab\all{\vA}{\Type}\SC\all{\va}{\vA}\SC\vA \\
\FN{id}\;=\;\lam{\vA}{\Type}\SC\lam{\va}{\vA}\SC\va  
}
}

We can build $\FN{id}$ either as a complete term, or by applying a sequence of
tactics.  To achieve this, we create a proof state initialised with the type of
$\FN{id}$ and apply a series of $\MO{Lambda}$ and $\MO{Fill}$ operations using
$\MO{Tactic}$.  Note that the types on each $\MO{Lambda}$ are taken from
the goal type.

\DM{
\AR{
\MO{MkId}\;\mq\;\RW{do}\;
 \AR{
   \MO{NewProof}\;\all{\vA}{\Type}\SC\all{\va}{\vA}\SC\vA \\
   \MO{Tactic}\;\MO{Attack} \\
   \MO{Tactic}\;(\MO{Lambda}\;\vA) \\
   \MO{Tactic}\;\MO{Attack} \\
   \MO{Tactic}\;(\MO{Lambda}\;\va) \\
   \MO{Tactic}\;(\MO{Fill}\;\va)\\ 
   \MO{Tactic}\;\MO{Solve}\\
   \MO{Tactic}\;\MO{Solve}\\
   \MO{Tactic}\;\MO{Solve}\\
   \MO{Term}
 }
}
}

\FFIG{
\begin{array}{lll}
\mbox{\textbf{Tactic}} & \mbox{\textbf{Resulting Proof Term}} &  
\mbox{\textbf{Queue}}
\medskip
\\
\mbox{Initial state} 
& \underline{\hole{\vx}{\all{\vA}{\Type}\SC\all{\va}{\vA}\SC\vA}\SC\vx}
& \langle\vx\rangle \\
%
\MO{Attack} &
\guess{\vx}{\all{\vA}{\Type}\SC\all{\va}{\vA}\SC\vA}{
\underline{\hole{\vh_0}{\all{\vA}{\Type}\SC\all{\va}{\vA}\SC\vA}\SC\vh_0}
}
\SC\vx
&
\langle\vh_0,\vx\rangle\\
%
\MO{Lambda}\;\vA &
\guess{\vx}{\all{\vA}{\Type}\SC\all{\va}{\vA}\SC\vA}{
\lam{\vA}{\Type}\SC\underline{
\hole{\vh_0}{\all{\va}{\vA}\SC\vA}\SC\vh_0
}
}
\SC\vx
&
\langle\vh_0,\vx\rangle\\
%
\MO{Attack} &
\guess{\vx}{\all{\vA}{\Type}\SC\all{\va}{\vA}\SC\vA}{
\lam{\vA}{\Type}\SC
\guess{\vh_0}{\all{\va}{\vA}\SC\vA}
{
\underline{
\hole{\vh_1}{\all{\va}{\vA}\SC\vA}\SC\vh_1}}
\SC\vh_0
}
\SC\vx
&
\langle\vh_1,\vh_0,\vx\rangle\\
%
\MO{Lambda}\;\va &
\guess{\vx}{\all{\vA}{\Type}\SC\all{\va}{\vA}\SC\vA}{
\lam{\vA}{\Type}\SC
\guess{\vh_0}{\all{\va}{\vA}\SC\vA}
{
\lam{\va}{\vA}\SC\underline{
\hole{\vh_1}{\vA}\SC\vh_1
}
}
\SC\vh_0
}
\SC\vx
&
\langle\vh_1,\vh_0,\vx\rangle\\
%
\MO{Fill}\;\va &
\guess{\vx}{\all{\vA}{\Type}\SC\all{\va}{\vA}\SC\vA}{
\lam{\vA}{\Type}\SC
%\\
%&
%\hg\hg\hg\hg\hg\hg\hg
\guess{\vh_0}{\all{\va}{\vA}\SC\vA}
{
\lam{\va}{\vA}\SC
\underline{
\guess{\vh_1}{\vA}{\va}\SC\vh_1
}
}
\SC\vh_0
}
\SC\vx
&
\langle\vh_1,\vh_0,\vx\rangle\\
%
\MO{Solve} &
\guess{\vx}{\all{\vA}{\Type}\SC\all{\va}{\vA}\SC\vA}{
\lam{\vA}{\Type}\SC
\underline{
\guess{\vh_0}{\all{\va}{\vA}\SC\vA}
{
\lam{\va}{\vA}\SC
\va
}
\SC\vh_0
}
}
\SC\vx
&
\langle\vh_0,\vx\rangle\\
%
\MO{Solve} &
\underline{
\guess{\vx}{\all{\vA}{\Type}\SC\all{\va}{\vA}\SC\vA}{
\lam{\vA}{\Type}\SC
\lam{\va}{\vA}\SC
\va
}
\SC\vx
}
&
\langle\vx\rangle\\
%
\MO{Solve} &
\lam{\vA}{\Type}\SC
\lam{\va}{\vA}\SC
\va
&
\langle\rangle\\

\end{array}
}
{Proof state development for $\MO{MkId}$}
{mkidrefine}

Figure \ref{mkidrefine} shows how each step of this sequence of tactics affects
the proof state and hole queue. In each step, the underlined fragment of the
term is the fragment in focus. One might observe that the 
$\MO{Attack}$/$\MO{Solve}$ pairs are unnecessary in this case, as the term
is already in a form suitable for introduction. However, we retain these for
uniform treatment of introductions.

To aid readability, we will elide $\MO{Tactic}$, and use a semi-colon to indicate
sequencing --- in the implementation we use wrapper functions for each tactic to
apply $\MO{Tactic}$.
Using this convention, we can build $\FN{id}$'s type and definition as shown
in Figure \ref{idelab}. Note that $\MO{Term}$ retrieves the proof term from the current proof
state. Both $\MO{MkIdType}$ and $\MO{MkId}$ finish by returning a completed \TT{} term.
Note in particular that each tactic which introduces a guess ($\MO{Fill}$ 
and $\MO{Attack}$) is closed with a $\MO{Solve}$.

Setting up elaboration in this way, with a proof state captured in a monad,
and a primitive collection of tactics, makes it easy to derive more complex
tactics for elaborating higher level language constructs, in much the same way
as the \texttt{Ltac} language in Coq~\cite{Delahaye2000}. As a result, the
description of elaboration of a language construct (or a program such as
$\FN{id}$) bears a strong resemblance to
a Coq proof script.

\FFIG{
\AR{
\MO{MkIdType}\;\mq\;\RW{do}\;
 \AR{
   \MO{NewProof}\;\Type\\
   \MO{Attack} ; \MO{Pi}\;(\vA\Hab\Type) ;
   \MO{Attack} ; \MO{Pi}\;(\va\Hab\vA) \\
   \MO{Fill}\;\vA \\
   \MO{Solve} ; \MO{Solve} ; \MO{Solve} \\
   \MO{Term}
 }
 \medskip\\
\MO{MkId}\;\mq\;\RW{do}\;
 \AR{
   \vt\;\leftarrow\;\MO{MkIdType}; \MO{NewProof}\;\vt\\
   \MO{Attack} ; \MO{Lambda}\;\vA ; 
   \MO{Attack} ; \MO{Lambda}\;\va \\
   \MO{Fill}\;\va \\
   \MO{Solve} ; \MO{Solve} ; \MO{Solve}\\
   \MO{Term}
 }
}
}
{Building $\FN{id}$ with tactics}
{idelab}

%--- give unify in full, esp. as it solves sub goals? Maybe...

% Unify' G x t             = Success (x, t) if ?x : t in G
% Unify' G t x             = Success (x, t) if ?x : t in G
% Unify' G (b x. e) (b' x'. e')   = Unify' G b b'; Unify' G;b e e'[x/x']
% Unify' G ((\x.e) x) e'   = Unify' G e e' 
% Unify' G e ((\x.e') x)   = Unify' G e e' 
% Unify' G (f es) (f' es') = vs <- Unify' G f f'; Injective f
                             
% Unify' G x y             = Success () if G |- x == y
% Unify' G . .             = Failure

% Unify' G (\x : t . e) (\x : t' . e') = Unify' G t t'; Unify' G e e'
% ...


%\DM{
%}

\subsection{System State}

\label{sect:sysstate}

To elaborate \Idris{} programs we will build expressions from high level
declarations of functions, data types and classes, and add the resulting
definitions to a global system state.
The system state is a tuple, $(\vC,\vA,\vI)$, containing:

\begin{itemize}
\item A global context, $\vC$, containing pattern matching definitions and their types
\item Implicit arguments, $\vA$, recording which arguments are implicit for each global name
\item Class information, $\vI$, containing method signatures and dictionaries for classes
\end{itemize}

In the implementation, the system state is captured in a monad, \texttt{Idris},
which includes additional information such as syntax overloadings, command line
options, and optimisations, which do not concern us here. 

For each global name, $\vA$ records whether its arguments are explicit, implicit,
or class constraints.  For example, recall the declaration
of \texttt{vAdd}:

\begin{SaveVerbatim}{vAddImpT}

vAdd : Num a => Vect n a -> Vect n a -> Vect n a

\end{SaveVerbatim}
\useverb{vAddImpT} 

\noindent
Written in full, and giving each argument an explicit name, we get the
type declaration:

\begin{SaveVerbatim}{vAddImpT}

vAdd : (a : _) -> (n : _) -> (c : Num a) -> 
       (xs : Vect n a) -> (ys : Vect n a) -> Vect n a

\end{SaveVerbatim}
\useverb{vAddImpT} 

\noindent
For \tFN{vAdd}, the state records that \texttt{a} and \texttt{n} are implicit, 
\texttt{c} is a constraint, and \texttt{xs} and \texttt{ys} are explicit. When
the elaborator encounters an application of \tFN{vAdd}, it knows that unless these arguments
are given explicitly, the application must be expanded.

\subsection{Elaborating Expressions}

\newcommand{\piimp}[2]{\mbox{\texttt{\{$#1$ : $#2$\} -> }}}
%\mathtt{\{}#1\mathtt{:}#2\mbox{\texttt{\} -> }}}
\newcommand{\piexp}[2]{\mbox{\texttt{($#1$ : $#2$) -> }}}
%\newcommand{\piexp}[2]{\mathtt{(}#1\mathtt{:}#2\mbox{\texttt{) -> }}}
\newcommand{\piconst}[1]{\mbox{\texttt{$#1$ => }}}
\newcommand{\icase}{\mathtt{case}}
\newcommand{\iwith}{\mathtt{with}}
\newcommand{\idata}{\mathtt{data}}
\newcommand{\iclass}{\mathtt{class}}
\newcommand{\iinstance}{\mathtt{instance}}
\newcommand{\iwhere}{\mathtt{where}}
\newcommand{\iof}{\mathtt{of}}
\newcommand{\ilet}[2]{\mathtt{let}\;#1\;\mathtt{=}\;#2\;\mathtt{in}}
\newcommand{\ilam}[1]{\mathtt{\backslash}\;#1\;\mathtt{=>}}
\newcommand{\iarg}[2]{\mbox{\texttt{\{$#1$ = $#2$\}}}}
\newcommand{\ihab}[2]{\mbox{\texttt{$#1$ : $#2$}}}
\newcommand{\carg}[1]{\mbox{\texttt{\{\{$#1$\}\}}}}
\newcommand{\fatarrow}{\mbox{\texttt{=>}}}
\newcommand{\ibar}{\mbox{\texttt{|}}}
\newcommand{\mvar}[1]{\mbox{\texttt{?}}#1}

\FFIG{
\AR{
\begin{array}{rll@{\hg}rll}
\ve, \vt ::= & \vc & (\mbox{constant}) &
\mid & \vx & (\mbox{variable}) \\
\mid & \ilam{\vx}\;\ve & (\mbox{lambda abstraction}) &
\mid & \ve\;\ta & (\mbox{function application}) \\
\mid & \ilet{\vx}{\ve}\;\ve & (\mbox{let binding}) &
\mid & \piexp{\vx}{\vt}\vt & (\mbox{explicit function space}) \\
%\mid & \icase\;\ve\;\iof\;\vec{\VV{alt}} & (\mbox{case expression}) &
%\mid & \mvar{\vx} & (\mbox{metavariable}) \\
\mid & \_ & (\mbox{placeholder})
\end{array}
\medskip\\
\begin{array}{rll}
\va :: = & \ve & (\mbox{normal argument}) \\
\mid & \iarg{\vx}{\ve} & (\mbox{implicit argument with value}) \\
\mid & \carg{\ve} & (\mbox{explicit class instance}) 
\end{array}
\medskip\\
\begin{array}{rll}
\VV{declty} ::= & \ve & (\mbox{expression}) \\
\mid & \piexp{\vx}{\vt}\VV{declty} & (\mbox{explicit function space}) \\
\mid & \piimp{\vx}{\vt}\VV{declty} & (\mbox{implicit function space}) \\
\mid & \VV{constr}\;\VV{declty} & (\mbox{constrained type}) \\
\end{array}
\medskip\\
\begin{array}{rll}
\VV{constr} ::= & \piconst{\ttt} & (\mbox{class constraint}) \\
\end{array}
}
}
{\IdrisM{} expressions}
{idrism}

\FFIG{
\AR{
\begin{array}{rll}
\vd ::= & \ihab{\vx}{\VV{declty}} & (\mbox{type declaration}) \\
\mid & \VV{pclause} & (\mbox{pattern clause}) \\
\mid & \VV{ddecl} & (\mbox{data type declaration}) \\
\mid & \VV{cdecl} & (\mbox{class declaration}) \\
\mid & \VV{idecl} & (\mbox{instance declaration}) \\
\end{array}
\medskip\\
\begin{array}{ll}
\begin{array}{rll}
\VV{pclause} ::= & 
\vx\;\ttt\;[\ibar\;\te]\mbox{\texttt{ = }}\ve \hg[\iwhere\;\td]\\ 
\mid & \AR{
\vx\;\ttt\;[\ibar\;\te]\;\iwith\;\ve\\
\hg\vec{\VV{pclause}}
}
\end{array}
&
\begin{array}{rll}
\VV{ddecl} ::= & \idata\;\ihab{\vx}{\VV{declty}}\;\iwhere\;\vec{\VV{con}}\\
\VV{con} ::= & \ihab{\vx}{\VV{declty}}\\
\medskip\\
\VV{cdecl} ::= & \iclass\;[\VV{constr}]\;\vx\;(\ihab{\tx}{\ttt})\;\iwhere\;\td
\\
\VV{idecl} ::= & \iinstance\;[\VV{constr}]\;\vx\;\ttt\;\iwhere\;\td
\end{array}
\end{array}
}
}
{\IdrisM{} declarations}
{idrismd}

We will use a subset of full \Idris{}, which we call \IdrisM{}, to explain
elaboration of expressions. \IdrisM{} expression syntax is given in Figure
\ref{idrism} and declaration syntax given in Figure \ref{idrismd}.
\IdrisM{} is a subset of \Idris{} without syntactic sugar --- that is, without
\texttt{do}-notation or infix operators --- and with implicit
arguments in types bound explicitly 
(e.g. $\piimp{\va}{\_}\piimp{\vn}{\_}\TC{Vect}\;\vn\;\va$
instead of simply $\Vect\;\vn\;\va$).
Note that we separate implicit and type constraint bindings, to ensure that they
only appear in top level declarations.
It is in general straightforward to
convert full \Idris{} to \IdrisM{} --- syntactic sugar is implemented by a
direct source transformation, and implicit arguments can be identified as the names
which are free in a type in a non-function position.
%
\IdrisM{} differs from \TT{} in several important respects. It has implicit
syntax and classes, and functions are applied to multiple arguments rather
than one at a time. 

%Extensions over \TT{}: implicit syntax, case expressions. Functions are applied to
%multiple arguments, rather than one at a time (this helps with implicit syntax).
%\IdrisM{} also supports explicit \demph{metavariables}. A metavariable \texttt{?mvar}
%creates a new global function \tFN{mvar}, with its type such that it would
%be type correct to apply \tFN{mvar} to all of the variables in scope.

%Implicit and type class arguments? Expanded at the application site (we need to know
%it's the global name after all and we do that by type).

To elaborate expressions, we define meta-operations $\ttinterp{\cdot}$ and
$\ttpinterp{\cdot}$, which run relative to a proof state (see Section
\ref{sect:prfstate}). $\ttpinterp{\cdot}$ elaborates \remph{patterns}, i.e.
expressions on the left hand side of a pattern matching definition, whereas
$\ttinterp{\cdot}$ elaborates expressions on the right hand side.
Their effect is
to update the proof state so that the hole in focus contains a representation
of the given expression, by applying tactics. We assume that the proof state
has already been set up, which means that elaboration can always be
\remph{type-directed} since the proof state contains the type of the expression we
are building.

In all cases except the elaboration of variables and constants, 
$\ttpinterp{\vx}\;=\;\ttinterp{\vx}$.

\subsubsection{Elaborating variables and constants}

In the simplest cases, there is a direct translation from an \IdrisM{} expression to
a \TT{} expression --- we build \TT{} representations of variables and constants using
the \MO{Fill} tactic:

\DM{
\AR{
\ttinterp{\vx}\;\mq\;\RW{do}\;\MO{Fill}\;\vx;\;\MO{Solve}\\
\ttinterp{\vc}\;\mq\;\RW{do}\;\MO{Fill}\;\vc;\;\MO{Solve}\\
}
}

Since $\MO{Fill}$ creates a guess binding, we use $\MO{Solve}$ to instantiate
the guess.  We need not concern ourselves with type checking variables or
constants here --- $\MO{Fill}$ will handle this, type checking $\vx$ or $\vc$
and unifying the result with the hole type. If there are any errors,
elaboration will fail.

If we are building the left hand side of a pattern clause, however, there is a
problem, as it is the left hand side which \remph{defines} variables. In this
context, we assume that attempting to elaborate a variable which does not type
check means that the variable is a pattern variable\footnote{In Haskell,
pattern variables are all lower case, so type checking can assume that a
lower case name in a pattern refers to a variable. In \Idris{} however, a
name in a pattern may refer to a function, if forced by some dependency.}:

\DM{
\ttpinterp{\vx}\;\mq\;
\MO{Try}\;\AR{(\RW{do}\;\MO{Fill}\;\vx;\;\MO{Solve})\\
(\MO{Pat}\;\vx)
}
}

We also need to elaborate \remph{placeholders}, which are subexpressions we expect to
solve by unification. In this case, we simply move on to the next hole in the queue, moving
the current hole to the end of the queue with $\MO{Unfocus}$:

\DM{
\ttinterp{\_}\;\mq\;\MO{Unfocus}
}

On encountering a placeholder, our assumption is that unification will
eventually solve the hole. At the end of elaboration, any holes remaining
unsolved on the left hand side become pattern variables. If there are any
unsolved on the right hand side, elaboration is incomplete and an error will
be reported when the resulting term is type checked.

\subsubsection{Elaborating bindings}

To elaborate a $\lambda$-binding, we $\MO{Attack}$ the hole, then apply the
$\MO{Lambda}$ tactic, which will fail if the goal is not a function type. We
then elaborate the scope and $\MO{Solve}$, which discharges the $\MO{Attack}$:

\DM{
\ttinterp{\ilam{\vx}\ve}\;\mq\;
\RW{do}\;
\AR{
\MO{Attack};\;\MO{Lambda}\;\vx\\
\ttinterp{\ve}\\
\MO{Solve}
}
}

Note that there is no type on the $\lambda$-binding in \IdrisM{}. There is no need --- since
elaboration is type directed, the $\MO{Lambda}$ tactic finds the type of the binding by
looking at the type of the hole. In full \Idris{}, types are allowed on bindings, and the
elaborator merely checks that the given type is equivalent to the inferred type.

Elaborating a function type is more tricky, since we have to elaborate the argument
type (itself an \IdrisM{} expression) then elaborate the scope. To achieve this, we
create a new goal $\vX$ for the argument type $\vt$, where $\vX$ is a fresh name,
and introduce a function binding with argument type $\vX$. We can then focus on
$\vX$, and elaborate it with the expression $\vt$. Finally, we elaborate the
scope of the binding.

\DM{
\AR{
\ttinterp{\piexp{\vx}{\vt}\ve}\;\mq\;
\RW{do}\;
\AR{
\MO{Attack};\;\MO{Claim}\;(\vX\Hab\Type)\\
\MO{Pi}\;(\vx\Hab\vX)\\
\MO{Focus}\;\vX\\
\ttinterp{\vt};\;
\ttinterp{\ve}\\
\MO{Solve}
}
}
}

Elaborating $\vt$ will involve solving unification problems, which will, if the
program is type correct, solve $\vX$. After focussing on $\vX$ and elaborating
$\vt$, there is no need to refocus on the hole representing the scope, as it was
previously at the head of the hole queue before focussing on $\vX$.
Elaboration of implicit and constraint
argument types is exactly the same --- \TT{} makes no distinction between then.

To elaborate a \texttt{let} binding, we take a similar approach, creating new subgoals
for the \texttt{let}-bound value and its type, then elaborating the scope. Again,
if elaborating the scope is successful, unification will solve the claimed variables
$\vV$ and $\vX$.

\DM{
\ttinterp{\ilet{\vx}{\vv}\;{\ve}}\;\mq\;\RW{do}\;
\AR{
\MO{Attack};\;
\MO{Claim}\;(\vX\Hab\Type);\; 
\MO{Claim}\;(\vV\Hab\vX)\\
\MO{Let}\;(\vx\Hab\vX\defq\vV)\\
\MO{Focus}\;\vV\\
\ttinterp{\vv};\;
\ttinterp{\ve}\\
\MO{Solve}
}
}
\subsubsection{Elaborating Applications}

\label{sect:elabapps}

There are two cases to consider when elaborating applications:

\begin{itemize}
\item Applying a global function name to arguments, some of which may be implicit.
\item Applying an expression, which does not have implicit arguments, to an argument.
\end{itemize}

In the first case, the elaborator must expand the application to include implicit arguments.
For example \texttt{vAdd xs ys} is expanded to
\texttt{vAdd \{a=\_\} \{n=\_\} \{\{\_\}\} xs ys}, adding the implicit arguments
\texttt{a} and \texttt{n} and a class instance argument. The meta-operation
$\MO{Expand}\;\vx\;\ta$, given a global function name $\vx$ and the
arguments supplied by the programmer $\ta$, returns a new argument list
$\ta'$ with implicit and class arguments added. Each value in
$\ta'$ is paired with an explicit name for the argument.

Implicit arguments are solved by unification
--- the type or value of another argument determines the value of an implicit argument,
with the appeal to $\MO{Unify}$ in the $\MO{Fill}$ tactic solving as many extra holes
as it can. However, unification problems can take several forms. For example, assuming
$\vf$, $\vg$, $\vx$ are holes, we might have unification problems of the following
forms:

\DM{
\AR{
\MO{Unify}_\Gamma\;\vf\;\TC{Int}\\
\MO{Unify}_\Gamma\;(\vf\;\vx)\;\TC{Int}\\
\MO{Unify}_\Gamma\;(\vf\;\vx)\;(\vg\;\vx)\\
}
}

The first problem has a solution: $\vf=\TC{Int}$. The second and third problems
have no solution without further information. In the second case, we cannot
conclude anything about $\vf$ or $\vx$ just from knowing that
$\vf\;\vx\;=\;\TC{Int}$.  In the third case, although we have $\vx\;=\;\vx$ in
argument position, we cannot conclude that $\vf\;=\;\vg$ in general as a
result.

Once $\vf$ or $\vg$ is solved in some other way, perhaps by being given
explicitly in another argument, these unification problems can make progress.
Otherwise, unification blocks.  In such a case, the elaborator stores the
problem (i.e. the expressions to be unified with their local context) and
refines it with further information when it becomes available. 
Elaborating a global function application makes a $\MO{Claim}$ for each
argument in turn. Then the function is elaborated, applied to each of the
claimed arguments. Each argument which has been given explicitly is
elaborated. 
Finally, any class instance arguments are resolved with the
built-in $\MO{Instance}$ tactic.

\DM{
\AR{
\ttinterp{\vx\;\ta}\;\mq\;
\RW{do}\;
\AR{
(\tn,\tv)\;\gets\;\MO{Expand}\;\vx\;\ta\\
\vec{\MO{Claim}}\;(\tn\Hab\tT)\\
\MO{Fill}\;\vx\;\tn\\
\vec{\MO{ElabArg}}\;\tn\;\tv\hg\mbox{(for non-placeholder $\vv$)}\\
\MO{Solve} \\
\vec{\MO{Instance}}\;\tn\hg\mbox{(for class constraint argument $\vn$)}
}
\medskip
\\
\MO{ElabArg}\;\vn\;\vv\;\mq\;
\MO{Focus}\;\vn;\;\ttinterp{\vv}
}
}

The $\MO{Instance}$ tactic focuses on the given hole and searches the context for
a class instance which would solve the goal directly. First, it examines the local
context, then recursively searches the global context. $\MO{Instance}$ is covered
in detail in Section \ref{sect:instance}.

To elaborate a simple function application, of an arbitrary expression to an
arbitrary argument (possibly with a dependent type), we need not worry about
implicit arguments or class constraints. Instead, the function and argument are
elaborated, and the results applied. Since elaboration is type directed,
however, there must be an appropriate type for the function. 

\DM{
\ttinterp{\ve\;\va}\;\mq\;\RW{do}\;
\AR{
\MO{Claim}\;(\vA\Hab\Type);\;\MO{Claim}\;(\vB\Hab\Type)\\
\MO{Claim}\;(\vf\Hab\vA\to\vB);\;\MO{Claim}\;(\vs\Hab\vA)\\
\MO{Focus}\;\vf;\;\ttinterp{\ve}\\
\MO{Focus}\;\vs;\;\ttinterp{\va}\\
}
}

%\subsubsection{Elaborating metavariables}

%\subsubsection{Elaborating \texttt{case} expressions}

\subsection{Elaborating Declarations}

To elaborate declarations, we define a meta-operation $\MO{Elab}$, which runs
relative to the system state (see Section \ref{sect:sysstate}) and has access to the
current proof state. $\MO{Elab}$ uses $\ttinterp{\cdot}$ to help translate
\IdrisM{} type, function and class declarations into \TT{} declarations.

\newcommand{\edo}[1]{\RW{do}\;\AR{#1}}

\subsubsection{Elaborating Type Declarations}

Elaborating a type declaration involves creating a new proof state,
translating the \IdrisM{} type to a \TT{} type, then adding the resulting type
as a \TT{} declaration:

\DM{
\MO{Elab}\;(\vx\Hab\vt)\;\mq
\;\edo{\MO{NewProof}\;\Type\\
       \ttinterp{\vt}\\
       \vt'\gets\MO{Term}\\
       \MO{TTDecl}\;(\vx\Hab\vt')
}}

The final $\MO{TTDecl}$ takes the result of elaboration, type checks it,
and adds it to the global context if type checking succeeds. This final type check
ensures that the elaboration process does not allow any ill-typed terms to creep
into the context.

\subsubsection{Elaborating Data Types}

Elaborating a data declaration involves elaborating the type declaration itself,
as a normal type declaration. This ensures that the type is in scope when
elaborating the constructor types. Then using the results, the elaborator
adds a \TT{} data declaration.

\DM{
\AR{
\MO{Elab}\;(\idata\;\ihab{\vx}{\vt}\;\iwhere\;\tc)\\
\hg\hg\mq\;\edo{\MO{NewProof}\;\Type\\
                \ttinterp{\vt}\\
                \vt'\gets\MO{Term}\\
                \MO{TTDecl}\;(\vx\Hab\vt')\\
                \tc'\gets\vec{\MO{ElabCon}}\;\tc\\
                \MO{TTDecl}\;(\Data\;\vx\Hab\vt'\;\Where\;\tc')
}
\AR{
\MO{ElabCon}\;(\vx\Hab\vt)\\
\hg\hg\mq\; \edo{\MO{NewProof}\;\Type\\
                 \ttinterp{\vt}\\
                 \vt'\gets\MO{Term}\\
                 \RW{return}\;(\vx\Hab\vt')
}
}}}

\subsubsection{Elaborating Pattern Matching}

Elaborating a pattern matching definition works clause by clause, elaborating
the left and right hand sides in turn. $\MO{ElabClause}$ returns the elaborated
left and right hand sides, and may have the side effect of adding entries
to the global context, such as definitions in $\iwhere$ clauses.
First, let us consider the simplest case, with no $\iwhere$ clause:

\DM{
\MO{ElabClause}\;(\vx\;\ttt\;=\;\ve)\;\mq\;?
}

How does the left hand side get elaborated, given that elaboration is type directed, and
its type is not known until after elaboration? 
This can be achieved without any change to the elaborator by defining 
a type $\TC{Infer}$:

\DM{
\Data\hg\TC{Infer}\Hab\Type\hg\Where\hg\DC{MkInfer}\Hab\all{\va}{\Type}\SC\va\to\TC{Infer}
}

Now elaboration of the left hand side proceeds by elaborating $\DC{MkInfer}\;\_\;(\vx\;\ttt)$ and
extracting the value and unified type when complete.

\DM{
\AR{
\MO{ElabClause}\;(\vx\;\ttt\;=\;\ve)\;\mq\;\\
\hg\hg\edo{
\MO{NewProof}\;\TC{Infer};\;
\ttpinterp{\DC{MkInfer}\;\_\;(\vx\;\ttt)}\\
\DC{MkInfer}\;\vT\;\VV{lhs}\gets\MO{Term}\\
\tp\gets\MO{Patterns}\\
\MO{NewTerm}\;\vT;\;
\ttinterp{\ve}\\
\VV{rhs}\gets\MO{Term}\\
\RW{return}\;(\RW{var}\;\tp\SC\VV{lhs}\;=\;\VV{rhs})
}
}
}

This infers a type for the left hand side, creating pattern variable bindings. Then, 
it elaborates the right hand side using the inferred type of the left hand side
and the inferred pattern bindings, which are retrieved from the state with the
$\MO{Patterns}$ operation. Finally, it returns a pattern clause in \TT{} form.
Elaborating a collection of pattern clauses then proceeds by mapping $\MO{ElabClause}$ over
the clauses and adding the resulting collection to \TT{}.

\DM{
\AR{
\MO{Elab}\;\vec{\VV{pclause}}\;\mq\;
\edo{
\tc\gets\vec{\MO{ElabClause}}\;\vec{\VV{pclause}}\\
\vec{\MO{TTDecl}}\;\tc
}
}
}

Elaborating a clause is made only slightly more complex by the presence of
$\iwhere$ clauses. In this case, after elaborating the left hand side, the pattern
bound variables are added as extra arguments to the declarations in the $\iwhere$ block,
which are then recursively elaborated before the right hand side.

\DM{
\AR{
\MO{ElabClause}\;(\vx\;\ttt\;=\;\ve\;\iwhere\;\td)\;\mq\;\\
\hg\hg\edo{
\MO{NewProof}\;\TC{Infer};\;
\ttpinterp{\DC{MkInfer}\;\_\;(\vx\;\ttt)}\\
\DC{MkInfer}\;\vT\;\VV{lhs}\gets\MO{Term}\\
\tp\gets\MO{Patterns}\\
(\td', \ve') \gets\MO{Lift}\;\tp\;\td\;\ve\\
\vec{\MO{Elab}}\;\td'\\
\MO{NewTerm}\;\vT;\;
\ttinterp{\ve'}\\
\VV{rhs}\gets\MO{Term}\\
\RW{return}\;(\RW{var}\;\tp\SC\VV{lhs}\;=\;\VV{rhs})
}
}
}

In \TT{}, all definitions must be at the top level. Therefore, the declarations 
in the $\iwhere$ block are lifted out, adding the pattern bound names as additional arguments
to ensure they are in scope, using the $\MO{Lift}$ operation. This also modifies
the right hand side $\ve$ to use the lifted definitions, rather than the original.

\subsubsection{Elaborating the \texttt{with} rule}

The \texttt{with} rule allows \remph{dependent} pattern matching on intermediate values.
Translating this into \TT{} involves constructing an auxiliary top level
definition for 
the intermediate pattern match. For example, recall \tFN{natToBin}:

\useverb{natToBin}

\noindent
This is equivalent to the following, using top level definitions only:

\begin{SaveVerbatim}{natToBinw}

natToBin : Nat -> List Bool
natToBin O = Nil
natToBin k = natToBin' k (parity k)

\end{SaveVerbatim}
\begin{SaveVerbatim}{natToBinwp}

natToBin' : (n : Nat) -> Parity n -> List Bool
natToBin' (j + j)     even = False :: natToBin j
natToBin' (S (j + j)) odd  = True  :: natToBin j

\end{SaveVerbatim}
\useverb{natToBinw}

\useverb{natToBinwp}

\noindent
Additionally, the elaborator abstracts over the term being analysed in the type
of the generated auxiliary definition\footnote{This is sometimes referred to
as ``magic \texttt{with}''.} and reorders the arguments as required.
For example:

\begin{SaveVerbatim}{magicwith}

f : Nat -> Bool

ftest : (n : Nat) -> so (f n) -> Nat
ftest n x with (f n)
     | True  = ...
     | False = ...

\end{SaveVerbatim}
\useverb{magicwith}

\noindent
Here, we may require the knowledge that \texttt{f n} is \texttt{True} or
\texttt{False} in the appropriate case. The elaborator abstracts over
\texttt{f n} in the auxiliary definition, specialising the type of
\texttt{x} in each case:

\begin{SaveVerbatim}{magicwithaux}

ftest : (n : Nat) -> so (f n) -> Nat
ftest n x = ftest' n (f n) x

ftest' : (n : Nat) -> (p : Bool) -> so p -> Nat
ftest' n True  x = ... 
ftest' n False x = ... 

\end{SaveVerbatim}
\useverb{magicwithaux}

\noindent
Elaboration of \texttt{with} builds the type declaration and auxiliary function
automatically. The elaborator checks the left hand side, resulting in a set of
patterns $\tp$, then infers a type $\vW$ for the expression $\vw$. The patterns
$\tp$ are then split into those necessary to type $\vW$, and the rest ($\tp_w$
and $\tp_n$ respectively) using
$\MO{SplitContext}$. Then, it builds a new declaration, abstracting over
the matched expression $\vw$ in the types of $\tp_n$.
$\MO{Abstract}\;\ve\;\vw\;\vp$ simply replaces any occurrence of $\vw$ in
$\vp$ with the name $\ve$.

\DM{
\AR{
\MO{ElabClause}\;(\vx\;\ttt\;\iwith\;\vw\;\vec{\VV{pclause}})\;\mq\\
\hg\hg\edo{
\MO{NewProof}\;\TC{Infer};\;
\ttpinterp{\DC{MkInfer}\;\_\;(\vx\;\ttt)}\\
\DC{MkInfer}\;\vT\;\VV{lhs}\gets\MO{Term}\\
\tp\gets\MO{Patterns}\\
\MO{NewTerm}\;\TC{Infer};\;
\ttinterp{\DC{MkInfer}\;\_\;\vw}\\
\DC{MkInfer}\;\vW\;\vw'\gets\MO{Term}\\
(\tp_w,\tp_n)\;\gets\;\MO{SplitContext}\;\vW\;\tp\\
\MO{TTDecl}\;(\vx'\Hab\tp_{w}\to(\ve\Hab\vW)\to\vec{\MO{Abstract}}\;\ve\;\vw'\;\tp_{n}\to\vT)\\
\tc'\gets\vec{\MO{MatchWith}}\;\tp\;\ttt\;\vec{\VV{pclause}}\\
\vec{\MO{Elab}}\;\tc'\\
\RW{return}\;(\RW{var}\;\tp\SC\VV{lhs}\;=\;\vx'\;\tp_{w}\;\vw'\;\tp_{n})
}
}
}
\DM{
\AR{
\MO{MatchWith}\;\tp_w\;\tp_n\;\ttt\;(\vx\;\tw\;\mid\;\ve\;\VV{rhs})\;\mq\\
\hg\hg\edo{
\tp'\gets\vec{\MO{Match}}\;\tw\;\ttt\\
(\tp'_w,\tp'_n)\;\gets\;\MO{SplitPats}\;\tp_w\;\tp_n\;\tp'\\
\RW{return}\;(\vx'\;\tp'_w\;\ve\;\tp'_{n}\;\VV{rhs})
}
}
}

A $\iwith$ block in a clause for $\vx$
results in an auxiliary definition $\vx'$, where $\vx'$ is a fresh name,
which takes an extra argument corresponding to the intermediate result which
is to be matched.
$\MO{MatchWith}$ matches the left hand side of the top level definition against
each $\VV{pclause}$, using the result of the match to build each pattern clause
for $\vx'$. 
The clauses in the with block must be of a specific form: they must have an initial
set of arguments $\tw$ which matches the outer clause arguments
$\ttt$, and an extra argument
$\ve$ which is the same type as the scrutinee of the \texttt{with}, $\vW$.
The right hand side, $\VV{rhs}$, may either return a value directly, containing 
\texttt{where} clauses, or be a nested \texttt{with} block. In any case, it
remains unchanged.
$\MO{Match}\;\vw\;\vt$ returns a list of pattern bindings containing
the variables in $\vt$ and their matches in $\vw$.
The resulting matches are split, using $\MO{SplitPats}$, into matches for those
variables necessary to type $\vW$ and the rest, as before.

\subsubsection{Elaborating Class and Instance Declarations}

Classes are implemented as dictionaries of functions for each instance,
following the implementation of first class type classes in Coq~\cite{Sozeau2008}.
Therefore, a class declaration elaborates to a record type containing all
of the functions in the class, and an instance is simply an instance of the
record. The methods are translated to top level functions which extract the
relevant function from the dictionary.  For example, for the \texttt{Show}
class, the \texttt{class} declaration is translated (in the elaborated code,
using the surface \Idris{} syntax for readability)
to:

\begin{SaveVerbatim}{showdata}

data Show : Type -> Type where
    InstanceShow : (show : a -> String) -> Show a

show : Show a -> a -> String
show (InstanceShow show') = show'

\end{SaveVerbatim}
\useverb{showdata} 

\noindent
An instance for a type \texttt{a} is then a value of type \texttt{Show a}; for example for Nat:

\begin{SaveVerbatim}{shownatf}

showNat : Show Nat
showNat = InstanceShow show' where
    show' : Nat -> String
    show' O = "O"
    show' (S k) = "s" ++ show k

\end{SaveVerbatim}
\useverb{shownatf} 

The call to \tFN{show} in the body of the instance uses the top level \tFN{show}
function, with the instance resolved by the elaborator.
Where instances have constraints, these constraints are passed on to
the function declaration for the instance. For example, for \texttt{Show (List a)}:

\begin{SaveVerbatim}{showlista}

showList : Show a => Show (List a)
showList = InstanceShow show' where
    show' : List a -> String
    show' []        = "[]"
    show' (x :: xs) = show x ++ " :: " ++ show xs

\end{SaveVerbatim}
\useverb{showlista} 

\noindent
Where classes have constraints, the
constraints are again passed on to the data type declaration, and resolved by the
elaborator. For example, for \texttt{Ord}:

\begin{SaveVerbatim}{eqorda}

class Eq a => Ord a where
    compare : a -> a -> Ordering
    max : a -> a -> a
    -- remaining methods elided

\end{SaveVerbatim}
\useverb{eqorda}

\noindent
This translates to:

\begin{SaveVerbatim}{orddata}

data Ord : Type -> Type where
    InstanceOrd : Eq a => (compare : a -> a -> Ordering) -> 
                          (max : a -> a -> a) ->  Ord a 

\end{SaveVerbatim}

\begin{SaveVerbatim}{orddatab}
compare : Ord a -> a -> a -> Ordering
compare (InstanceOrd compare' max') = compare'

max : Ord a -> a -> a -> a
max (InstanceOrd compare' max') = max'

\end{SaveVerbatim}
\useverb{orddata}

\useverb{orddatab}

\noindent
The elaborator also adds functions to retrieve parent classes from a
dictionary, to assist with class resolution. In this case:

\begin{SaveVerbatim}{ordparent}

OrdEq : Ord a -> Eq a
OrdEq (InstanceOrd {{eq}} compare' max') = eq

\end{SaveVerbatim}
\useverb{ordparent} 

\noindent
In general, \texttt{class} declarations are elaborated as follows:

\DM{
\AR{
\MO{Elab}\;(\iclass\;\piconst{\tc}\TC{C}\;(\ta\Hab\ttt)\;\iwhere\;\td)
\\
\hg\hg
\mq\;\RW{do}\;\AR{
%(\vec{\VV{meth}}\Hab\ttt)\gets\td\\
\MO{Elab}\;
\AR{(\RW{data}\;\TC{C}\Hab(\ta\Hab\ttt)\to\Type\;\iwhere\\
\hg\DC{InstanceC}\Hab\piconst{\tc}\td\to\TC{C}\;\ta)}\\
\vec{\MO{ElabMeth}}\;\td\\ % (\vec{\VV{meth}}\Hab\ttt)\\
\vec{\MO{ElabParent}}\;\tc
}
\medskip\\
\MO{ElabMeth}\;(\vf\Hab\vt)\\
\hg\hg\mq\;\RW{do}\;\AR{
\MO{Elab}\;(\vf\Hab\TC{C}\;\ta\to\vt)\\
\MO{Elab}\;(\vf\;(\DC{InstanceC}\;\vec{\VV{meth}})\;=\;\vec{\VV{meth}[\vf]})
}
\medskip\\
\MO{ElabParent}\;\vc\\
\hg\hg\mq\;\RW{do}\;\AR{
\MO{Elab}\;(\VV{cC}\Hab\TC{C}\;\ta\to\vc)\\
\MO{Elab}\;(\VV{cC}\;(\DC{InstanceC}\;\carg{\vc'}\;\vec{\VV{meth}})\;=\;\vc')
}
}
}

\noindent
Then \texttt{instance} declarations are elaborated as follows. One complication
is that the function definitions in an \texttt{instance} declaration do not
have explicit types, so we must retrieve the method types from the system state
and substitute $\TC{C}\;\ttt$, using $\MO{ExpandTypes}$.
The name $\FN{instanceC}$ is a fresh name generated for the newly
defined instance:

\DM{
\AR{
\MO{Elab}\;(\iinstance\;\piconst{\tc}\TC{C}\;\ttt\;\iwhere\;\td)\\
\hg\hg\mq\;\RW{do}\;\AR{
(\vec{\VV{meth}}, \vec{\VV{ty}})\gets\MO{ExpandTypes}\;\td\\
\vec{\VV{ty'}}\;=\;\vec{\VV{ty}}[\ttt/\ta]\\
\MO{Elab}\;(\FN{instanceC}\Hab\piconst{\tc}\TC{C}\;\ttt)\\
\MO{Elab}\;\AR{
(\FN{instanceC}\;=\;\DC{InstanceC}\;\vec{\VV{meth'}}\;\iwhere
\\
\hg\vec{\VV{meth'}}\Hab\vec{\VV{ty'}}\\
\hg\td
)}
}
}
}

Adding default definitions is straightforward: simply insert the default definition where
there is a method missing in an \texttt{instance} declaration.

\textbf{Remark:} Here, we have added a higher level language construct
(classes) simply by elaborating in terms of lower level language constructs
(data types and functions).  We achieve type class resolution by implementing a
tactic, $\MO{Instance}$.  We can build several extensions this way because we
have effectively built, bottom up, an Embedded Domain Specific Language for
constructing programs in \TT{}.

\subsubsection{The $\MO{Instance}$ tactic}

\label{sect:instance}

When elaborating applications, recall that implicit arguments are filled in by
unification as described in Section \ref{sect:elabapps}, and class
constraints by an $\MO{Instance}$ tactic which searches the context for an
instance of the required class. 

The $\MO{Instance}$ tactic begins by trying to find a solution in the local context,
using the $\MO{Trivial}$ tactic, which attempts
to solve the current goal with each local variable in turn. It is used as a very
basic primitive for proof search, and is defined as follows:

\DM{
\AR{
\MO{Trivial}\;\mq\;
\RW{do}\;\AR{
      (\tv\Hab\ttt)\gets\MO{Context}\\
      \MO{TryAll}\;\tv
    }
\medskip\\
\MO{TryAll}\;(\vv_1,\vv_2\ldots)\;\mq\;\MO{Try}\;(\MO{Fill}\;\vv_1)\;(\MO{TryAll}\;\vv_2\ldots)
\\
\MO{TryAll}\;\langle\rangle\;\mq\;\MO{Fail}
}
}

For example, in the following function, the class constraint required for
the $\texttt{>}$ operator is resolved by finding an instance for \texttt{Ord a}
in the local context:

\begin{SaveVerbatim}{ordlocal}

max : Ord a => a -> a -> a
max x y = if (x > y) then x else y

\end{SaveVerbatim}
\useverb{ordlocal} 

If $\MO{Trivial}$ does not find a solution, $\MO{Instance}$ begins a global search,
applying every class instance, and attempting to resolve the instance's parameters
recursively. Given an instance function $\FN{instanceC}$ with parameters
of type $\ttt$:

\DM{
\MO{TryInstance}\;\FN{instanceC}\;\mq\;\RW{do}\;
\AR{
\vec{\MO{Claim}}\;(\tx\Hab\ttt)\\
\ttinterp{\FN{instanceC};\tx}\\
\vec{\MO{Focus}}\;\tx;\;\vec{\MO{Instance}}
}
}

We define $\MO{Instance}$ as follows, using $\MO{AllInstances}$ to retrieve
all functions which define instances: 

\DM{
\MO{Instance}\;\mq\;
\MO{Try}\;
\AR{
\MO{Trivial}\\
(\RW{do}\;\AR{
\ti\gets\MO{AllInstances}\\
\vec{\MO{TryInstance}}\;\ti
)
}
}
}

In practice, for efficiency and to ensure that resolution terminates,
we constrain the search by recording which instances are defined for each
class, and by ensuring that a recursive call of $\MO{Instance}$ is either
searching for a different class, or a structurally smaller instance of the
current class. 

For example, in the following function, the constraints required by the
\tFN{show} function are resolved by a global search which locates a \tTC{Show}
instance for \tTC{List a}, which in turn requires a \tTC{Show} instance for
\texttt{a}.  In this case \texttt{a} is instantiated by \tTC{Int}, so resolution finishes
by locating a \tTC{Show} instance for \tTC{Int}:

\begin{SaveVerbatim}{showinst}

main : IO ()
main = putStrLn (show [1,2,3,4]) 

\end{SaveVerbatim}
\useverb{showinst} 

\subsection{Syntax Extensions}

We now have a complete elaborator for \IdrisM{}, covering 
dependent pattern matching definitions and expansion of implicit arguments
and classes. In order to make the language truly general purpose, however, we
will need to add higher level extensions. Full \Idris{} is \IdrisM{} extended
with \texttt{do}-notation, idiom brackets~\cite{McBride2007}, \texttt{case} expressions,
pattern matching \texttt{let}, metavariables and tactic based theorem proving.
The majority of these extensions are straightforward transformations of high level
\Idris{} programs --- for example, \texttt{do}-notation can be reduced directly to
\IdrisM{} function applications. Some, however, require further support. In this section,
we complete the presentation of \IdrisM{} by extending it with metavariables
and $\icase$ expressions:

\DM{
\AR{
\begin{array}{rll@{\hg}rll}
\ve ::= & \ldots \\
\mid & \mvar{\vx} & (\mbox{metavariable}) &
\mid & \icase\;\ve\;\iof\;\vec{\VV{alt}} & (\mbox{case expression}) \\
\medskip\\
\VV{alt} ::= & \ve\;\fatarrow\;\ve & (\mbox{case alternative})\\
\end{array}
}
}

\subsubsection{Metavariables}

\demph{Metavariables} are terms which stand for incomplete programs. A
metavariable serves a similar
purpose to a hole binding in \TT{}, but gives rise to a global rather than a
local variable. For example, the following is an incomplete definition of
the vector append function, containing a metavariable \texttt{append\_rec}:

\begin{SaveVerbatim}{vappmv}

(++) : Vect n a -> Vect m a -> Vect (n + m) a
Nil       ++ ys = ys
(x :: xs) ++ ys = x :: ?append_rec

\end{SaveVerbatim}
\useverb{vappmv} 

This generates a new (as yet undefined)
function \texttt{append\_rec}, which takes all of the variables
in scope at the point it is used, and returns a value of the required type.
We can inspect the type of \texttt{append\_rec} at the \Idris{} prompt:

\begin{SaveVerbatim}{appendrec}

*vec> :t append_rec
append_rec : (a : Type) -> (m : Nat) -> (n : Nat) -> a -> 
             Vect n a -> Vect m a -> Vect (n + m) a

\end{SaveVerbatim}
\useverb{appendrec} 

Metavariables serve two purposes: Firstly, they aid type directed program development
by allowing a programmer direct access to the inferred types of local variables, and
the types of subexpressions. Secondly, they allow a separation of program structure
from proof details. Metavariables can be solved either by directly editing program
source, or by providing a definition elsewhere in the file. For example, we can later say:

\begin{SaveVerbatim}{appendrec_def}

append_rec a m n x xs ys = app xs ys

\end{SaveVerbatim}
\useverb{appendrec_def} 

Elaborating a metavariable involves adding a new top level definition which is applied
to all of the variables in local scope.

\DM{
\ttinterp{\mvar{\vx}}\;\mq\;
\edo{
(\tv\Hab\ttt)\gets\MO{Context}\\
\vT\gets\MO{Type}\\
\MO{TTDecl}\;(\vx\Hab(\tv\Hab\ttt)\to\vT)\\
\ttinterp{\vx\;\tv}
}
}

\subsubsection{\texttt{case} expressions}

A \texttt{case} expression allows pattern matching on intermediate values. The difficulty
in elaborating \texttt{case} expressions is that \TT{} allows matching only on
\emph{top level} values. Elaborating a \texttt{case} expression, therefore,
involves creating a new top level function standing for the expression, and applying it. 
The natural
way to implement this is to use a metavariable. For example, we have already seen
\texttt{lookup\_default}:

\useverb{listlookup}

\noindent
This elaborates as follows:

\begin{SaveVerbatim}{listlookupmv}

lookup_default : Nat -> List a -> a -> a
lookup_default i xs def = 
   let scrutinee = list_lookup i xs in ?lookup_default_case

lookup_default_case i xs def Nothing  = def
lookup_default_case i xs def (Just x) = x

\end{SaveVerbatim}
\useverb{listlookupmv} 

To elaborate a $\icase$ expression involves \texttt{let} binding the scrutinee
of the $\icase$ expression, introducing a new name $\VV{scrutinee}$, and
creating a metavariable $\VV{xcase}$ to implement the pattern matching. Then
the elaborator builds a new pattern matching definition for $\VV{xcase}$ and
elaborates it:

\DM{
\AR{
\ttinterp{\icase\;\ve\;\iof\;\vec{\VV{alt}}}\;\mq\;
\edo{
\ttinterp{\ilet{\VV{scrutinee}}{\ve}\;\mvar{\VV{xcase}}}\\
(\tv\Hab\ttt)\gets\MO{Context}\\
\vec{\MO{MkCase}}\;\tv\;\VV{alt}\\
}
\medskip\\
\MO{MkCase}\;\tv\;(\vl\;\fatarrow\;\vr)\;\mq\;\MO{Elab}\;(\VV{xcase}\;\tv\;\vl\;=\;\vr)
}
}

Any nested $\icase$ expressions will be handled by the recursive call to
$\MO{Elab}$. Again, because of the way we have set up an Embedded Domain Specific
Language for describing elaboration, we have been able to implement a new higher
level language feature in terms of elaboration of lower level language features.


%\subsubsection{Tactic-implicit arguments}


%\subsubsection{Pairs and Dependent Pairs}

% Other extensions: do notation, idiom brackets, pairs, etc, are easily expressed
% by transformations of the high level syntax

