\documentclass{jfp1}
%\documentclass[acmtoplas]{acmtrans2m}

\usepackage[draft]{comments}
%\usepackage[final]{comments}
% \newcommand{\comment}[2]{[#1: #2]}
\newcommand{\khcomment}[1]{\comment{KH}{#1}}
\newcommand{\ebcomment}[1]{\comment{EB}{#1}}

\usepackage{epsfig}
%\usepackage{path}
\usepackage{url}
%\usepackage{amsmath} 
\usepackage{fancyvrb}
\usepackage{todonotes}

\newenvironment{template}{\sffamily}

%\usepackage{graphics,epsfig}
\usepackage{stmaryrd}

\input{./macros.ltx}
\input{./library.ltx}

\NatPackage
\FinPackage

\newcounter{per}
\setcounter{per}{1}

\newcommand{\Ivor}{\textsc{Ivor}}
\newcommand{\Idris}{\textsc{Idris}}
\newcommand{\IdrisM}{\textsc{Idris}$^-$}
\newcommand{\TT}{\textsf{TT}}
\newcommand{\TTdev}{\textsf{TT$_{dev}$}}
\newcommand{\Funl}{\textsc{Funl}}
\newcommand{\Agda}{\textsc{Agda}}
\newcommand{\LamPi}{$\lambda_\Pi$}

\newcommand{\perule}[1]{\vspace*{0.1cm}\noindent
\begin{center}
\fbox{
\begin{minipage}{7.5cm}\textbf{Rule \theper:} #1\addtocounter{per}{1}
\end{minipage}}
\end{center}
\vspace*{0.1cm}
}

\newcommand{\mysubsubsection}[1]{
\noindent
\textbf{#1}
}
\newcommand{\hdecl}[1]{\texttt{#1}}


\title
[Idris, a General Purpose Dependently Typed Programm Language]
{Idris, a General Purpose Dependently Typed Programming Language:
Design and Implementation}
%Implementing General Purpose Dependently Typed Programming Languages}
%\subtitle{Implementing Domain Specific Languages by Overloading}

\author[Edwin Brady]
{EDWIN BRADY\\
School of Computer Science, University of St Andrews, St Andrews,
KY16 9SX, UK}

\begin{document}

\maketitle

\begin{abstract}
Many components of a dependently typed programming language are by now well
understood, for example the underlying type theory, type checking, unification and
evaluation.  How to combine these components into a realistic and usable high
level language is, however, folklore, discovered anew by successive
language implementations.  In this paper, I describe the implementation of a
new dependently typed functional programming language, \Idris{}.
\Idris{} is intended to be a \emph{general purpose} programming language
and as such provides high level concepts such as implicit syntax, 
type classes and \texttt{do} notation. 
I describe the high level language and the underlying type theory, and present
a method for \emph{elaborating} concrete high level syntax with implicit
arguments and type classes into a fully explicit type theory. Furthermore,
I show how this method,
based on a domain specific language embedded in Haskell, facilitates the
implementation of new high level language constructs.

%I describe the implementation of a dependently typed functional
%programming language, \Idris{}. Much has been written about various
%aspects of dependently typed language implementation (e.g. checking
%dependent types, unification, optimisation) but nothing yet about how
%to bring it all together into a complete, practical, usable tool. This paper
%attempts to do so. In particular, I explain what is needed to turn 
%concrete syntax with implicit arguments into fully elaborated type
%theory, using unification and a tactic engine.
\end{abstract}


%\category{D.3.2}{Programming Languages}{Language
%  Classifications}[Applicative (functional) Languages]
%\category{D.3.4}{Programming Languages}{Processors}[Compilers]
%\terms{Languages, Verification, Performance}
%\keywords{Dependent Types, Typechecking}


%\begin{bottomstuff}
%Author's address: Edwin Brady, School of Computer Science, North Haugh, St Andrews,
%KY16 9SX
%\end{bottomstuff}

\section{Introduction}

Dependently typed programming languages, such as Agda \cite{norell2007thesis}
and Coq \cite{Bertot2004}, have emerged in recent years as a promising approach
to ensuring the correctness of software. Type checking ensures a program has
the intended meaning; \emph{dependent} types, where types may be predicated
on values, allow a programmer to give a program a more precise type and 
hence have increased confidence of its correctness.
The \Idris{} language
\cite{Brady2011a} aims to take this idea further, by providing support for
verification of \emph{general purpose} systems software. In contrast to Agda and Coq,
which have arisen from the theorem proving community, \Idris{} takes Haskell as
its main influence.  Recent Haskell extensions such as GADTs and type families have
given some of the power of dependent types to Haskell programmers. In the short
term this approach has a clear advantage, since it builds on a mature language
infrastructure with extensive library support.
Taking a longer term view, however,
these extensions are inherently limited in that they are required to maintain
backwards compatibility with existing Haskell implementations.  \Idris{}, being a new
language, has no such limitation, essentially asking the question:

\begin{center}
\emph{``What if Haskell had \emph{full} dependent types?''}
\end{center}

By \emph{full} dependent types, we mean that there is no restriction on which
values may appear in types.  It is important for the sake of usability of a
programming language to provide a notation which allows programmers to express
high level concepts in a natural way. Taking Haskell as a starting point means
that \Idris{} offers a variety of high level structures such as type 
classes, \texttt{do}-notation, primitive types and monadic I/O, for example.
Furthermore, a goal of \Idris{} is to support high level domain specific
language implementation, providing appropriate notation for \emph{domain
experts} and \emph{systems programmers}, who should not be required to be type
theorists in order to solve programming problems.  Nevertheless, it is
important for the sake of correctness of the language implementation to have a
well-defined core language with well-understood meta-theory \cite{Altenkirch2010}. 
How can we achieve both of these goals?

This paper describes a method for elaborating a high level dependently typed
functional programming language to a low level core based on dependent
type theory.  The method involves building an elaboration monad which captures
the state of incomplete programs and a collection of \emph{tactics} used by the
machine to refine programs, directed by high level syntax.  As we shall see,
this method allows higher level language constructs to be elaborated in a
straightforward manner, without compromising the simplicity of the underlying
type theory.

\subsection{Contributions}

A dependently typed programming language relies on several components, many of
which are now well understood. For example, we rely on a type checker for
the core type theory \cite{Chapman2005epigram,loh2010tutorial}, a
unification algorithm \cite{Miller1992} and an evaluator. However, it is less
well understood how to combine these components effectively into a practical
programming language. 

The primary contribution of this paper is a tactic based method for translating
programs in a high level dependently typed programming language to a small core
type theory, \TT{}, based on UTT \cite{luo1994}. The paper describes the
structure of an elaboration monad capturing proof and system state, and
introduces a collection of \remph{tactics} which the machine uses to
manipulate incomplete programs.  Secondly, the paper gives a detailed
description of the core type theory used by \Idris{}, including a full
description of the typing rules.  Finally, through describing the specific
tactics, the paper shows how to extend \Idris{} with higher level features.
While we apply these ideas to \Idris{} specifically, the method for term
construction is equally applicable to other typed programming languages, and
indeed the tactics themselves are applicable to any high level language which
can be explained in terms of \TT{}.

\subsection{Outline}

Translating an \Idris{} source program to an executable proceeds through several
phases, illustrated below:

\begin{center}
\DM{
\mbox{\Idris{}}
\;
\xrightarrow{\mathrm{ (desugaring) }}
\;
\mbox{\IdrisM{}}
\;
\xrightarrow{\mathrm{ (elaboration) }}
\;
\mbox{\TT{}}
\;
\xrightarrow{\mathrm{ (compilation) }}
\;
\mbox{Executable}
}
\end{center}

\noindent
The main focus of this paper is the elaboration phase, which translates
a desugared language \IdrisM{} into a core language \TT{}. In order to put
this into its proper context, I give an overview of the high level language
\Idris{} in Section \ref{sect:hll} and explain the core language \TT{} and
its typing rules in Section \ref{sect:typechecking}; 
Section \ref{sect:elaboration} describes the elaboration process itself,
beginning with a tactic based system for constructing
\TT{} programs, then introducing the desugared language \IdrisM{} and
showing how this is translated into \TT{} using tactics;
Section
\ref{sect:delab} describes the process for translating \TT{} back to \Idris{}
and properties of the translation; finally, 
Section \ref{sect:related} discusses related work and Section \ref{sect:conclusion}
concludes.

\subsection{Typographical conventions}

This paper presents programs in two different but related languages: a high level
language \Idris{} intended for programmers, and a low level language \TT{} to
which \Idris{} is elaborated. We distinguish these languages typographically as
follows:

\begin{itemize}
\item \Idris{} programs are written in \texttt{typewriter} font, as they are written
in a conventional text editor. We use \texttt{e$_i$} to stand for non-terminal
expressions.
\item \TT{} programs are written in mathematical notation, with names arising
from \Idris{} expressions written in \texttt{typewriter} font. We use vector notation
$\te$ to stand for sequences of expressions.
\end{itemize}

Additionally, we describe the translation from \Idris{} to \TT{} in the form
of \emph{meta-operations}, which in practice are Haskell programs. Meta-operations
are operations on \Idris{} and \TT{} syntax, and are identified by their names being
in $\MO{SmallCaps}$.

\subsection{Elaboration Example}

\Idris{} is a Haskell-like pure functional programming language with dependent
types.  A simple example program is the following, which adds corresponding
elements of vectors of the same length:

\begin{SaveVerbatim}{vadd}

vAdd : Num a => Vect n a -> Vect n a -> Vect n a
vAdd Nil       Nil       = Nil
vAdd (x :: xs) (y :: ys) = x + y :: vAdd xs ys

\end{SaveVerbatim}
\useverb{vadd}

\noindent
This illustrates some basic features of \Idris{}:

\begin{itemize}
\item Functions are defined by pattern matching, with a top level type signature.
Names which are free in the type signature (\texttt{a} and \texttt{n}) here
are implicitly bound.
\item The type system ensures that both input vectors are the same length (\texttt{n})
and have the same element type (\texttt{a}), and that the element type and length
are preserved in the output vector.
\item Functions can be overloaded using \remph{classes}. Here, the element type
of the vector \texttt{a} must be numeric and therefore supports the \texttt{+} operator.
\item A single colon is used for type signatures, and a double colon for the
cons operator, emphasising the importance of types.
\end{itemize}

\noindent
\Idris{} programs elaborate into a small core language, \TT{}, which is a $\lambda$-calculus
with dependent types, augmented with algebraic data types and pattern matching.
\TT{} programs are fully explicitly typed, including the names \texttt{a} and \texttt{n}
in the type signature, and the names bound in each pattern match clause.
Classes are also made explicit.
The \TT{} translation of \texttt{vAdd} is:

\DM{
\AR{
\FN{vAdd} \Hab(\va\Hab\Type)\to(\vn\Hab\Nat)\to\TC{Num}\;\va\to
  \Vect\;\vn\;\va\to\Vect\;\vn\;\va\to\Vect\;\vn\;\va\\
\RW{var}\;\va\Hab\Type,\;\vc\Hab\TC{Num}\;\va\SC\\
\hg\FN{vAdd} \; \va \; \Z \; \vc \; (\DC{Nil}\;\va) \; (\DC{Nil}\;\va) \; 
\cq\;\DC{Nil}\;\va \\
\RW{var}\;\AR{
\va\Hab\Type,\;\vk\Hab\Nat,\;\vc\Hab\TC{Num}\;\va,\\
\vx\Hab\va,\;\vxs\Hab\Vect\;\vk\;\va,\;
\vy\Hab\va,\;\vys\Hab\Vect\;\vk\;\va
\SC
\\
\FN{vAdd} \; \va \; (\suc\;\vk) \; \vc 
  \; ((\DC{::})\;\va\;\vk\;\vx\;\vxs) 
  \; ((\DC{::})\;\va\;\vk\;\vy\;\vys) 
   \\
   \hg\hg\cq\:((\DC{::})\;\va\;\vk\;((+)\;\vc\;\vx\;\vy)\; (\FN{vAdd}\;\va\;\vk\;\vc\;\vxs\;\vys))
}
}
}

\noindent
The rest of this paper describes \Idris{} and \TT{}, and systematically explains
how to translate from one to the other.



%\section{overview}

High level view of steps we've taken to implement this new version:

\begin{itemize}
\item Syntax tree for raw and well-typed terms
\item An evaluator for well-typed terms
\item The important bit: a simple type checker. No unification or
  inference.
\item The proof state and a tactic engine (Oleg style~\cite{mcbride-thesis}).
\begin{itemize}
  \item Dealing with names: typechecking and evaluation in a context, managing de Bruijn
        indices becomes tricky.
\end{itemize}
\end{itemize}

For the high level language: we need nothing more than the core type theory,
and a way of putting stuff together. So we typecheck each pattern match clause
individually, in an appropriate context. There is no need for the core type
theory to have pattern matching.


\section{\Idris{} --- the High Level Language}

\label{sect:hll}

\Idris{} is
a pure functional programming language with dependent types. It is
eagerly evaluated by default, and compiled via the Epic supercombinator
library \cite{brady2011epic}.
In this section, I will give a brief introduction to programming in \Idris{},
covering the most important features. A full tutorial is available elsewhere
\cite{idristutorial}. 

\subsection{Preliminaries}

\Idris{} defines several primitive types: \tTC{Int}, \tTC{Integer} and
\tTC{Float} for numeric operations, \tTC{Char} and \tTC{String} for
text manipulation, and \tTC{Ptr} which represents foreign pointers.
There are also several data types declared in the library, including
\tTC{Bool}, with values \tDC{True} and \tDC{False}. All of the usual
arithmetic and comparison operators are defined for the primitive types,
and are overloaded using type classes.

An \Idris{} program consists of a module declaration, followed by an optional
list of imports and a collection of definitions and declarations, for example:

\begin{SaveVerbatim}{constprims}

module main

x : Int
x = 42

main : IO ()
main = putStrLn ("The answer is " ++ show x)

\end{SaveVerbatim}
\useverb{constprims}

\noindent
Like Haskell, the main function
is called \texttt{main}, and input and output is managed with an \texttt{IO}
monad. Unlike Haskell, however, we use a single colon \texttt{:} for type
declarations, emphasising the importance of types, 
and \remph{all} functions must have a top level type
declaration. This is due to type inference being, in general, 
undecidable for languages with
dependent types.

A module declaration also opens a \remph{namespace}. The fully qualified names
declared in this module are \texttt{main.x} and \texttt{main.main}.


\subsection{Types and Functions}

Data types are declared in a similar way to Haskell data types, with a similar
syntax. Natural numbers and lists, for example, are declared as follows in the
library:

\begin{SaveVerbatim}{natlist}

data Nat    = O   | S Nat           -- Natural numbers
                                    -- (zero and successor)
data List a = Nil | (::) a (List a) -- Polymorphic lists

\end{SaveVerbatim}
\useverb{natlist}

\noindent
The above declarations are taken from the standard library. Unary natural
numbers can be either zero, or
the successor of another natural number (\texttt{S k}). 
Lists can either be empty (\texttt{Nil})
or a value added to the front of another list (\texttt{x :: xs}).
In the declaration for \tTC{List}, we used an infix operator \tDC{::}. New operators
such as this can be added using a fixity declaration, as follows:

\begin{SaveVerbatim}{infixcons}

infixr 10 :: 

\end{SaveVerbatim}
\useverb{infixcons}

\noindent
This declares that \texttt{::} is a right associative operator (\texttt{infixr})
with a precedence level of 10.
Functions, data constructors and type constructors may all be given infix
operators as names. They may be used in prefix form if enclosed in brackets,
e.g. \tDC{(::)}. 

\subsection{Functions}

Functions are implemented by pattern matching, again using a similar syntax to
Haskell. Some natural number arithmetic functions can be
defined as follows, again taken from the standard library:

\begin{SaveVerbatim}{natfns}

-- Unary addition
plus : Nat -> Nat -> Nat
plus O     y = y
plus (S k) y = S (plus k y)

-- Unary multiplication
mult : Nat -> Nat -> Nat
mult O     y = O
mult (S k) y = plus y (mult k y)

\end{SaveVerbatim}
\useverb{natfns}

\noindent
The standard arithmetic operators \texttt{+} and \texttt{*} are also overloaded
for use by \texttt{Nat}, and are implemented
using the above functions.  Unlike Haskell, there is no restriction on whether
types and function names must begin with a capital letter or not. 
%Function
%names (\tFN{plus} and \tFN{mult} above), data constructors (\tDC{O}, \tDC{S},
%\tDC{Nil} and \tDC{::}) and type constructors (\tTC{Nat} and \tTC{List}) are
%all part of the same namespace.

\Idris{} has an interactive prompt, at which we can test these functions:

\begin{SaveVerbatim}{fntest}

Idris> plus (S (S O)) (S (S O))
S (S (S (S O))) : Nat
Idris> mult (S (S (S O))) (plus (S (S O)) (S (S O)))
S (S (S (S (S (S (S (S (S (S (S (S O))))))))))) : Nat

\end{SaveVerbatim}
\useverb{fntest}

\noindent
Like arithmetic operations, integer literals are also overloaded using type classes, 
meaning that we can also test the functions as follows:

\begin{SaveVerbatim}{fntest}

Idris> plus 2 2 
S (S (S (S O))) : Nat
Idris> mult 3 (plus 2 2)
S (S (S (S (S (S (S (S (S (S (S (S O))))))))))) : Nat

\end{SaveVerbatim}
\useverb{fntest}

\subsubsection{\texttt{where} clauses}

Functions can also be defined \emph{locally} using \texttt{where} clauses. For example,
to define a function which reverses a list, we can use an auxiliary function which
accumulates the new, reversed list, and which does not need to be visible globally:

\begin{SaveVerbatim}{revwhere}

reverse : List a -> List a
reverse xs = revAcc [] xs where
  revAcc : List a -> List a -> List a
  revAcc acc [] = acc
  revAcc acc (x :: xs) = revAcc (x :: acc) xs

\end{SaveVerbatim}
\useverb{revwhere}

\noindent
Indentation is significant --- functions in the \texttt{where} block must be indented
further than the outer function.

\textbf{Scope:} 
Any names which are visible in the outer scope are also visible in the \texttt{where}
clause (unless they have been redefined, such as \texttt{xs} here).
\emph{However}, names which appear in the type are \emph{not} automatically
in scope. In particular,
in the above example, the \texttt{a} in the top level type and the \texttt{a} in the
auxiliary definition \texttt{revAcc} are \emph{not} the same. If this is the required
behaviour, the \texttt{a} can be brought into scope as follows:

\begin{SaveVerbatim}{revwhereb}

reverse : List a -> List a
reverse {a} xs = revAcc [] xs where
  revAcc : List a -> List a -> List a
  ...

\end{SaveVerbatim}
\useverb{revwhereb}

\subsubsection{Dependent Types}

A standard example of a dependent type is the type of ``lists with length'',
conventionally called ``vectors'' in the dependently typed programming
literature. In \Idris{}, we declare vectors as follows:

\begin{SaveVerbatim}{vect}

data Vect : Set -> Nat -> Set where
   Nil  : Vect a O
   (::) : a -> Vect a k -> Vect a (S k)

\end{SaveVerbatim}
\useverb{vect}

\noindent
Note that we have used the same constructor names as for \tTC{List}. Ad-hoc
name overloading such as this is accepted by \Idris{}, provided that the names
are declared in different namespaces (in practice, normally in different modules).
Ambiguous constructor names are resolved by type. 

This declares a family of types, which requires a different form of declaration
from the simple type declarations above. It resembles a Haskell GADT
declaration: we explicitly state the type
of the type constructor \tTC{Vect} --- it takes a type and a \tTC{Nat} as an
argument, where \tTC{Set} stands for the type of types. We say that \tTC{Vect}
is \emph{parameterised} by a type, and \emph{indexed} over \tTC{Nat}. Each
constructor targets a different part of the family of types. \tDC{Nil} can only
be used to construct vectors with zero length, and \tDC{::} to construct
vectors with non-zero length. In the type of \tDC{::}, we state explicitly that an element
of type \texttt{a} and a tail of type \texttt{Vect a k} (i.e., a vector of length \texttt{k})
combine to make a vector of length \texttt{S k}.

We can define functions on dependent types such as \tTC{Vect} in the same way
as on simple types such as \tTC{List} and \tTC{Nat} above, by pattern matching.
The type of a function over \tTC{Vect} will describe what happens to the
lengths of the vectors involved. For example, \tFN{++}, defined in the
library, appends two \tTC{Vect}s:

\begin{SaveVerbatim}{vapp}

(++) : Vect A n -> Vect A m -> Vect A (n + m)
(++) Nil       ys = ys
(++) (x :: xs) ys = x :: xs ++ ys

\end{SaveVerbatim}
\useverb{vapp}

\subsubsection{The Finite Sets}

Finite sets, as the name suggests, are sets with a finite number of elements.
They are declared as follows in the library:

\begin{SaveVerbatim}{findecl}

data Fin : Nat -> Set where
   fO : Fin (S k)
   fS : Fin k -> Fin (S k)

\end{SaveVerbatim}
\useverb{findecl}

\noindent
This declares
\tDC{fO} as the zeroth element of a finite set with \texttt{S k} elements; 
\texttt{fS n} as the
\texttt{n+1}th element of a finite set with \texttt{S k} elements. 
\tTC{Fin} is indexed by a \tTC{Nat}, which
represents the number of elements in the set. 
Neither constructor targets \texttt{Fin O}, because we cannot construct an
element of an empty set.

A useful application of the \tTC{Fin} family is to represent bounded
natural numbers. Since the first \tTC{n} natural numbers form a finite
set of \tTC{n} elements, we can treat \tTC{Fin n} as the set of natural
numbers bounded by \tTC{n}. 

For example, the following function which looks up an element in a \tTC{Vect},
by a bounded index given as a \tTC{Fin n}, is defined in the library:

\begin{SaveVerbatim}{vindex}

index : Fin n -> Vect a n -> a
index fO     (x :: xs) = x
index (fS k) (x :: xs) = index k xs

\end{SaveVerbatim}
\useverb{vindex}

\noindent
This function looks up a value at a given location in a vector. The location is
bounded by the length of the vector (\texttt{n} in each case), so there is no
need for a run-time bounds check. The type checker guarantees that the location
is no larger than the length of the vector.

Note also that there is no case for \texttt{Nil} here. It would be impossible
to apply such a case ---
since there is no element of \texttt{Fin O}, and the location is a
\texttt{Fin n}, then \texttt{n} can not be \tDC{O}.  As a result, attempting to
look up an element in an empty vector would give a compile time type error,
since it would force \texttt{n} to be \tDC{O}.

\subsubsection{Implicit Arguments}

Let us take a closer look at the type of \texttt{index}:

\begin{SaveVerbatim}{vindexty}

index : Fin n -> Vect a n -> a

\end{SaveVerbatim}
\useverb{vindexty}

\noindent
It takes two arguments, an element of the finite set of \texttt{n} elements, and a vector
with \texttt{n} elements of type \texttt{a}. But there are also two names, 
\texttt{n} and \texttt{a}, which are not declared explicitly. These are \emph{implicit}
arguments to \texttt{index}. We could also write the type of \texttt{index} as:

\begin{SaveVerbatim}{vindeximppl}

index : {a:_} -> {n:_} -> Fin n -> Vect a n -> a

\end{SaveVerbatim}
\useverb{vindeximppl}

\noindent
Here we have given bindings for \texttt{a} and \texttt{n} with placeholders for
their types, to be inferred by the machine. We could also give the types explicitly:

\begin{SaveVerbatim}{vindeximpty}

index : {a:Set} -> {n:Nat} -> Fin n -> Vect a n -> a

\end{SaveVerbatim}
\useverb{vindeximpty}

\noindent
Implicit arguments, given in braces \texttt{\{\}} in the type declaration, are not given in
applications of \texttt{index}; their values can be inferred from the types of
the \texttt{Fin n} and \texttt{Vect a n} arguments. Any name which appears as a parameter
or index in a type declaration, but which is otherwise free, will be automatically
bound as an implicit argument. Indeed, binding arguments in this way is the essence
of dependent types.
Implicit arguments can still be given explicitly in applications, using the syntax
\texttt{\{a=value\}} and \texttt{\{n=value\}}, for example:

\begin{SaveVerbatim}{vindexexp}

index {a=Int} {n=2} fO (2 :: 3 :: Nil)

\end{SaveVerbatim}
\useverb{vindexexp}

\noindent
In fact, any argument, implicit or explicit, may be given a name. For example,
we could have declared the type of \texttt{index} as:

\begin{SaveVerbatim}{vindexn}

index : (i:Fin n) -> (xs:Vect a n) -> a

\end{SaveVerbatim}
\useverb{vindexn}

\noindent
This can be useful for improving the readability of type signatures, particularly
where the name suggests the argument's purpose.

\subsection{Type Classes}

\Idris{} supports overloading in two ways. Firstly, as we have already seen with
the constructors of \texttt{List} and \texttt{Vect}, names
can be overloaded in an ad-hoc manner and resolved according to the context in which
they are used. This is mostly for convenience, to eliminate the need to decorate
constructor names in similarly structured data types, and eliminate explicit qualification
of ambiguous names where only one is well-typed --- this is especially useful
for disambiguating record field names\footnote{Records are however beyond the scope
of this paper}.

Secondly, \Idris{} implements \remph{type classes}, following Haskell.  This
allows a more principled approach to overloading --- a type class gives a
collection of overloaded operations which describe the interface for
\remph{instances} of that class.

A simple example
is the \texttt{Show} type class, which is defined in the library and
provides an interface for converting values to
\texttt{String}s:

\begin{SaveVerbatim}{showclass}

class Show a where
    show : a -> String

\end{SaveVerbatim}
\useverb{showclass}

\noindent
This generates a function of the following type (which we call a \emph{method} of the 
\texttt{Show} class):

\begin{SaveVerbatim}{showty}

show : Show a => a -> String

\end{SaveVerbatim}
\useverb{showty}

An instance of a class
is defined with an \texttt{instance} declaration, which provides implementations of
the function for a specific type. For example, the \texttt{Show} instance for \texttt{Nat}
could be defined as:

\begin{SaveVerbatim}{shownat}

instance Show Nat where
    show O = "O"
    show (S k) = "s" ++ show k

\end{SaveVerbatim}
\useverb{shownat}

\begin{SaveVerbatim}{shownati}

Idris> show (S (S (S O))) 
"sssO" : String

\end{SaveVerbatim}
\useverb{shownati}

\noindent
Only one instance of a class can be given for a type --- instances may not overlap.
Instance declarations can themselves have constraints. For example, to define a
\texttt{Show} instance for vectors, we need to know that there is a \texttt{Show} 
instance for the element type, because we are going to use it to convert each element
to a \texttt{String}:

\begin{SaveVerbatim}{showvec}

instance Show a => Show (Vect a n) where
    show xs = "[" ++ show' xs ++ "]" where
        show' : Vect a n' -> String
        show' Nil        = ""
        show' (x :: Nil) = show x
        show' (x :: xs)  = show x ++ ", " ++ show' xs

\end{SaveVerbatim}
\useverb{showvec}

\noindent
\textbf{Remark: } The type of the auxiliary function \texttt{show'} is
important. The type variables \texttt{a} and \texttt{n} which are part of the
instance declaration for \texttt{Show (Vect a n)} are fixed across the entire
instance declaration. As a result, we do not have to constrain \texttt{a}
again. Furthermore, it means that if we use \texttt{n} in the type, it refers
to the (fixed) length of the outermost list \texttt{xs}. Therefore,
we use a different name for the length \texttt{n'} in \texttt{show'}.

Like Haskell type classes, default definitions can be given in the class declaration.
Otherwise, all methods must be given in an instance. For example, there is an
\texttt{Eq} class:

\begin{SaveVerbatim}{eqdefault}

class Eq a where
    (==) : a -> a -> Bool
    (/=) : a -> a -> Bool

    x /= y = not (x == y)
    y == y = not (x /= y)

\end{SaveVerbatim}
\useverb{eqdefault}

\noindent
Classes can also be extended. A logical next step from an equality relation \texttt{Eq}
is to define an ordering relation \texttt{Ord}. We can define an \texttt{Ord} class
which inherits methods from \texttt{Eq} as well as defining some of its own:

\begin{SaveVerbatim}{ord}

data Ordering = LT | EQ | GT

\end{SaveVerbatim}
\useverb{ord} 

\begin{SaveVerbatim}{eqord}

class Eq a => Ord a where
    compare : a -> a -> Ordering
    (<) : a -> a -> Bool
    -- etc

\end{SaveVerbatim}
\useverb{eqord}

\subsection{Matching on intermediate values}

\subsubsection{\texttt{let} bindings}

Intermediate values can be calculated using \texttt{let} bindings:

\begin{SaveVerbatim}{letb}

mirror : List a -> List a
mirror xs = let xs' = rev xs in
                app xs xs'

\end{SaveVerbatim}
\useverb{letb} 

\noindent
We can also pattern match in \texttt{let} bindings. For example, we can extract
fields from a record as follows, as well as by pattern matching at the top level:

\begin{SaveVerbatim}{letp}

data Person = MkPerson String Int

showPerson : Person -> String
showPerson p = let MkPerson name age = p in
                   name ++ " is " ++ show age ++ " years old"

\end{SaveVerbatim}
\useverb{letp} 
\subsubsection{\texttt{case} expressions}

Another way of inspecting intermediate values of \emph{simple} types
is to use a \texttt{case} expression.
For example, \texttt{list\_lookup}
looks up an index in a list, returning \texttt{Nothing} if the index is out
of bounds. We can use this to write \texttt{lookup\_default}, which
looks up an index and returns a default value if the index is out of bounds:

\begin{SaveVerbatim}{listlookup}

lookup_default : Nat -> List a -> a -> a
lookup_default i xs def = case list_lookup i xs of
                              Nothing => def
                              Just x => x

\end{SaveVerbatim}
\useverb{listlookup} 

The \texttt{case} construct is intended for simple analysis
of intermediate expressions to avoid the need to write auxiliary functions, and is
also used internally to implement pattern matching \texttt{let} and lambda bindings. 
It will \emph{only} work if:

\begin{itemize}
\item Each branch \emph{matches} a value of the same type, and \emph{returns} a
value of the same type.
\item The type of the result is ``known''. i.e. the type of the expression can be
determined \emph{without} type checking the \texttt{case}-expression itself. 
\end{itemize}

\subsubsection{The \texttt{with} rule}

Since types can depend on values, the form of some arguments can be determined
by the value of others. For example, if we were to write down the implicit
length arguments to \texttt{(++)}, we'd see that the form of the length argument was
determined by whether the vector was empty or not:

\begin{SaveVerbatim}{appdep}

(++) : Vect a n -> Vect a m -> Vect a (n + m)
(++) {n=O}   []        [] = []
(++) {n=S k} (x :: xs) ys = x :: xs ++ ys

\end{SaveVerbatim}
\useverb{appdep}

\noindent
If \texttt{n} was a successor in the \texttt{[]} case, or zero in the 
\texttt{::} case, the definition
would not be well typed.

Often, we need to match on the result of an intermediate computation
with a dependent type.
\Idris{} provides a construct for this, the \texttt{with} rule, 
inspired by views in \Epigram~\cite{McBride2004a},
which takes account of the
fact that matching on a value in a dependently typed language can affect what
we know about the forms of other values. 

For example, a \texttt{Nat} is either even or odd. 
If it is even it will
be the sum of two equal \texttt{Nat}s. Otherwise, it is the sum of two equal \texttt{Nat}s 
plus one:

\begin{SaveVerbatim}{parity}

data Parity : Nat -> Set where
   even : Parity (n + n)
   odd  : Parity (S (n + n))

\end{SaveVerbatim}
\useverb{parity}

\noindent
We say \texttt{Parity} is a \emph{view} of \texttt{Nat}. 
It has a \emph{covering function} which tests whether
it is even or odd and constructs the predicate accordingly.

\begin{SaveVerbatim}{parityty}

parity : (n:Nat) -> Parity n

\end{SaveVerbatim}
\useverb{parityty}

\noindent
Using this, we can write a function which converts a natural number to a list
of binary digits (least significant first) as follows, using the \texttt{with}
rule:

\begin{SaveVerbatim}{natToBin}

natToBin : Nat -> List Bool
natToBin O = Nil
natToBin k with (parity k)
   natToBin (j + j)     | even = False :: natToBin j
   natToBin (S (j + j)) | odd  = True  :: natToBin j

\end{SaveVerbatim}
\useverb{natToBin}

\noindent
The value of the result of \texttt{parity k} affects the form of \texttt{k}, 
because the result
of \texttt{parity k} depends on \texttt{k}. 
So, as well as the patterns for the result of the
intermediate computation (\texttt{even} and \texttt{odd}) right of the 
\texttt{$\mid$}, we also write how
the results affect the other patterns left of the $\mid$. Note that there is a
function in the patterns (\texttt{+}) and repeated occurrences of \texttt{j} --- 
this is allowed
because another argument has determined the form of these patterns.




\section{The Core Type Theory}

\label{sect:typechecking}

High level \Idris{} programs, as described in Section \ref{sect:hll}, are 
\remph{elaborated} to a small core language, \TT{}, then type checked. 
\TT{} is a dependently typed $\lambda$-calculus with inductive families
and pattern matching definitions similar to UTT~\cite{luo1994}, and building
on an earlier implementation, \Ivor{}~\cite{Brady2006b}.
The
core language is kept as small as is reasonably possible, which has several advantages: it is
easy to type check, since type checking dependent type theory is well understood
~\cite{loh2010tutorial}; and it is easy to transform, optimise and compile. Keeping
the core language small increases confidence that these important components of
the language are correct. In this section, we describe \TT{} and
its semantics.

\subsection{\TT{} syntax}

The syntax of \TT{} expressions is given in Figure \ref{ttsyn}. This defines:

\begin{itemize}
\item \demph{Terms}, $\vt$, which can be variables, bindings, applications or constants.
\item \demph{Bindings}, $\vb$, which can be lambda abstractions, let bindings, or function spaces.
\item \demph{Constants}, $\vc$, which can be integer or string literals, or $\Type_i$, the
type of types, and may be extended with other primitives. 
\end{itemize}

The function space $\all{\vx}{\vS}\SC\vT$ may also be written as $(\vx\Hab\vS)\to\vT$,
or $\vS\to\vT$ if $\vx$ is not free in $\vT$, to make the notation more consistent with
the high level language. Universe levels on types ($\Type_i$) may be left implicit and
inferred by the machine~\cite{pollack1990implicit}.

\FFIG{
\begin{array}{rll@{\hg}rll}
\mbox{Terms,}\;\vt ::= & \vc & (\mbox{constant}) &
\;\mbox{Binders,}\;\vb ::= & \lam{\vx}{\vt} & (\mbox{abstraction}) \\

 \mid  & \vx & (\mbox{variable}) &
 \mid & \LET\;\vx\defq\vt\Hab\vt & (\mbox{let binding}) \\

 \mid   & \vb\SC\;\vt & (\mbox{binding}) &
 \mid & \all{\vx}{\vt} & (\mbox{function space}) \\

 \mid   & \vt\;\vt & (\mbox{application}) &
% \mid & \pat{\vx}{\vt} & (\mbox{pattern variable}) \\
% & & &
% \mid & \pty{\vx}{\vt} & (\mbox{pattern type}) \\
\medskip\\
\mbox{Constants,}\;\vc ::= & \Type_i & (\mbox{type universes}) \\
   \mid & \vi & (\mbox{integer literal}) \\
   \mid & \VV{str} & (\mbox{string literal}) \\
\end{array}
}
{\TT{} expression syntax}
{ttsyn}

A \TT{} program is a collection of \demph{inductive family} definitions (Section 
\ref{sect:inductivefams}) and \demph{pattern matching} function definitions (Section
\ref{sect:patdefs}), as well as primitive operators on constants. 
Before defining these, let us define the semantics of \TT{}
expressions.

\subsection{\TT{} semantics}

The static and dynamic semantics of \TT{} are defined mutually, since
evaluation relies on a well-typed term, and type checking relies on 
evaluation. Everything is defined relative
to a context, $\Gamma$. The empty context
is valid, as is a context extended with a $\lambda$, $\forall$ or
$\LET$ binding:

\DM{
\Axiom{\proves\RW{valid}}
\hg
\Rule{\Gamma\proves\vS\Hab\Type_i}
{\Gamma;\lam{\vx}{\vS}\proves\RW{valid}}
\hg
\Rule{\Gamma\proves\vS\Hab\Type_i}
{\Gamma;\all{\vx}{\vS}\proves\RW{valid}}
\hg
\Rule{\Gamma\proves\vS\Hab\Type_i\hg\Gamma\proves\vs\Hab\vS}
{\Gamma;\LET\;\vx\;\defq\;\vs\Hab\vS\proves\RW{valid}}
}

\subsubsection{Evaluation}

\label{sect:evaluation}

Evaluation of \TT{} is defined by contraction schemes, given in Figure
\ref{ttcontract}. \demph{Contraction}, relative to a context $\Gamma$, is given
by one of the following schemes:

\begin{itemize}
\item $\beta$-contraction, which substitutes a value applied to a $\lambda$-binding for
the bound variable. 
%We define $\beta$-contraction simply by replacing the $\lambda$-binding
%with a $\LET$ binding.
\item $\delta$-contraction, which replaces a $\LET$ bound variable with its value.
\end{itemize}

\noindent
The following contextual closure rule reduces a $\LET$ binding by creating
a $\delta$-reducible expression:

\DM{
\Rule{\Gamma;\LET\;\vx\defq\vs\Hab\vS\proves\vt\leadsto\vu}
{\Gamma\proves\LET\;\vx\defq\vs\Hab\vS\SC\vt\leadsto\vu}
}

\demph{Reduction} ($\reduces$) is the structural closure of contraction, and evaluation
is ($\reducesto$) is the transitive closure of reduction. \demph{Conversion} ($\converts$)
is the smallest equivalence relation closed under reduction. If $\Gamma\proves\vx\converts\vy$
then $\vy$ can be obtained from $\vx$ by a finite, possibly empty, sequence of
contractions and reversed contractions. Terms which are convertible are also said to
be definitionally equal.
The evaluator can also be extended by defining pattern matching functions, which
will be described in more detail in Section \ref{sect:patdefs}. In principle, pattern
matching functions can be understood as extending the core language with high level
reduction rules.

\FFIG{
\begin{array}{lc}
\beta\mathrm{-contraction} &
\Axiom{
\Gamma\proves(\lam{\vx}{\vS}\SC\vt)\;\vs\leadsto
\vt[\vs/\vx]
%\LET\;\vx\defq\vs\Hab\vS\SC\vt
} \\
\delta\mathrm{-contraction} &
\Axiom{
\Gamma;\LET\;\vx\defq\vs\Hab\vS;\Gamma'
\proves
\vx\leadsto\vs
}
\end{array}
}
{\TT{} contraction schemes}
{ttcontract}



\subsubsection{Typing rules}

\label{sect:typerules}

The type inference rules for \TT{} expressions are given in Figure
\ref{typerules}.  These rules use the \demph{cumulativity} relation, defined in
Figure \ref{cumul}. The type of types, $\Type_i$ is parameterised by a universe
level, to prevent Girard's paradox~\cite{coquand1986analysis}.  There is an
infinite hierarchy of predicative universes.  Cumulativity allows programs at
lower universe levels to inhabit higher universe levels. In practice, universe levels
can be left implicit (and will be left implicit in the rest of this paper) ---
the type checker generates a graph of constraints between universe levels (such
as that produced by the $\mathsf{Forall}$ typing rule) and checks that there
are no cycles. Otherwise, the typing rules are standard and type inference can
be implemented in the usual way~\cite{loh2010tutorial}.

\FFIG{\begin{array}{c}
\mathsf{Type}\;
\Rule{\Gamma\proves\RW{valid}}
{\Gamma\vdash\Type_n\Hab\Type_{n+1}}
\\
\mathsf{Const}_1\;
\Rule{\Gamma\proves\RW{valid}}
{\Gamma\vdash\vi\Hab\TC{Int}}
\hg
\mathsf{Const}_2\;
\Rule{\Gamma\proves\RW{valid}}
{\Gamma\vdash\VV{str}\Hab\TC{String}}
\\
\mathsf{Const}_3\;
\Rule{\Gamma\proves\RW{valid}}
{\Gamma\vdash\TC{Int}\Hab\Type_0}
\hg
\mathsf{Const}_4\;
\Rule{\Gamma\proves\RW{valid}}
{\Gamma\vdash\TC{String}\Hab\Type_0}
\\
%\mathsf{Pat}_1\;
%\Rule{(\pat{\vx}{\vS})\in\Gamma}
%{\Gamma\vdash\vx\Hab\vS}
%\hg
%\mathsf{Pat}_2\;
%\Rule{(\pty{\vx}{\vS})\in\Gamma}
%{\Gamma\vdash\vx\Hab\vS}
\\
\mathsf{Var}_1\;
\Rule{(\lam{\vx}{\vS})\in\Gamma}
{\Gamma\vdash\vx\Hab\vS}
\hg
\mathsf{Var}_2\;
\Rule{(\all{\vx}{\vS})\in\Gamma}
{\Gamma\vdash\vx\Hab\vS}
\hg
\mathsf{Val}\;
\Rule{(\LET\;\vx\defq\vs\Hab\vS)\in\Gamma}
{\Gamma\vdash\vx\Hab\vS}
\\
\mathsf{App}\;
\Rule{\Gamma\vdash\vf\Hab\fbind{\vx}{\vS}{\vT}\hg\Gamma\vdash\vs\Hab\vS}
{\Gamma\vdash\vf\;\vs\Hab\vT[\vs/\vx]} % \LET\;\vx\Hab\vS\;\defq\;\vs\;\IN\;\vT}
\\
\mathsf{Lam}\;
\Rule{\Gamma;\lam{\vx}{\vS}\vdash\ve\Hab\vT\hg\Gamma\proves\fbind{\vx}{\vS}{\vT}\Hab\Type_n}
{\Gamma\vdash\lam{\vx}{\vS}.\ve\Hab\fbind{\vx}{\vS}{\vT}}
\\
\mathsf{Forall}\;
\Rule{\Gamma;\all{\vx}{\vS}\vdash\vT\Hab\Type_m\hg\Gamma\vdash\vS\Hab\Type_n}
{\Gamma\vdash\fbind{\vx}{\vS}{\vT}\Hab\Type_p}
\;(\exists\vp.\vm\le\vp,\;\vn\le\vp)
\\
\mathsf{Let}\;
\Rule{\begin{array}{c}\Gamma\proves\ve_1\Hab\vS\hg
      \Gamma;\LET\;\vx\defq\ve_1\Hab\vS\proves\ve_2\Hab\vT\\
      \Gamma\proves\vS\Hab\Type_n\hg
      \Gamma;\LET\;\vx\defq\ve_1\Hab\vS\proves\vT\Hab\Type_n\end{array}
      }
{\Gamma\vdash\LET\;\vx\defq\ve_1\Hab\vS\SC\;\ve_2\Hab
   \vT[\ve_1/\vx]}   
%\Let\;\vx\Hab\vS\defq\ve_1\;\IN\;\vT}
\\

\mathsf{Conv}\;
\Rule{\Gamma\proves\vx\Hab\vA\hg\Gamma\proves\vA'\Hab\Type_n\hg
      \Gamma\proves\vA\cumul\vA'}
     {\Gamma\proves\vx\Hab\vA'}
\end{array}
}
{Typing rules for \TT{}}
{typerules}

\FFIG{
\begin{array}{c}
\Rule{\Gamma\proves\vS\converts\vT}
{\Gamma\proves\vS\cumul\vT}
\hg
\Axiom{\Gamma\proves\Type_n\cumul\Type_{n+1}}
\\
\Rule{\Gamma\proves\vR\cumul\vS\hg\Gamma\proves\vS\cumul\vT}
{\Gamma\proves\vR\cumul\vT}
\\
\Rule{\Gamma\proves\vS_1\converts\vS_2\hg\Gamma;\vx\Hab\vS_1\proves\vT_1\cumul\vT_2}
{\Gamma\proves\all{\vx}{\vS_1}\SC\vT_1\cumul\all{\vx}{\vS_2}{\vT_2}}
\end{array}
}
{Cumulativity}
{cumul}


\subsection{Inductive Families}

\label{sect:inductivefams}

Inductive families \cite{dybjer1994inductive} are a form of simultaneously
defined collection of algebraic data types which can be parameterised over
\remph{values} as well as types.  An inductive family is declared 
in a similar style to a Haskell GADT declaration~\cite{pj2006gadts}
as
follows, using vector notation, $\tx$, to indicate a
sequence of zero or more $\vx$ (i.e., $\vx_1,\vx_2,\ldots,\vx_n$):

\DM{
\AR{
\Data\hg\TC{T}\;(\tx\Hab\ttt)\Hab\vt\hg\Where\hg
\DC{c}_1\Hab\vt\;\mid\;\ldots\;\mid\;\DC{c}_n\Hab\vt
}
}

Constructors may take recursive arguments in the family $\TC{T}$, which may be
higher order. Higher order recursive arguments may be computed from any type
which does not involve $\TC{T}$ itself. This restriction is known as
\demph{strict positivity} and ensures that recursive arguments of the
constructor are structurally smaller than the value itself.

For example, the \Idris{} data type \tTC{Nat} would be declared in \TT{} as follows:

\DM{
\Data\hg\Nat\Hab\Type\hg\Where\hg\Z\Hab\Nat\;\mid\;\suc\Hab\fbind{\vk}{\Nat}{\Nat}
}

A data type may have zero or more parameters (which are invariant
across a structure) and a number of indices, given by the type. For
example, the \TT{} equivalent of \tTC{List} is parameterised over its element type:

\DM{
\AR{
\Data\hg\List\Hab(\va\Hab\Type)\to\Type\hg\Where\\
\hg\hg
\ARd{
& \nil\Hab\List\;\va\\
\mid & (\cons)\Hab\fbind{\vx}{\va}{\fbind{\vxs}{\List\;\va}{\List\;\va}}
}
}
}

Types can be
parameterised over values. Using this, we can declare the type of
vectors (lists with length), where the empty list is statically known
to have length zero, and the non empty list is statically known to
have a non zero length. The \TT{} equivalent of \tTC{Vect} is parameterised over its element type,
like $\List$, but \remph{indexed} over its length. Note also that the length
index $\vk$ is given \remph{explicitly}.

\DM{
\AR{
\Data\hg\Vect\Hab\Nat\to(\va\Hab\Type)\to\Type\hg\Where \\
\hg\hg\ARd{
& \nil\Hab\Vect\;\Z\;\va\\
\mid & (\cons)\Hab\fbind{\vk}{\Nat}{
\fbind{\vx}{\va}{\fbind{\vxs}{\Vect\;\vk\;\va}{\Vect\;(\suc\;\vk)\;\va}}
}
}
}
}

\subsection{Pattern matching definitions}

\label{sect:patdefs}

%\subsubsection{Syntax}


A pattern matching definition for a function named $\FN{f}$ takes the following form,
consisting of a type declaration followed by one or more pattern clauses:

\DM{
\AR{
\FN{f}\Hab\vt\\
\pat{\tx_1}{\ttt_1}\SC\FN{f}\;\ttt_1\;=\;\vt_1\\
\ldots\\
\pat{\tx_n}{\ttt_n}\SC\FN{f}\;\ttt_n\;=\;\vt_n\\
}
}

A pattern clause consists of a list of pattern variable bindings, introduced by
$\RW{var}$,
and a left and right hand side, both of which
are \TT{} expressions. Each side is type checked relative to the variable bindings,
and the types of each side must be convertible. Additionally, the
left hand side must take the form of $\FN{f}$ applied to a number of \TT{} expressions,
and the number of arguments must be the same in each clause. The right hand
sides may include applications of $\FN{f}$, i.e. pattern matching definitions may
be recursive. Termination analysis is implemented separately. The validity of a pattern
clause is defined by the following rule:

\DM{
\Rule{
\Gamma;\lam{\tx}{\tU}\proves\FN{f}\;\tts\Hab\vS\hg
\Gamma;\lam{\tx}{\tU}\proves\ve\Hab\vT\hg
\Gamma\proves\vS\converts\vT}
{
\Gamma\proves\pat{\tx}{\tU}\SC\FN{f}\;\tts\;=\;\ve\;\RW{valid}
}
}

A valid pattern matching definition effectively extends \TT{} with a new
constant, with the given type (extending the initial typing rules given in
Section \ref{sect:typerules}) and reduction behaviour (extending the initial
reduction rules given in Section \ref{sect:evaluation}). 
Patterns are separated into the accessible patterns (variables and constructor
forms which may be inspected) and inaccessible patterns, following
Agda~\cite{norell2007thesis} then implemented by compilation into case
trees~\cite{Augustsson1985}.

% \subsubsection{Semantics}
% 
% Matching on an expression proceeds by comparing the expression to
% each match clause in order, resulting in either:
% 
% \begin{itemize}
% \item \demph{Success}, with pattern variables mapping to expressions
% \item \demph{Failure}, with matching continuing by proceeding to the next match clause
% \item Matching being \demph{blocked}, for example by attempting to match a variable
% against a constructor pattern. In this case, no reduction occurs, because instantiating
% the variable may provide enough information for the clause to match.
% \end{itemize}
% 
% % Pat(x args) => x (Pat' args)
% % Pat (x) => x
% % Pat _ => _
% 
% In order to implement pattern matching, we must separate the \remph{accessible} and
% \remph{inaccessible} patterns. A pattern is \demph{accessible} (that is, it is possible
% to match against it) if it is constructor headed, or a variable. Inaccessible patterns
% are converted to ``match anything'' patterns. Clauses are converted to matchable pattern
% clauses with the $\MO{Clause}$ operation, given in Figure \ref{mkclause}.
% We extend the vector notation to meta-operations: the notation $\vec{\MO{Pat}}$ lifts the
% $\MO{Pat}$ operation across a list of arguments.
% 
% \FFIG{
% \AR{
% \PA{\A}{
% & \MO{Clause} & (\pat{\tx}{\tU}\SC\FN{f}\;\tts\;=\;\ve)
%  & \MoRet{\FN{f}\;(\vec{\MO{Pat}}\;\tts)\;=\;\ve}
% }
% \medskip\\
% \PA{\A}{
% & \MO{Pat} & (\vx\;\tts) & \MoRet{\vx\;(\vec{\MO{Pat}}\;\tts) 
%   \hg\mbox{(if $\MO{Con}\;\vx$ and $\MO{Arity}\;\vx = \MO{Length}\;\tts$)}} \\
% & \MO{Pat} & \vx & \MoRet{\vx} \hg\mbox(if $\vx$ is a pattern variable)\\
% & \MO{Pat} & \cdot & \MoRet{\_} \hg\mbox(in all other cases)\\
% }
% }
% }
% {Building matchable pattern clauses}
% {mkclause}
% 
% 
% % Match (c args, c args') => Match' (args, args')
% % Match (c args, x)       => Blocked
% % Match (x, t)            => Success (x => t)
% % Match (_, t)            => Success ()  
% 
% \newcommand{\Blocked}{\mathsf{Blocked}}
% \newcommand{\Success}{\mathsf{Success}}
% \newcommand{\Failure}{\mathsf{Failure}}
% 
% Pattern matching is invoked by the evaluator on encountering a function with
% a pattern matching definition applied to some arguments. If successful, pattern
% matching returns a new expression, which can then be reduced further.
% Figure \ref{pmatch} gives the semantics of pattern matching a collection of 
% matchable clauses
% $\tc$ against an expression $\ve$. $\MO{Match}$ attempts to match each clause in
% turn against $\ve$. If matching a clause returns $\Success$ or $\Blocked$, then
% matching terminates. If matching a clause returns $\Failure$, then matching proceeds
% with the next clause. The algorithm uses the following meta-operations:
% 
% \begin{itemize}
% \item $\MO{Con}\;\vx$, which returns true if $\vx$ is a constructor name
% \item $\MO{Arity}\;\vx$, which, if $\vx$ is a constructor name, returns the number of arguments
% $\vx$ requires.
% \item $\MO{Length}\;\tts$, which returns the length of the vector $\tts$
% \end{itemize}
% 
% 
% \FFIG{
% \AR{
% \PA{\A\A}{
% & \MO{MatchArg} & (\vx\;\tts) & (\vx\;\tts')
%     & \MoRet{\vec{\MO{MatchArg}}\;\tts\;\tts'\hg\mbox{(if $\MO{Con}\;\vx$)}}\\
% & \MO{MatchArg} & (\vx\;\tts) & \vx & \MoRet{\Blocked} \\
% & \MO{MatchArg} & \vx & \vt & \MoRet{\Success\;(\vx\mapsto\vt)} \\
% & \MO{MatchArg} & \_ & \vt & \MoRet{\Success\;()} \\
% & \MO{MatchArg} & \cdot & \cdot & \MoRet{\Failure} 
% }
% \medskip\\
% \PA{\A\A}{
% & \MO{MatchClause} & (\vf\;\tp = \ve) & (\vf\;\tts) &
%  \MoRet{
%  \AR{
%  \Success\;\ve[\tx/\tv]\hg\mbox{(if $\vec{\MO{MatchArg}}\;\tp\;\tts \mq \Success\;(\tx,\tv)$)}\\
%  \Blocked\;\;\;\hg\hg\mbox{(if $\vec{\MO{MatchArg}}\;\tp\;\tts \mq \Blocked$)}\\
%  \Failure\;\;\;\;\;\hg\hg\mbox{(if $\vec{\MO{MatchArg}}\;\tp\;\tts \mq \Failure$)}\\
%  }
%  } \\
% & \MO{MatchClause} & (\vf\;\tp = \ve) & \cdot & \MoRet{\Failure} \\
% }
% \medskip\\
% \PA{\A\A}{
% & \MO{Match} & (\vc ; \tc) & \ve &
%  \MoRet{
%  \AR{
%  \Success\;\vx\;\;\hg\mbox{(if $\MO{MatchClause}\;\vc\;\ve \mq \Success\;\vx)$} \\
%  \Blocked\hg\;\;\;\;\mbox{(if $\MO{MatchClause}\;\vc\;\ve \mq \Blocked)$} \\
%  \MO{Match}\;\tc\;\ve\hg\mbox{(otherwise)}
%  }
%  } \\
% & \MO{Match} & \cdot & \ve & \MoRet{\Failure}
% }
% }
% }
% {Pattern matching semantics}
% {pmatch}
% 
% By convention, we write the application of a meta-operation $\MO{Op}$ 
% across a vector $\tx$ as $\vec{\MO{Op}}\;\tx$.
% In practice, for efficiency, pattern matching is implemented by compiling the match clauses to
% a tree of case expressions~\cite{Augustsson1985}. 

\subsection{Metatheory}

\FFIG{
\begin{array}{ll}
\mbox{\textbf{Church-Rosser}} &
\begin{array}{ll}
\mbox{If}\hg\Gamma\proves\vs\converts\vt \\
\mbox{then there is a common reduct $\vr$ such that}\;\\
\Gamma\proves\vs\reducesto\vr\hg\Gamma\proves\vt\reducesto\vr
\end{array}
\medskip
\\
\mbox{\textbf{Subject Reduction}} &
\begin{array}{ll}
\mbox{If} & \Gamma\proves\vs\Hab\vS\hg\Gamma\proves\vs\reduces\vt\\
\mbox{then}& \Gamma\proves\vt\Hab\vS
\end{array}
\medskip
\\
\mbox{\textbf{Cut}} &
\begin{array}{ll}
\mbox{If} & \Gamma_0,\LET\;\vx\defq\vs\Hab\vS,\Gamma_1\proves\vt\Hab\vT\\
\mbox{then} & \Gamma_0, \Gamma_1[\vs/\vx]\proves\vt[\vs/\vx]\Hab\vT[\vs/\vx]
\end{array}
\medskip
\\
\mbox{\textbf{Strengthening}} &
\begin{array}{ll}
\mbox{If} & \Gamma_0,\mathcal{B}(\vx, \vT),\Gamma_1\proves\vs\Hab\vS
\hg\vx\not\in\Gamma_1,\vs,\vS\\
\mbox{then} & \Gamma_0,\Gamma_1\proves\vs\Hab\vS\\
(\mbox{where} & \mathcal{B}(\vx, \vT) \proves \vx\Hab\vT)
\end{array}
\\
\end{array}
}
{Metatheoretic properties of \TT{}}
{metatheory}

We conjecture that \TT{} respects the usual metatheoretic properties, as shown in
Figure \ref{metatheory}.  Specifically: the \demph{Church-Rosser} property
(i.e. that distinct reduction sequences lead to the same normal form);
\demph{subject reduction} (i.e. that computation preserves type); the
\demph{cut} property (i.e. that $\RW{let}$-bound terms may be substituted into
their scope); and \demph{strengthening} (i.e. that removing unused definitions
from scope does not affect type checking).

One property which is noticeably absent is \demph{strong normalisation}.
Although desirable to preserve termination and decidability of type checking,
we allow diverging terms for pragmatic reasons as \Idris{} is primarily a
programming language rather than a theorem prover. Nevertheless, we do
\emph{check} for totality in order to be sure which terms are safe to
evaluate at compile time.

\subsection{Totality checking}

In order to ensure termination of type checking we must distinguish terms for
which evaluation definitely terminates, and those which may diverge. \TT{}
takes a simple but pragmatic and effective approach to termination checking:
any functions which do not satisfy a syntactic constraint on recursive calls
are marked as \emph{partial}. Additionally, any function which calls a partial
function or uses a data type which is not strictly positive is also marked as
partial. We use the size change principle~\cite{Lee2001} to determine whether
(possibly mutually defined) recursive functions are guaranteed to
terminate.
\TT{} also marks functions which do not cover
all possible inputs as partial. This totality checking is independent of the
rest of the type theory, and can be extended.

This approach, separating the termination requirement from the type theory,
means that an \Idris{} programmer makes the decision about the importance of
totality for each function rather than having the totality requirement imposed
by the type theory.

\subsection{From \Idris{} to \TT{}}

\TT{} is a very small language, consisting only of data declarations and pattern matching
function definitions. There is no significant innovation in the design of \TT{}, and this
is a deliberate choice --- it is a combination of small, well-understood components.
The kernel of the \TT{} implementation, consisting of a type checker, evaluator and
pattern match compiler, is less than 1000 lines of Haskell code. If we are confident
in the correctness of the kernel of \TT{}, and any higher level language feature
can be translated into \TT{}, we can be more confident of the correctness of the high
level language implementation than if it were implemented directly.

The process of elaborating a high level language into a core type theory like \TT{} is,
however, less well understood, and presents several challenges depending on the
features of the high level language. 
%Elaboration has been implemented in various different ways, 
%for example in Agda~\cite{norell2007thesis} and Epigram~\cite{McBride2004a}.

\subsubsection{Example Elaboration}

Recall the following \Idris{} function:

\useverb{vadd} 

\noindent
In order to elaborate
this to \TT{}, we must resolve the implicit arguments, and make the type class explicit.
The first step is to make the implicit arguments explicit, using a placeholder
to stand for the arguments we have not yet resolved. The type class argument is
also treated as an implicit argument, to which we give the name \texttt{c}:

\begin{SaveVerbatim}{vAddImp}

vAdd : (a : _) -> (n : _) -> Num a -> Vect n a -> Vect n a -> Vect n a
vAdd _ _ c (Nil _)         (Nil _)         = (Nil _)
vAdd _ _ c ((::) _ _ x xs) ((::) _ _ y ys) 
                = (::) _ _ ((+) _ x y) (vAdd _ _ _ xs ys)

\end{SaveVerbatim}
\useverb{vAddImp} 

Next, we resolve the implicit arguments. Each implicit argument can only take
one value for this program to be type correct --- these are solved by a unification
algorithm:

\begin{SaveVerbatim}{vAddImpSolve}

vAdd : (a : Type) -> (n : Nat) -> Num a -> Vect n a -> Vect n a -> Vect n a
vAdd a O     c (Nil a)         (Nil a)         = Nil a
vAdd a (S k) c ((::) a k x xs) ((::) a k y ys) 
                = (::) a k ((+) c x y) (vAdd a k c xs ys)

\end{SaveVerbatim}
\useverb{vAddImpSolve} 

Finally, to build the \TT{} definition, we need to find the type of each
pattern variable and state it explicitly. This leads to the following
\TT{} definition, switching to \TT{} notation from the ASCII \Idris{}
syntax:

\DM{
\AR{
\FN{vAdd} \Hab(\va\Hab\Type)\to(\vn\Hab\Nat)\to\TC{Num}\;\va\to
  \Vect\;\vn\;\va\to\Vect\;\vn\;\va\to\Vect\;\vn\;\va\\
\RW{var}\;\va\Hab\Type,\;\vc\Hab\TC{Num}\;\va\SC\\
\hg\FN{vAdd} \; \va \; \Z \; \vc \; (\DC{Nil}\;\va) \; (\DC{Nil}\;\va) \; 
\cq\;\DC{Nil}\;\va \\
\RW{var}\;\AR{
\va\Hab\Type,\;\vk\Hab\Nat,\;\vc\Hab\TC{Num}\;\va,\\
\vx\Hab\va,\;\vxs\Hab\Vect\;\vk\;\va,\;
\vy\Hab\va,\;\vys\Hab\Vect\;\vk\;\va
\SC
\\
\FN{vAdd} \; \va \; (\suc\;\vk) \; \vc 
  \; ((\DC{::})\;\va\;\vk\;\vx\;\vxs) 
  \; ((\DC{::})\;\va\;\vk\;\vy\;\vys) 
   \\
   \hg\hg\cq\:((\DC{::})\;\va\;\vk\;((+)\;\vc\;\vx\;\vy)\; (\FN{vAdd}\;\va\;\vk\;\vc\;\vxs\;\vys))
}
}
}

\subsubsection{An Observation: Programming vs Theorem Proving}

\Idris{} programs may contain several high level constructs not present in \TT{}, such
as implicit arguments, type classes, \texttt{where} clauses, pattern matching \texttt{let}
and \texttt{case} constructs. We would like the high level language to be as expressive
as possible, while remaining possible to translate to \TT{}.

Before considering how to achieve this, we make an observation about the distinction between
programming and theorem proving with dependent types, and appropriate mechanisms for
constructing programs and theorems:

\begin{itemize}
\item \remph{Pattern matching} is a convenient abstraction for humans to write
programs, in that it allows a programmer to express exactly the computational
behaviour of a function.
\item \remph{Tactics}, such as those used in the Coq theorem
prover~\cite{Bertot2004}, are a convenient abstraction for building proofs and
programs by \remph{refinement}.
\end{itemize}

The idea behind the \Idris{} elaborator, therefore, is to use the high level
program to direct \demph{tactics} to build \TT{} programs by refinement.  The
elaborator is implemented as a Haskell monad capturing proof state, with a
collection of tactics for updating and refining the proof state.  The remainder
of this paper describes this elaborator and demonstrates how it is used to
implement the high level features of \Idris{}.





%\section{Proof State}

\subsection{Tactics}

\subsection{Unification}

\subsection{The Elaboration DSL}


\newcommand{\ttinterp}[1]{\mathcal{E}\interp{#1}}

\section{Elaborating \Idris{}}

An \Idris{} program consists of a series of declarations --- data types, functions,
type classes and instances. In this section, we describe how these high level declarations
are translated into a \TT{} program consisting of inductive families and pattern matching
function definitions. We will need to work at the \remph{declaration} level, and at
the \remph{expression} level, defining the following meta-operations:

\begin{itemize}
\item $\ttinterp{\cdot}$, which builds a \TT{} expression from an \Idris{} expression
\item $\MO{Elab}$, which processes a top level \Idris{} declaration by generating
one or more \TT{} declarations.
\end{itemize}

\subsection{The Development Calculus \TTdev}

We build \TT{} expressions by using high level \Idris{} expressions to
direct a tactic based theorem prover, which builds the \TT{} expressions
step by step, by refinement. In order to build expressions in this way,
the type theory needs to support
\remph{incomplete} terms, and a method for term construction. 
To achieve this, we extend \TT{} with \remph{holes},
calling the extended calculus \TTdev{}.
Holes stand for the parts of programs which have not yet been
instantiated; this largely follows the \Oleg{} development
calculus~\cite{McBride1999}.

The basic idea is to extend the syntax for binders with a \remph{hole}
binding and a \remph{guess} binding. 
These extensions are given in Figure \ref{ttdev}.
The \remph{guess} binding is
similar to a $\LET$ binding, but without any computational force,
i.e. the bound names do not reduce.
Using binders to represent holes is useful in a dependently typed setting since
one value may determine another. Attaching a ``guess'' to a binder ensures that
instantiating one such value also instantiates all of its dependencies. The
typing rules for binders ensure that no $?$ bindings leak into types.

\FFIG{
\AR{
\vb ::= \ldots 
 \:\mid\: \hole{\vx}{\vt} \:\:(\mbox{hole binding}) \:\:
 \:\mid\: \guess{\vx}{\vt}{\vt} \:\:(\mbox{guess})
\medskip\\
\Rule{
\Gamma;\hole{\vx}{\vS}\proves\ve\Hab\vT
}
{
\Gamma\proves\hole{\vx}{\vS}\SC\ve\Hab\vT
}
\hspace*{0.1cm}\vx\not\in\vT
\hspace*{0.1in}\mathsf{Hole}
\hg
\Rule{
\Gamma;\guess{\vx}{\vS}{\ve_1}\proves\ve_2\Hab\vT
}
{
\Gamma\proves\guess{\vx}{\vS}{\ve_1}\SC\ve_2\Hab\vT
}
\hspace*{0.1cm}\vx\not\in\vT
\hspace*{0.1in}\mathsf{Guess}
}
}
{\TTdev{} extensions}
{ttdev}


\subsection{Proof State}

A proof state is a tuple, $(\vC, \Delta, \vT, \vQ)$, containing:

\begin{itemize}
\item A global context, $\vC$, containing pattern matching definitions
\item A local context, $\Delta$, containing pattern bindings
\item A proof term, $\vT$, in \TTdev{}
\item A hole queue, $\vQ$
%\item \remph{Deferred} definitions, $\vD$, for introducing global metavariables
\end{itemize}

The \remph{hole queue} is a list of names of hole and guess binders in the proof term ---
we ensure that each bound name is unique. Holes essentially refer to \remph{sub goals}
in the proof.
When this queue is empty, the proof term is complete.
Creating a \TT{} expression from an \Idris{} expresson involves creating
a new proof state, with an empty proof term, and using the high level definition
to direct the building of a final proof state, with a complete proof term.

In the implementation, the proof state is captured in an elaboration monad,
\texttt{Elab}, which includes various operations for querying and updating
the proof state, manipulating terms, generating fresh names, etc. However, we will
describe \Idris{} elaboration in terms of meta-operations on the proof state,
in order to capture the essence of the elaboration process without being distracted
by implementation details. These meta-operations include: 

\begin{itemize}
\item \demph{Queries} which retrieve values from the proof state, without modifying
the state. For example, we can:
\begin{itemize}
\item Get the type of the current sub goal
\item Retrieve the local context $\Gamma$ at the current sub goal
\item Type check or normalise a term relative to $\Gamma$
\end{itemize}
\item \demph{Unification}, which unifies two terms (potentially solving sub goals) 
relative to $\Gamma$
\item \demph{Tactics} which update the proof term. Tactics operate on the sub term
at the binder specified by the head of the hole queue $\vQ$.
\item \demph{Focussing} on a specific sub goal, which brings a different sub goal to the
head of the hole queue.
%\item \demph{Deferring} a sub goal, which adds a new definition to the global context
%$\vC$ which solves the sub goal.
\end{itemize}

Elaboration of an \Idris{} expression involves creating a new proof state, running
a series of tactics to build a complete proof term, then retrieving and \remph{rechecking}
the final proof term, which must be a \TT{} program (i.e. does not contain any of the
\TTdev{} extensions).

\subsection{System State}

The system state is a tuple, $(\vC,\vA,\vI)$, containing:

\begin{itemize}
\item A global context, $\vC$, containing pattern matching definitions
\item Implicit arguments, $\vA$, recording which arguments are implicit for each global name
\item Type class instances, $\vI$, containing dictionaries for type classes
\end{itemize}

In the implementation, the system state is captured in a monad, \texttt{Idris}, and
includes additional information such as syntax overloadings,
command line options, and optimisations, which do not concern us here. Elaboration
of expressions requires access to the system state in particular in order to expand
implicit arguments and resolve type classes. 

For each global name, $\vA$ records whether its arguments are explicit, implicit,
or type class constraints.  For example, recall the declaration
of \texttt{vAdd}:

\begin{SaveVerbatim}{vAddImpT}

vAdd : Num a => Vect a n -> Vect a n -> Vect a n

\end{SaveVerbatim}
\useverb{vAddImpT} 

\noindent
Written in full, and giving each argument an explicit name, we get the
type declaration:

\begin{SaveVerbatim}{vAddImpT}

vAdd : (a : _) -> (n : _) -> (c : Num a) -> 
       (xs : Vect a n) -> (ys : Vect a n) -> Vect a n

\end{SaveVerbatim}
\useverb{vAddImpT} 

\noindent
For \tFN{vAdd}, we record that \texttt{a} and \texttt{n} are implicit, 
\texttt{c} is a constraint, and \texttt{xs} and \texttt{ys} are explicit. When
the elaborator encounters an application of \tFN{vAdd}, it knows that unless these arguments
are given explicitly, the application must be expanded.

\newcommand{\Check}{\MO{Check}_\Gamma}
\newcommand{\Eval}{\MO{Normalise}_\Gamma}
\newcommand{\Unify}{\MO{Unify}_\Gamma}

\subsection{Tactics}

% Meta-operations Check, Normalise, Unify 
In order to build \TT{} expressions from \Idris{} programs, we define a collection
of meta-operations for querying and modifying the proof state. Meta-operations
may have side-effects including failure, or updating the proof state. We have the following
primitive meta-operations:

\begin{itemize}
\item $\MO{Focus}\:\vn$, which moves $\vn$ to the head of the hole queue
\item $\MO{Check}$, which type checks an expression relative to a context
$\Gamma$. \\
$\Check\:\ve\:\mq\:(\vv,\vt)$ means that checking an expression $\ve$
returns a well typed value $\vv$ and its type $\vt$. $\MO{Check}$ will fail
if the expression is not well-typed.
\item $\MO{Normalise}$, which evaluates a well-typed expression relative to a context 
$\Gamma$.\\
$\Eval\:\ve\:\mq\:\vv$ means that evaluating the expression $\ve$ returns the
value $\vv$.
\item $\MO{Unify}$, which unifies two well-typed expressions.
\\
$\Unify\:\ve_1\:\ve_2\:\mq\:\tu$ means that unifying $\ve_1$ and $\ve_2$ produces a
mapping $\tu$ from hole names to the values the holes must be instantiated with for
$\ve_1$ and $\ve_2$ to be convertible relative to $\Gamma$ 
(i.e. for $\Gamma\proves\ve_1\converts\ve_2$ to hold). $\MO{Unify}$ will fail
if it cannot find such a mapping.
\end{itemize}

\remph{Tactics} are specifically meta-operations which operate on the sub-term given
by the hole at the head of the hole queue in the proof state. They take the following form:

\DM{
\PA{\A\A}{
\MO{Tactic}_\Gamma & \vec{\VV{args}} & \vt & \MoRet{\vt'}
}
}

A tactic takes a sequence of zero or more arguments $\vec{\VV{args}}$ followed by the sub-term $\vt$
on which it is operating. It runs relative to a context $\Gamma$ which contains all the bindings
and pattern bindings
in scope at that point in the term. The sub-term $\vt$ will either be a hole binding 
$\hole{\vx}{\vT}\SC\ve$ or a guess binding $\guess{\vx}{\vT}{\vv}\SC\ve$. The tactic returns a 
new term $\vt'$ which can take any form, provided it is well-typed. Tactics are executed by a higher
level meta-operation $\MO{RunTac}$, which locates the appropriate sub-term, replaces it with the
term returns by the tactic, updates the hole queue in the proof state, and updates holes which
have been solved by unification.

Creating and destroying holes: Claim, Fill, Solve

\DM{
\PA{\A\A}{
\MO{Claim}_\Gamma & (\vy \Hab\vS) & (\hole{\vx}{\vT}\SC\ve) & 
   \MoRet{\hole{\vx}{\vT}\SC\hole{\vy}{\vS}\SC\ve} \\
}
}

\DM{
\PA{\A\A}{
\MO{Fill}_\Gamma & \vv & (\hole{\vx}{\vT}\SC\ve) & 
   \MoRet{\RW{do}\:\AR{
   (\vv',\vT') \leftarrow \Check\:\vv\\
   \Unify\:\vT\:\vT'\\
   \RW{return}\:\guess{\vx}{\vT}{\vv'}\SC\SC\ve}
   } \\
}
}

\DM{
\PA{\A}{
\MO{Solve}_\Gamma & (\guess{\vx}{\vT}{\vv}\SC\ve) &
   \MoRet{\ve[\vv/\vx]}
}
}

Attack, Lambda, Pi, Let

Apply


%--- give unify in full, esp. as it solves sub goals? Maybe...

% Unify' G x t             = Success (x, t) if ?x : t in G
% Unify' G t x             = Success (x, t) if ?x : t in G
% Unify' G (b x. e) (b' x'. e')   = Unify' G b b'; Unify' G;b e e'[x/x']
% Unify' G ((\x.e) x) e'   = Unify' G e e' 
% Unify' G e ((\x.e') x)   = Unify' G e e' 
% Unify' G (f es) (f' es') = vs <- Unify' G f f'; Injective f
                             
% Unify' G x y             = Success () if G |- x == y
% Unify' G . .             = Failure

% Unify' G (\x : t . e) (\x : t' . e') = Unify' G t t'; Unify' G e e'
% ...


%\DM{
%}

\subsection{Elaborating Expressions}


\IdrisM{}, a subset of \Idris{} not including syntactic sugar (e.g. pairs, do notation, etc).

Implicit and type class arguments? Expanded at the application site (we need to know
it's the global name after all and we do that by type).

\subsection{Elaborating Data Types}

\subsection{Elaborating Pattern Matching}



\section{Reversing Elaboration}

\label{sect:delab}

As well as translating from \Idris{} to \TT{}, so that programs can be type
checked and evaluated, it is valuable to define the reverse transformation. This
serves two principal purposes:

\begin{itemize}
\item To assist the user, it is preferable that the results of evaluation, and any
error messages produced by the elaborator, are presented in \Idris{} syntax
rather than \TT{}.
\item For correctness, we would like to ensure as far as possible that the 
result of elaboration is equivalent to the original program. Informally, we can
achieve this by checking that reversing the elaboration process yields the original
program (with implicit arguments expanded).
\end{itemize}

\noindent
In this section, we describe the process for reversing elaboration and the required
properties of the elaboration process as a whole. Fortunately, translating from
\TT{} to \Idris{} is significantly easier than \Idris{} to \TT{}, because it is
primarily \emph{erasing} information.

\subsection{From \TT{} to \Idris{}}

We define a meta-operation $\uninterp{\vt}$, which converts a \TT{} expression
$\vt$ to an \Idris{} expression which would elaborate to $\vt$:

\DM{
\AR{
\begin{array}{rl}
\uninterp{\vx} &\mq\:\vx\\
\uninterp{\vc} &\mq\:\vc\\
\uninterp{\vx\;\ta} &\mq\:\vx\;(\vec{\MO{Impl}}\:\vx\:\ta) \\
\uninterp{\vf\:\va} &\mq\:\uninterp{\vf}\:\uninterp{\va}\\
\uninterp{\lam{\vx}{\vT}\SC\ve} &\mq \:\ilam{\vx}\uninterp{\ve}  \\
\uninterp{\all{\vx}{\vT}\SC\ve} &\mq \:\piexp{\vx}{\uninterp{\vT}}\uninterp{\ve}  \\
\uninterp{\LET\:\vx\defq\vt\Hab\vT\SC\ve} &\mq  
\ilet{\vx}{\uninterp{\vt}}\uninterp{\ve}\\
\end{array}
\medskip\\
\begin{array}{rll}
\MO{Impl}\:\vx\:\va_i\:\mq
&
\iarg{\vn}{\ttinterp{\va_i}}&
\mbox{(if the $i$th argument to $\vx$ is implicit argument $\vn$)}\\
& \carg{\ttinterp{\va_i}} &
\mbox{(if the $i$th argument to $\vx$ is a constraint argument)}\\
& \ttinterp{\va_i} & \mbox{(otherwise)}
\end{array}
}
}

\noindent
This is mostly a straightforward
traversal of the \TT{} expression, translating directly to an \Idris{} equivalent.
The interesting case is for applications of named functions,
$\uninterp{\vx\:\ta}$, where the arguments are translated to either implicit,
constraint or explicit arguments according to the definition of $\vx$.
Since only type declarations are allowed to have implicit or constraint arguments,
and $\uninterp{\cdot}$ translates \emph{expressions},
all function types are assumed to take explicit arguments.

We also define an operation $\MO{Unelab}$, which translates \TT{} declarations
to corresponding \Idris{} declarations. This generates data declarations and
pattern matching definitions only --- it makes no attempt to reconstruct 
\texttt{class}
or \texttt{instance} declarations, or rebuild \texttt{case} expressions.
First, we define the reverse elaboration of type declarations, which must reconstruct
which arguments are implicit or constraint arguments:

\DM{
\AR{
\begin{array}{l}
\MO{UnelabType}\:(\vx\Hab\vt)\:\mq\:\vx\Hab\MO{UnelabTyDecl}\:0\:\vt
\end{array}
\medskip\\
\begin{array}{ll}
\MO{UnelabTyDecl}\:\vi\:(\all{\vx}{\vT}\SC\ve)
\mq & \piimp{\vx}{\uninterp{\vT}}(\MO{UnelabTyDecl}\:(\vi+1)\:\ve) \\
 & \hg\mbox{(if the $i$th argument to $\vx$ is an implicit argument)} \\
 & \piconst{\uninterp{\vT}}(\MO{UnelabTyDecl}\:(\vi+1)\:\ve) \\
 & \hg\mbox{(if the $i$th argument to $\vx$ is a constraint argument)} \\
 & \piexp{\vx}{\uninterp{\vT}}(\MO{UnelabTyDecl}\:(\vi+1)\:\ve) \\
 & \hg\mbox{(otherwise)} \\
\end{array}
}
}

\noindent
Using this, we define $\MO{Unelab}$ for top level declarations. For pattern
matching clauses, we reverse elaborate as follows, discarding the explicit
pattern variable bindings and applying $\MO{UnelabType}$ to reconstruct
the type declaration:

\DM{
\AR{
\MO{Unelab}\:(\vx\Hab\vt)\:\mq\:\MO{UnelabType}\:(\vx\Hab\vt)\\
\MO{Unelab}\:(\pat{\tx}{\tU}\SC\FN{f}\:\tts\:=\:\ve)
\:\mq\:\uninterp{\FN{f}\:\tts}\:=\:\uninterp{\ve}
}
}

\noindent
For data type declarations, we unelaborate as follows, applying $\MO{UnelabType}$
for each of the top level type declarations:

\DM{
\AR{
\MO{Unelab}\:(\Data\;\TC{T}\:(\tx\Hab\ttt)\Hab\vT\;\Where\;\vec{\VV{cons}})
\:\\
\hg\hg\mq\:\idata\:\MO{UnelabType}\:(\TC{T}\Hab\all{\tx}{\ttt}\SC\vT)\:\iwhere\:
(\vec{\MO{UnelabType}}\:\vec{\VV{cons}})
}
}


\subsection{Elaboration Properties}

\todo[inline]{
What are the properties of elaboration?
Need to define unelaboration, and say that elaborating then unelaborating
an expression yields the original expression.
(Works for expressions but not declarations)
}

Properties:
\begin{itemize}
\item Elaboration produces a well-typed term
\item $\ve \MO{Matches} \uninterp{\ttinterp{\ve}}$
\end{itemize}




%\section{Compiling}



%\section{Syntax Extensions}

\Idris{} supports the implementation of Embedded Domain Specific Languages (EDSLs) in
several ways~\cite{res-dsl-padl12}. One way, as we have already seen, is through
extending \texttt{do} notation. Another important way is to allow extension of the core
syntax. In this section we describe two ways of extending the syntax: \texttt{syntax}
rules and \texttt{dsl} notation.

\subsection{\texttt{syntax} rules}

We have seen \texttt{if...then...else} expressions, but these
are not built in --- instead, we define a function in the prelude\ldots

\begin{SaveVerbatim}{boolelim}

boolElim : (x:Bool) -> |(t : a) -> |(f : a) -> a; 
boolElim True  t e = t;
boolElim False t e = e;

\end{SaveVerbatim}
\useverb{boolelim}

\noindent
\ldots and extend the core syntax with a \texttt{syntax} declaration:

\begin{SaveVerbatim}{syntaxif}

syntax if [test] then [t] else [e] = boolElim test t e;

\end{SaveVerbatim}
\useverb{syntaxif}

\noindent
The left hand side of a \texttt{syntax} declaration describes the syntax rule, and the right
hand side describes its expansion. The syntax rule itself consists of:

\begin{itemize}
\item \textbf{Keywords} --- here, \texttt{if}, \texttt{then} and \texttt{else}, which must
be valid identifiers
\item \textbf{Non-terminals} --- included in square brackets, \texttt{[test]}, \texttt{[t]}
and \texttt{[e]} here, which stand for arbitrary expressions. To avoid parsing ambiguities, 
these expressions cannot use syntax extensions at the top level (though they can be used
in parentheses).
\item \textbf{Names} --- included in braces, which stand for names which may be bound
on the right hand side.
\item \textbf{Symbols} --- included in quotations marks, e.g. \texttt{":="}. This can
also be used to include reserved words in syntax rules, such as \texttt{"let"} or \texttt{"in"}.
\end{itemize}

\noindent
The limitations on the form of a syntax rule are that it must include at least one
symbol or keyword, and there must be no repeated variables standing for non-terminals.
Rules can use previously defined rules, but may not be recursive.
The following syntax extensions would therefore be valid:

\begin{SaveVerbatim}{syntaxex}

syntax [var] ":=" [val]              = Assign var val;
syntax [test] "?" [t] ":" [e]        = if test then t else e;
syntax select [x] from [t] where [w] = SelectWhere x t w;
syntax select [x] from [t]           = Select x t;

\end{SaveVerbatim}
\useverb{syntaxex}

\noindent
Syntax macros can be further restricted to apply only in patterns (i.e., only on the left
hand side of a pattern match clause) or only in terms (i.e. everywhere but the left hand side
of a pattern match clause) by being marked as \texttt{pattern} or \texttt{term} syntax
rules. For example, we might define an interval as follows, with a static check
that the lower bound is below the upper bound using \texttt{so}:

\begin{SaveVerbatim}{interval}

data Interval : Type where
   MkInterval : (lower : Float) -> (upper : Float) -> 
                so (lower < upper) -> Interval

\end{SaveVerbatim}
\useverb{interval}

\noindent
We can define a syntax which, in patterns, always matches \texttt{oh} for the proof 
argument, and in terms requires a proof term to be provided:

\begin{SaveVerbatim}{intervalsyn}

pattern syntax "[" [x] "..." [y] "]" = MkInterval x y oh
term    syntax "[" [x] "..." [y] "]" = MkInterval x y ?bounds_lemma

\end{SaveVerbatim}
\useverb{intervalsyn} 

\noindent
In terms, the syntax \texttt{[x...y]} will generate a proof obligation
\texttt{bounds\_lemma} (possibly renamed).

Finally, syntax rules may be used to introduce alternative binding forms. For
exampe, a \texttt{for} loop binds a variable on each iteration:

\begin{SaveVerbatim}{forloop}

syntax for {x} "in" [xs] [body] = forLoop xs (\x => body)
  
main : IO ()
main = do for x in [1..10] do
              putStrLn ("Number " ++ show x)
          putStrLn "Done!"

\end{SaveVerbatim}
\useverb{forloop} 

\noindent
Note that we have used the \texttt{\{x\}} form to state that \texttt{x} represents
a bound variable, substituted on the right hand side. We have also put \texttt{"in"} in
quotation marks since it is already a reserved word.

\subsection{\texttt{dsl} notation}

The well-typed interpreter in Section \ref{sect:interp} is a simple example of
a common programming pattern with dependent types, namely: describe an
\emph{object language}
and its type system with dependent types to guarantee that only well-typed programs
can be represented, then program using that representation. Using this approach
we can, for example, write programs for serialising binary data~\cite{plpv11} or
running concurrent processes safely~\cite{cbconc-fi}.

Unfortunately, the form of object language programs makes it rather hard to program
this way in practice. Recall the factorial program in \texttt{Expr} for example:

\useverb{facttest}

\noindent
Since this is a particularly useful pattern, \Idris{} provides syntax
overloading~\cite{res-dsl-padl12} to make it easier to program in such
object languages:

\begin{SaveVerbatim}{exprdsl}

dsl expr
    lambda      = Lam
    variable    = Var
    index_first = stop
    index_next  = pop

\end{SaveVerbatim}
\useverb{exprdsl} 

\noindent
A \texttt{dsl} block describes how each syntactic construct is represented in an
object language. Here, in the \texttt{expr} language, any \Idris{} lambda is
translated to a \texttt{Lam} constructor; any variable is translated to the
\texttt{Var} constructor, using \texttt{pop} and \texttt{stop} to construct the
de Bruijn index (i.e., to count how many bindings since the variable itself was bound).
It is also possible to overload \texttt{let} in this way. We can now write \texttt{fact}
as follows:

\begin{SaveVerbatim}{factb}

fact : Expr G (TyFun TyInt TyInt)
fact = expr (\x => If (Op (==) x (Val 0))
                      (Val 1) (Op (*) (app fact (Op (-) x (Val 1))) x))

\end{SaveVerbatim}
\useverb{factb} 

\noindent
In this new version, \texttt{expr} declares that the next expression will be overloaded.
We can take this further, using idiom brackets, by declaring:

\begin{SaveVerbatim}{idiomexpr}

(<$>) : |(f : Expr G (TyFun a t)) -> Expr G a -> Expr G t
(<$>) = \f, a => App f a

pure : Expr G a -> Expr G a
pure = id

\end{SaveVerbatim}
\useverb{idiomexpr} 

\noindent
Note that there is no need for these to be part of an instance of \texttt{Applicative},
since idiom bracket notation translates directly to the names \texttt{<\$>} and
\texttt{pure}, and ad-hoc type-directed overloading is allowed. We can now say:

\begin{SaveVerbatim}{factc}

fact : Expr G (TyFun TyInt TyInt)
fact = expr (\x => If (Op (==) x (Val 0))
                      (Val 1) (Op (*) [| fact (Op (-) x (Val 1)) |] x))

\end{SaveVerbatim}
\useverb{factc} 

\noindent
With some more ad-hoc overloading and type class instances, and a new
syntax rule, we can even go as far as:

\begin{SaveVerbatim}{factd}

syntax IF [x] THEN [t] ELSE [e] = If x t e

fact : Expr G (TyFun TyInt TyInt)
fact = expr (\x => IF x == 0 THEN 1 ELSE [| fact (x - 1) |] * x)

\end{SaveVerbatim}
\useverb{factd} 





\section{Related Work}

\label{sect:related}

Dependently typed programming languages have becoming more prominent in recent
years as tools for verifying software correctness, and several experimental
languages are being developed, in particular Agda \cite{norell2007thesis},
Epigram \cite{McBride2004a,Levitation2010} and Trellys \cite{Kimmell2012}.
Furthermore, recent extensions to Haskell~\cite{Haskell98}, implemented in the
Glasgow Haskell Compiler, are bringing more of the power of dependent types to
Haskell. The problem of refining high level syntax to a core type theory also
applies to theorem provers based on dependent types such as
Coq~\cite{Bertot2004} and Matita~\cite{Asperti2011}.

Checking advanced type system features in Haskell involves a type system
parameterised over an underlying constraint system $X$ which captures
constraints such as type classes, constrained data types and type families.
Types are checked using an inference algorithm
\textsc{OutsideIn(X)}~\cite{Vytiniotis2011}, which is stratified into an
inference engine independent of the constraint
system, and a constraint solver for $X$. An additional difficulty faced by
Haskell, and hence any extensions, is the desire to support type \emph{inference}, in
which principal types may be inferred for top level functions. We have avoided
such difficulties since, in general, type
inference is undecidable for full dependent types. Indeed, it is not clear
that type inference is even desirable in many cases, as programmers
can use dependent types to state their intentions (hence a program
specification) more precisely. However in future work we
may consider adapting the \textsc{OutsideIn} approach to provide limited type
inference.

An earlier implementation of \Idris{} was built on the \Ivor{} proof engine
\cite{Brady2006b}. This implementation differed in one important way --- unlike
the present implementation, there was limited separation between the type
theory and the high level language. The type theory itself supported implicit
syntax and unification, with high level constructs such as the \texttt{with}
rule implemented directly. Two important disadvantages were found with this
approach, however: firstly, the type checker is much more complicated when
combined with unification, making it harder to maintain; secondly, adding new
high level features requires the type checker to support those features
directly. In contrast, elaboration by tactics gives a clean separation between
the low level and high level languages.

Matita uses a bi-directional refinement algorithm~\cite{Asperti}. This is a
type directed approach, maintaining a set of yet to be solved unification
problems and relying on a small kernel, similar to the approach we now take
with \Idris{}, However, their approach uses refinement rules rather than
actics. This leads to good error messages though it is not clear how
easy it would be to extend to additional high level language features, unlike
the tactic based approach.

The Agda implementation is based on a type theory with
implicit syntax and pattern matching --- Norell gives an algorithm for type checking
a dependently typed language with pattern matching and metavariables 
\cite{norell2007thesis}. Unlike the present \Idris{} implementation, metavariables
and implicit arguments are part of the type theory. This has the advantage that
implicit arguments can be used more freely (for example, in higher order
function arguments) at the expense of complicating the type system.

Epigram \cite{McBride2004a} and Oleg \cite{McBride1999} 
have provided much of the inspiration for the \Idris{} elaborator. Indeed,
the hole and guess bindings of \TTdev{} are taken directly from Oleg.
\Epigram{} does not implement pattern matching directly, but rather translates
pattern matching into elimination rules \cite{McBride2002}. This has the
advantage that
elimination rules provide termination and coverage proofs \emph{by construction}.
Furthermore they simplify implementation of the evaluator and provide easy
optimisation opportunities \cite{Brady2003}. However, it requires the
implementation of extra machinery for constructor manipulation
\cite{McBride2006} and so we have avoided it in the present implementation.



%[Observation: separate elaboration and type checking, sort of like in GHC which
%type checks the high level language and produces a type correct core language.
%Elaboration is effectively a type checker for the high level language, so we have
%a hope of providing reasonable error messages related to the original code.]

\section{Conclusion}

\label{sect:conclusion}

In this paper, I have given an overview of the programming language \Idris{},
and its core type theory \TT{}, giving a detailed algorithm for translating
high level programs into \TT{}.
\TT{} itself is deliberately small and simple, and the design has deliberately
resisted innovation so that we can rely on existing metatheoretic properties
being preserved. The kernel of the \Idris{} implementation consists of a type checker
and evaluator for \TT{} along with a pattern match compiler, which are implemented
in under 1000 lines of Haskell code. It is important that this kernel remains small
--- the correctness of the language implementation relies to a large extent on
the correctness of the underlying type system, and keeping the implementation small
reduces the possibility of errors.

The approach we have taken to implementing the high level language,
implementing an elaboration monad with tactics for program construction, allows
us to build programs on top of a small and unchanging kernel, rather than
extending the core language to deal with implicit syntax, unification and type
classes.  High level \Idris{} features are implemented by describing the
corresponding sequence of tactics to build an equivalent program in \TT{}, via
a development calculus of incomplete terms, \TTdev{}. A significant advantage
we have found with this approach is that higher level features can easily be
implemented in terms of existing elaborator components. For example, once we
have implemented elaboration for data types and functions, it is easy to add
several features:

\begin{itemize}
\item \textbf{Type classes}: A dictionary is merely a record containing the
functions which implement a type class instance. Since we have a tactic based
refinement engine, we can implement type class resolution as a tactic.
\item \textbf{\texttt{where} clauses}: We have access to local variables and
their types, so we can
elaborate \texttt{where} clauses at the point of definition simply by lifting
them to the top level. 
\item \textbf{\texttt{case} expressions}: Similar to \texttt{where} clauses,
these are implemented by lifting the branches out to a top level function.
\end{itemize}

We do not need to make any changes to the core language type system in order to 
implement these high level features. 
Other high level features such as dependent records, tuples and monad comprehensions
can be added equally easily --- and indeed have been added in the full implementation.  
Furthermore, 
since we have taken a tactic-based approach to elaborating \Idris{} to \TT{},
it is possible to expose tactics to the programmer. This opens up
the possibility of implementing domain specific decision procedures, or
implementing user defined tactics in a style similar to Coq's \texttt{Ltac}
language \cite{Delahaye2000}.  Although \TT{} is primarily intended as a core
language for \Idris{}, its rich type system also means that it could be used
as a core language for other high level languages, especially when augmented
with primitive operators, and used to express additional properties of those
languages.  

We have not discussed the performance of the elaboration
algorithm, or described how \Idris{} compiles to executable code. In practice,
we have found performance to be acceptable --- for example, the \Idris{}
library (31 files, 3718 lines of code in total at the time of writing)
elaborates in around 12 seconds\footnote{On a MacBook Pro, 2.8GHz Intel Core 2
Duo, 4Gb RAM}. Profiling suggests that the main bottleneck is locating holes
in a proof term, which can be improved by choosing a better representation
for proof terms, perhaps based on a zipper \cite{Huet1997}. Compilation is made
straightforward by the Epic library \cite{brady2011epic}, with I/O and foreign
functions handled using command-response interaction trees \cite{Hancock2000}.

The objective of this implementation of \Idris{} is to provide a platform
for experimenting with realistic, general purpose programming with dependent
types, by implementing a Haskell-like language augmented with \emph{full}
dependent types. 
In this paper, we have seen how such a high level language can be implemented
by building on top of a small, well-understood, easy to reason about type
theory. 
However, a programming language implementation is not an end in itself. 
Programming languages exist to support research and practice in many different
domains. In future work, therefore, I plan to apply domain specific
language based techniques to realistic problems in important safety critical
domains such as security and network protocol design and implementation. In
order to be successful, this will require a language which is expressive enough
to describe protocol specifications at a high level, and robust enough to
guarantee correct implementation of those protocols. \Idris{}, I believe, is
the right tool for this work.



\section*{Acknowledgements}

This work was funded by the Scottish Informatics and Computer Science Alliance
(SICSA) and by EU Framework 7 Project No. 248828 (ADVANCE).  My thanks to
Philip H\"{o}lzenspies, Kevin Hammond and Vilhelm Sj\"{o}berg for their
comments on an earlier draft of this paper, and to the anonymous referees for
their many insightful and constructive comments.

\bibliographystyle{jfp}
\bibliography{library.bib}

\appendix

%\section{TODOs}
%\listoftodos{}

%\section{Elaboration meta-operations}

%It's possible that it would be useful to have a quick reference of meta-operations
%used by the elaborator here.

%They are: $\ttinterp{\cdot}$, $\MO{Elab}$, $\MO{TTDecl}$,
%$\MO{NewProof}$, $\MO{NewTerm}$,
%$\MO{Term}$, $\MO{Type}$, $\MO{Context}$, $\MO{Patterns}$, $\MO{Lift}$, $\MO{Expand}$.

%\input{code}

\end{document}
