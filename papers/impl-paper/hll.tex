\section{\Idris{} --- the High Level Language}

\label{sect:hll}

\Idris{} is
a pure functional programming language with dependent types. It is
eagerly evaluated by default, and compiled via the Epic supercombinator
library~\cite{brady2011epic}, with irrelevant values and proof terms
automatically erased~\cite{Brady2003,Brady2005}.
In this section, I will give a brief introduction to programming in \Idris{},
covering the most important features. A full tutorial is available elsewhere
\cite{idristutorial}. 

\subsection{Preliminaries}

\Idris{} defines several primitive types: fixed width integers
\tTC{Int}, arbitrary width integers \tTC{Integer}, and
\tTC{Float} for numeric operations, \tTC{Char} and \tTC{String} for
text manipulation, and \tTC{Ptr} which represents foreign pointers.
There are also several data types declared in the library, including
\tTC{Bool}, with values \tDC{True} and \tDC{False}. All of the usual
arithmetic and comparison operators are defined for the primitive types,
and are overloaded using type classes.

An \Idris{} program consists of a module declaration, followed by an optional
list of imports and a collection of definitions and declarations, for example:

\begin{SaveVerbatim}{constprims}

module Main

x : Int
x = 42

main : IO ()
main = putStrLn ("The answer is " ++ show x)

\end{SaveVerbatim}
\useverb{constprims}

\noindent
Like Haskell, the main function is called \texttt{main}, and input and output
is managed with an \texttt{IO} monad. Unlike Haskell, however, \remph{all} top
level functions must have a type signature. This is due to type inference
being, in general, undecidable for languages with dependent types.

A module declaration also opens a \remph{namespace}. The fully qualified names
declared in this module are \texttt{Main.x} and \texttt{Main.main}.


\subsection{Types and Functions}

Data types are declared in a similar way to Haskell data types, with a similar
syntax. Natural numbers and lists, for example, are declared as follows in the
library:

\begin{SaveVerbatim}{natlist}

data Nat    = O   | S Nat           -- Natural numbers
                                    -- (zero and successor)
data List a = Nil | (::) a (List a) -- Polymorphic lists

\end{SaveVerbatim}
\useverb{natlist}

\noindent
Unary natural numbers can be either zero, or
the successor of another natural number (\texttt{S k}). 
Lists can either be empty (\texttt{Nil})
or a value added to the front of another list (\texttt{x :: xs}).
In the declaration for \tTC{List}, we used an infix operator \tDC{::}. New operators
such as this can be added using a fixity declaration, as follows:

\begin{SaveVerbatim}{infixcons}

infixr 10 :: 

\end{SaveVerbatim}
\useverb{infixcons}

\noindent
This declares that \texttt{::} is a right associative operator (\texttt{infixr})
with a precedence level of 10.
Functions, data constructors and type constructors may all be given infix
operators as names. They may be used in prefix form if enclosed in brackets,
e.g. \tDC{(::)}. 

\subsection{Functions}

Functions are implemented by pattern matching, again using a similar syntax to
Haskell. Some natural number arithmetic functions can be
defined as follows, again taken from the standard library:

\begin{SaveVerbatim}{natfns}

-- Unary addition
plus : Nat -> Nat -> Nat
plus O     y = y
plus (S k) y = S (plus k y)

-- Unary multiplication
mult : Nat -> Nat -> Nat
mult O     y = O
mult (S k) y = plus y (mult k y)

\end{SaveVerbatim}
\useverb{natfns}

\noindent
The standard arithmetic operators \texttt{+} and \texttt{*} are also overloaded
for use by \texttt{Nat}, and are implemented
using the above functions.  Unlike Haskell, there is no restriction on whether
types and function names must begin with a capital letter or not. 
%Function
%names (\tFN{plus} and \tFN{mult} above), data constructors (\tDC{O}, \tDC{S},
%\tDC{Nil} and \tDC{::}) and type constructors (\tTC{Nat} and \tTC{List}) are
%all part of the same namespace.

\Idris{} has an interactive prompt, at which we can test these functions:

\begin{SaveVerbatim}{fntest}

Idris> plus (S (S O)) (S (S O))
S (S (S (S O))) : Nat
Idris> mult (S (S (S O))) (plus (S (S O)) (S (S O)))
S (S (S (S (S (S (S (S (S (S (S (S O))))))))))) : Nat

\end{SaveVerbatim}
\useverb{fntest}

\noindent
Like arithmetic operations, integer literals are also overloaded using type classes, 
meaning that we can also test the functions as follows:

\begin{SaveVerbatim}{fntest}

Idris> plus 2 2 
S (S (S (S O))) : Nat
Idris> mult 3 (plus 2 2)
S (S (S (S (S (S (S (S (S (S (S (S O))))))))))) : Nat

\end{SaveVerbatim}
\useverb{fntest}

\subsubsection{\texttt{where} clauses}

Functions can also be defined \emph{locally} using \texttt{where} clauses. For example,
to define a function which reverses a list, we can use an auxiliary function which
accumulates the new, reversed list, and which does not need to be visible globally:

\begin{SaveVerbatim}{revwhere}

reverse : List a -> List a
reverse xs = revAcc [] xs where
  revAcc : List a -> List a -> List a
  revAcc acc [] = acc
  revAcc acc (x :: xs) = revAcc (x :: acc) xs

\end{SaveVerbatim}
\useverb{revwhere}

\noindent
Indentation is significant --- functions in the \texttt{where} block must be indented
further than the outer function.

\textbf{Remark (scope):} 
Any names which are visible in the outer scope are also visible in the
\texttt{where} clause (unless they are redefined in the \texttt{where} clause,
such as \texttt{xs} here). A name which appears only in the type will be in
scope in the \texttt{where} clause if it is a \remph{parameter} to one of the
types, i.e. it is fixed across the entire data structure.  In particular, this means
that the \texttt{a} in the definition of \texttt{reverse} above is the
\remph{same} \texttt{a} as in the definition of \texttt{revAcc}, as \texttt{a}
is a parameter of \texttt{List}.

\subsubsection{Dependent Types}

A standard example of a dependent type is the type of ``lists with length'',
conventionally called ``vectors'' in the dependently typed programming
literature. In \Idris{}, vectors are declared as follows:

\begin{SaveVerbatim}{vect}

data Vect : Type -> Nat -> Type where
   Nil  : Vect a O
   (::) : a -> Vect a k -> Vect a (S k)

\end{SaveVerbatim}
\useverb{vect}

\noindent
Note that this uses the same constructor names as for \tTC{List}. Ad-hoc name
overloading such as this is accepted by \Idris{}, provided that the names are
declared in different namespaces (in practice, normally in different modules)
so that the names are different internally. Namespace resolution can be made
explicitly (e.g. \texttt{List.Nil} or \texttt{Vect.Nil}) or more commonly
by type.

The above declaration creates a family of types, and requires a different form
of declaration from the simple type declarations above. It resembles a Haskell
GADT declaration: it explicitly states the type
of the type constructor \tTC{Vect} --- it takes a type and a \tTC{Nat} as an
argument, where \tTC{Type} stands for the type of types. We say that \tTC{Vect}
is \emph{parameterised} by a type, and \emph{indexed} over \tTC{Nat}. 
The distinction between parameters and indices is that a parameter is fixed
across an entire data structure, whereas an index may vary.
Each constructor targets a different part of the family of types. \tDC{Nil} can
only be used to construct vectors with zero length, and \tDC{::} to construct
vectors with non-zero length. The type of \tDC{::} states explicitly that an element
of type \texttt{a} and a tail of type \texttt{Vect a k} (i.e., a vector of length \texttt{k})
combine to make a vector of length \texttt{S k}.

Functions on dependent types such as \tTC{Vect} are declared in the same way
as on simple types such as \tTC{List} and \tTC{Nat} above, by pattern matching.
The type of a function over \tTC{Vect} will describe what happens to the
lengths of the vectors involved. For example, \tFN{++}, defined in the
library, appends two \tTC{Vect}s:

\begin{SaveVerbatim}{vapp}

(++) : Vect A n -> Vect A m -> Vect A (n + m)
Nil       ++ ys = ys
(x :: xs) ++ ys = x :: xs ++ ys

\end{SaveVerbatim}
\useverb{vapp}

\subsubsection*{Example: The Finite Sets}

Finite sets, as the name suggests, are sets with a finite number of elements.
They are declared as follows in the library:

\begin{SaveVerbatim}{findecl}

data Fin : Nat -> Type where
   fO : Fin (S k)
   fS : Fin k -> Fin (S k)

\end{SaveVerbatim}
\useverb{findecl}

\noindent
This declares
\tDC{fO} as the zeroth element of a finite set with \texttt{S k} elements; 
\texttt{fS n} as the
\texttt{n+1}th element of a finite set with \texttt{S k} elements. 
\tTC{Fin} is indexed by \tTC{Nat}, which
represents the number of elements in the set. 
Neither constructor targets \texttt{Fin O}, because we cannot construct an
element of an empty set.

A useful application of the \tTC{Fin} family is to represent bounded
natural numbers. Since the first \tTC{n} natural numbers form a finite
set of \tTC{n} elements, we can treat \tTC{Fin n} as the set of natural
numbers bounded by \tTC{n}. 

For example, the following function which looks up an element in a \tTC{Vect},
by a bounded index given as a \tTC{Fin n}, is defined in the library:

\begin{SaveVerbatim}{vindex}

index : Fin n -> Vect a n -> a
index fO     (x :: xs) = x
index (fS k) (x :: xs) = index k xs

\end{SaveVerbatim}
\useverb{vindex}

\noindent
This function looks up a value at a given location in a vector. The location is
bounded by the length of the vector (\texttt{n} in each case), so there is no
need for a run-time bounds check. The type checker guarantees that the location
is no larger than the length of the vector.

Note also that there is no case for \texttt{Nil} here. It would be impossible
to apply such a case --- since there is no element of \texttt{Fin O}, and the
location is a \texttt{Fin n}, then \texttt{n} can not be \tDC{O}.  As a result,
attempting to look up an element in an empty vector would give a compile time
type error.

\subsubsection{Implicit Arguments}

Let us take a closer look at the type of \texttt{index}:

\begin{SaveVerbatim}{vindexty}

index : Fin n -> Vect a n -> a

\end{SaveVerbatim}
\useverb{vindexty}

\noindent
It takes two arguments, an element of the finite set of \texttt{n} elements, and a vector
with \texttt{n} elements of type \texttt{a}. But there are also two names, 
\texttt{n} and \texttt{a}, which are not declared explicitly. These are \emph{implicit}
arguments to \texttt{index}. The type of \texttt{index} could also be written as:

\begin{SaveVerbatim}{vindeximppl}

index : {a:_} -> {n:_} -> Fin n -> Vect a n -> a

\end{SaveVerbatim}
\useverb{vindeximppl}

\noindent
This gives bindings for \texttt{a} and \texttt{n} with placeholders for
their types, to be inferred by the machine. These types could also be given explicitly:

\begin{SaveVerbatim}{vindeximpty}

index : {a:Type} -> {n:Nat} -> Fin n -> Vect a n -> a

\end{SaveVerbatim}
\useverb{vindeximpty}

\noindent
Implicit arguments, given in braces \texttt{\{\}} in the type signature, are
not given in applications of \texttt{index}; their values can be inferred from
the types of the \texttt{Fin n} and \texttt{Vect a n} arguments. Any name which
appears as a parameter or index in a type signature, but which is otherwise
free, will be automatically bound as an implicit argument.  Implicit arguments
can still be given explicitly in applications, using the syntax
\texttt{\{a=value\}} and \texttt{\{n=value\}}, for example:

\begin{SaveVerbatim}{vindexexp}

index {a=Int} {n=2} fO (2 :: 3 :: Nil)

\end{SaveVerbatim}
\useverb{vindexexp}

\noindent
In fact, any argument, implicit or explicit, may be given a name. For example,
the type of \texttt{index} could be declared as:

\begin{SaveVerbatim}{vindexn}

index : (i:Fin n) -> (xs:Vect a n) -> a

\end{SaveVerbatim}
\useverb{vindexn}

\noindent
This can be useful for improving the readability of type signatures, particularly
where the name suggests the argument's purpose.

\subsubsection{Totality Checking}

Internally, \Idris{} programs are checked for \emph{totality} --- that they
produce an answer in finite time for all possible inputs --- but they are not
\emph{required} to be total by default. Totality checking serves two
purposes: firstly, if a program terminates for all inputs 
then its type gives a strong
guarantee about the properties specified by its type; secondly, we can
optimise total programs more aggressively~\cite{Brady2003}. 

Totality checking can be enforced by using the \texttt{total} keyword. For
example, recall the \texttt{vAdd} function:

\begin{SaveVerbatim}{vadd}

total vAdd : Num a => Vect a n -> Vect a n -> Vect a n
vAdd []        []        = []
vAdd (x :: xs) (y :: ys) = x + y :: vAdd xs ys

\end{SaveVerbatim}
\useverb{vadd}

\noindent
The elaborator can verify that this is total by checking that it covers all
possible patterns --- in this case, both arguments must be of the same form
as the type requires that the input vectors are the same length --- and that
recursive calls are on structurally smaller values. The totality checker is
implemented independently of the type checker and elaborator presented in the
remainder of this paper. While we could consider building totality proofs
by translating functions to eliminators~\cite{McBride2004a} we have taken
the more pragmatic approach of allowing more flexibility for the programmer
at the expense of simplicity of totality checking. \Idris{} also supports
coinductive definitions with a corresponding productivity checker, although
futher details are beyond the scope of this paper.

\subsection{Type Classes}

\Idris{} supports overloading in two ways. Firstly, as we have already seen with
the constructors of \texttt{List} and \texttt{Vect}, names
can be overloaded in an ad-hoc manner and resolved according to the context in which
they are used. This is mostly for convenience, to eliminate the need to decorate
constructor names in similarly structured data types, and eliminate explicit qualification
of ambiguous names where only one is well-typed --- this is especially useful
for disambiguating record field names\footnote{Records are however beyond the scope
of this paper}.

Secondly, \Idris{} implements \remph{type classes}, following Haskell.  This
allows a more principled approach to overloading --- a type class gives a
collection of overloaded operations which describe the interface for
\remph{instances} of that class.

A simple example
is the \texttt{Show} type class, which is defined in the library and
provides an interface for converting values to
\texttt{String}s:

\begin{SaveVerbatim}{showclass}

class Show a where
    show : a -> String

\end{SaveVerbatim}
\useverb{showclass}

\noindent
This declares a function of the following type (which we call a \emph{method} of the 
\texttt{Show} class):

\begin{SaveVerbatim}{showty}

show : Show a => a -> String

\end{SaveVerbatim}
\useverb{showty}

An instance of a class
is defined with an \texttt{instance} declaration, which provides implementations of
the function for a specific type. For example, the \texttt{Show} instance for \texttt{Nat}
could be defined as:

\begin{SaveVerbatim}{shownat}

instance Show Nat where
    show O = "O"
    show (S k) = "s" ++ show k

\end{SaveVerbatim}
\useverb{shownat}

\begin{SaveVerbatim}{shownati}

Idris> show (S (S (S O))) 
"sssO" : String

\end{SaveVerbatim}
\useverb{shownati}

\noindent
Only one instance of a class can be given for a type --- instances may not overlap.
Instance declarations can themselves have constraints. For example, to define a
\texttt{Show} instance for vectors, we need to know that there is a \texttt{Show} 
instance for the element type, because we are going to use it to convert each element
to a \texttt{String}:

\begin{SaveVerbatim}{showvec}

instance Show a => Show (Vect a n) where
    show xs = "[" ++ show' xs ++ "]" where
        show' : Vect a n -> String
        show' Nil        = ""
        show' (x :: Nil) = show x
        show' (x :: xs)  = show x ++ ", " ++ show' xs

\end{SaveVerbatim}
\useverb{showvec}

%\noindent
%\textbf{Remark: } The type of the auxiliary function \texttt{show'} is
%important. The type variables \texttt{a} and \texttt{n} which are part of the
%instance declaration for \texttt{Show (Vect a n)} are fixed across the entire
%instance declaration. As a result, \texttt{a} need not be constrained
%again. Furthermore, it means that if \texttt{n} is used again in the type, it refers
%to the (fixed) length of the outermost list \texttt{xs}. Therefore,
%there is a different name for the length \texttt{n'} in \texttt{show'}.

\noindent
Like Haskell type classes, default definitions can be given in the class declaration.
Otherwise, all methods must be given in an instance. For example, there is an
\texttt{Eq} class:

\begin{SaveVerbatim}{eqdefault}

class Eq a where
    (==) : a -> a -> Bool
    (/=) : a -> a -> Bool

    x /= y = not (x == y)
    y == y = not (x /= y)

\end{SaveVerbatim}
\useverb{eqdefault}

\noindent
Classes can also be extended. A logical next step from an equality relation \texttt{Eq}
is to define an ordering relation \texttt{Ord}. We can define an \texttt{Ord} class
which inherits methods from \texttt{Eq} as well as defining some of its own:

\begin{SaveVerbatim}{ord}

data Ordering = LT | EQ | GT

\end{SaveVerbatim}
\useverb{ord} 

\begin{SaveVerbatim}{eqord}

class Eq a => Ord a where
    compare : a -> a -> Ordering
    (<) : a -> a -> Bool
    -- etc

\end{SaveVerbatim}
\useverb{eqord}

\subsection{Matching on intermediate values}

%\subsubsection{\texttt{let} bindings}
%
%Intermediate values can be calculated using \texttt{let} bindings:
%
%\begin{SaveVerbatim}{letb}
%
%mirror : List a -> List a
%mirror xs = let xs' = rev xs in
%                app xs xs'
%
%\end{SaveVerbatim}
%\useverb{letb} 
%
%\noindent
%Pattern matching is also supported in \texttt{let} bindings. For example, extracting
%fields from a record can be achieved as follows, as well as by pattern matching at the top level:
%
%\begin{SaveVerbatim}{letp}
%
%data Person = MkPerson String Int
%
%showPerson : Person -> String
%showPerson p = let MkPerson name age = p in
%                   name ++ " is " ++ show age ++ " years old"
%
%\end{SaveVerbatim}
%\useverb{letp} 

\subsubsection{\texttt{case} expressions}

Intermediate values of \emph{non-dependent} types can be inspected using a
\texttt{case} expression.  For example, \texttt{list\_lookup} looks up an index
in a list, returning \texttt{Nothing} if the index is out of bounds. This can
be used to write \texttt{lookup\_default}, which looks up an index and
returns a default value if the index is out of bounds:

\begin{SaveVerbatim}{listlookup}

lookup_default : Nat -> List a -> a -> a
lookup_default i xs def = case list_lookup i xs of
                              Nothing => def
                              Just x => x

\end{SaveVerbatim}
\useverb{listlookup} 

The \texttt{case} construct is intended for simple analysis of intermediate
expressions to avoid the need to write auxiliary functions.  It will
\emph{only} work if each branch \emph{matches} a value of the same type, and
\emph{returns} a value of the same type.

\subsubsection{The \texttt{with} rule}

Since types can depend on values, the form of some arguments can be determined
by the value of others. For example, if we were to write down the implicit
length arguments to \texttt{(++)}, we would see that the form of the length argument was
determined by whether the vector was empty or not:

\begin{SaveVerbatim}{appdep}

(++) : Vect a n -> Vect a m -> Vect a (n + m)
(++) {n=O}   []        [] = []
(++) {n=S k} (x :: xs) ys = x :: xs ++ ys

\end{SaveVerbatim}
\useverb{appdep}

\noindent
If \texttt{n} was a successor in the \texttt{[]} case, or zero in the 
\texttt{::} case, the definition
would not be well typed.

Often, matching is required on the result of an intermediate computation
with a dependent type.
\Idris{} provides a construct for this, the \texttt{with} rule, 
inspired by views in \Epigram~\cite{McBride2004a},
which takes account of the
fact that matching on a value in a dependently typed language can affect what
is known about the forms of other values. 

For example, a \texttt{Nat} is either even or odd. 
If it is even it will
be the sum of two equal \texttt{Nat}s. Otherwise, it is the sum of two equal \texttt{Nat}s 
plus one:

\begin{SaveVerbatim}{parity}

data Parity : Nat -> Type where
   even : Parity (n + n)
   odd  : Parity (S (n + n))

\end{SaveVerbatim}
\useverb{parity}

\noindent
We say \texttt{Parity} is a \emph{view} of \texttt{Nat}. 
It has a \emph{covering function} which tests whether
it is even or odd and constructs the predicate accordingly.

\begin{SaveVerbatim}{parityty}

parity : (n:Nat) -> Parity n

\end{SaveVerbatim}
\useverb{parityty}

\noindent
Using this, a function which converts a natural number to a list
of binary digits (least significant first) is written as follows, using the \texttt{with}
rule:

\begin{SaveVerbatim}{natToBin}

natToBin : Nat -> List Bool
natToBin O = Nil
natToBin k with (parity k)
   natToBin (j + j)     | even = False :: natToBin j
   natToBin (S (j + j)) | odd  = True  :: natToBin j

\end{SaveVerbatim}
\useverb{natToBin}

\noindent
The value of the result of \texttt{parity k} affects the form of \texttt{k}, 
because the result
of \texttt{parity k} depends on \texttt{k}. 
So, as well as the patterns for the result of the
intermediate computation (\texttt{even} and \texttt{odd}) right of the 
\texttt{$\mid$}, the definition also expresses how
the results affect the other patterns left of the $\mid$. Note that there is a
function in the patterns (\texttt{+}) and repeated occurrences of \texttt{j} --- 
this is allowed
because another argument has determined the form of these patterns.


