\section{Related Work}

\label{sect:related}

Dependently typed programming languages have become more prominent in recent
years as tools for verifying software correctness, and several experimental
languages are being developed, in particular Agda \cite{norell2007thesis},
Epigram \cite{McBride2004a,Levitation2010} and Trellys \cite{Kimmell2012}.
Furthermore, recent extensions to Haskell~\cite{Haskell98}, implemented in the
Glasgow Haskell Compiler, are bringing more of the power of dependent types to
Haskell. The problem of refining high level syntax to a core type theory also
applies to theorem provers based on dependent types such as
Coq~\cite{Bertot2004} and Matita~\cite{Asperti2011}.

Checking advanced type system features in Haskell involves a type system
parameterised over an underlying constraint system $X$ which captures
constraints such as type classes, constrained data types and type families.
Types are checked using an inference algorithm
\textsc{OutsideIn(X)}~\cite{Vytiniotis2011}, which is stratified into an
inference engine independent of the constraint
system, and a constraint solver for $X$. An additional difficulty faced by
Haskell, and hence any extensions, is the desire to support type \emph{inference}, in
which principal types may be inferred for top level functions. We have avoided
such difficulties since, in general, type
inference is undecidable for full dependent types. Indeed, it is not clear
that type inference is even desirable in many cases, as programmers
can use dependent types to state their intentions (hence a program
specification) more precisely. However in future work we
may consider adapting the \textsc{OutsideIn} approach to provide limited type
inference.

An earlier implementation of \Idris{} was built on the \Ivor{} proof engine
\cite{Brady2006b}. This implementation differed in one important way --- unlike
the present implementation, there was limited separation between the type
theory and the high level language. The type theory itself supported implicit
syntax and unification, with high level constructs such as the \texttt{with}
rule implemented directly. Two important disadvantages were found with this
approach, however: firstly, the type checker is much more complicated when
combined with unification, making it harder to maintain; secondly, adding new
high level features requires the type checker to support those features
directly. In contrast, elaboration by tactics gives a clean separation between
the low level and high level languages, and results in programs in a core
type theory which is separately checkable, perhaps even by an independently
written checker which implements the \TT{} rules directly.

Matita uses a bi-directional refinement algorithm~\cite{Asperti}. This is a
type directed approach, maintaining a set of yet to be solved unification
problems and relying on a small kernel, similar to the approach we now take
with \Idris{}, However, their approach uses refinement rules rather than
tactics. This leads to good error messages though it is not clear how
easy it would be to extend to additional high level language features, unlike
the tactic based approach.

The Agda implementation is based on a type theory with
implicit syntax and pattern matching --- Norell gives an algorithm for type checking
a dependently typed language with pattern matching and metavariables 
\cite{norell2007thesis}. Unlike the present \Idris{} implementation, metavariables
and implicit arguments are part of the type theory. This has the advantage that
implicit arguments can be used more freely (for example, in higher order
function arguments) at the expense of complicating the type system.

Epigram \cite{McBride2004a} and Oleg \cite{McBride1999} 
have provided much of the inspiration for the \Idris{} elaborator. Indeed,
the hole and guess bindings of \TTdev{} are taken directly from Oleg.
\Epigram{} does not implement pattern matching directly, but rather translates
pattern matching into elimination rules \cite{McBride2002}. This has the
advantage that
elimination rules provide termination and coverage proofs \emph{by construction}.
Furthermore they simplify implementation of the evaluator and provide easy
optimisation opportunities \cite{Brady2003}. However, it requires the
implementation of extra machinery for constructor manipulation
\cite{McBride2006} and so we have avoided it in the present implementation.



%[Observation: separate elaboration and type checking, sort of like in GHC which
%type checks the high level language and produces a type correct core language.
%Elaboration is effectively a type checker for the high level language, so we have
%a hope of providing reasonable error messages related to the original code.]

\section{Conclusion}

\label{sect:conclusion}

In this paper, I have given an overview of the programming language \Idris{},
and its core type theory \TT{}, giving a detailed algorithm for translating
high level programs into \TT{}.
\TT{} itself is deliberately small and simple, and the design has deliberately
resisted innovation so that we can rely on existing metatheoretic properties
being preserved. The kernel of the \Idris{} implementation consists of a type checker
and evaluator for \TT{} along with a pattern match compiler, which are implemented
in under 1000 lines of Haskell code. It is important that this kernel remains small
--- the correctness of the language implementation relies to a large extent on
the correctness of the underlying type system, and keeping the implementation small
reduces the possibility of errors.

The approach we have taken to implementing the high level language,
implementing an elaboration monad with tactics for program construction, allows
us to build programs on top of a small and unchanging kernel, rather than
extending the core language to deal with implicit syntax, unification and type
classes.  High level \Idris{} features are implemented by describing the
corresponding sequence of tactics to build an equivalent program in \TT{}, via
a development calculus of incomplete terms, \TTdev{}. A significant advantage
we have found with this approach is that higher level features can easily be
implemented in terms of existing elaborator components. For example, once we
have implemented elaboration for data types and functions, it is easy to add
several features:

\begin{itemize}
\item \textbf{Type classes}: A dictionary is merely a record containing the
functions which implement a type class instance. Since we have a tactic based
refinement engine, we can implement type class resolution as a tactic.
\item \textbf{\texttt{where} clauses}: We have access to local variables and
their types, so we can
elaborate \texttt{where} clauses at the point of definition simply by lifting
them to the top level. 
\item \textbf{\texttt{case} expressions}: Similar to \texttt{where} clauses,
these are implemented by lifting the branches out to a top level function.
\end{itemize}

We do not need to make any changes to the core language type system in order to 
implement these high level features. 
Other high level features such as dependent records, tuples and monad comprehensions
can be added equally easily --- and indeed have been added in the full implementation.  
Furthermore, 
since we have taken a tactic-based approach to elaborating \Idris{} to \TT{},
it is possible to expose tactics to the programmer. This opens up
the possibility of implementing domain specific decision procedures, or
implementing user defined tactics in a style similar to Coq's \texttt{Ltac}
language \cite{Delahaye2000}.  Although \TT{} is primarily intended as a core
language for \Idris{}, its rich type system also means that it could be used
as a core language for other high level languages, especially when augmented
with primitive operators, and used to express additional properties of those
languages.  

We have not discussed the performance of the elaboration
algorithm, or described how \Idris{} compiles to executable code. In practice,
we have found performance to be acceptable --- for example, the \Idris{}
library (31 files, 3718 lines of code in total at the time of writing)
elaborates in around 12 seconds\footnote{On a MacBook Pro, 2.8GHz Intel Core 2
Duo, 4Gb RAM}. Profiling suggests that the main bottleneck is locating holes
in a proof term, which can be improved by choosing a better representation
for proof terms, perhaps based on a zipper \cite{Huet1997}. Compilation is made
straightforward by the Epic library \cite{brady2011epic}, with I/O and foreign
functions handled using command-response interaction trees \cite{Hancock2000}.
Although I have not yet run detailed benchmarks of the performance of compiled
code, I believe that rich type information and guaranteed termination will
allow aggressive optimisations~\cite{Brady2003,Brady2005}. I will investigate
this in future work.

The objective of this implementation of \Idris{} is to provide a platform
for experimenting with realistic, general purpose programming with dependent
types, by implementing a Haskell-like language augmented with \emph{full}
dependent types. 
In this paper, we have seen how such a high level language can be implemented
by building on top of a small, well-understood, easy to reason about type
theory. 
However, a programming language implementation is not an end in itself. 
Programming languages exist to support research and practice in many different
domains. In future work, therefore, I plan to apply domain specific
language based techniques to realistic problems in important safety critical
domains such as security and network protocol design and implementation. In
order to be successful, this will require a language which is expressive enough
to describe protocol specifications at a high level, and robust enough to
guarantee correct implementation of those protocols. \Idris{}, I believe, is
the right tool for this work.

