\section{Introduction}

Dependently typed programming languages, such as Agda \cite{norell2007thesis}
and Coq \cite{Bertot2004}, have emerged in recent years as a promising approach
to ensuring the correctness of software. Type checking ensures a program has
the intended meaning; \emph{dependent} types, where types may be predicated
on values, allow a programmer to give a program a more precise type and 
hence have increased confidence of its correctness.
The \Idris{} language
\cite{Brady2011a} aims to take this idea further, by providing support for
verification of \emph{general purpose} systems software. In contrast to Agda and Coq,
which have arisen from the theorem proving community, \Idris{} takes Haskell as
its main influence.  Recent Haskell extensions such as GADTs and type families have
given some of the power of dependent types to Haskell programmers. In the short
term this approach has a clear advantage, since it builds on a mature language
infrastructure with extensive library support.
Taking a longer term view, however,
these extensions are inherently limited in that they are required to maintain
backwards compatibility with existing Haskell implementations.  \Idris{}, being a new
language, has no such limitation, essentially asking the question:

\begin{center}
\emph{``What if Haskell had \emph{full} dependent types?''}
\end{center}

By \emph{full} dependent types, we mean that there is no restriction on which
values may appear in types.  It is important for the sake of usability of a
programming language to provide a notation which allows programmers to express
high level concepts in a natural way. Taking Haskell as a starting point means
that \Idris{} offers a variety of high level structures such as type classes,
\texttt{do}-notation, primitive types and monadic I/O, for example.
Furthermore, a goal of \Idris{} is to support high level domain specific
language implementation, providing appropriate notation for \emph{domain
experts} and \emph{systems programmers}, who should not be required to be type
theorists in order to solve programming problems.  Nevertheless, it is
important for the sake of correctness of the language implementation to have a
well-defined core language with well-understood meta-theory \cite{Altenkirch2010}. 
How can we achieve both of these goals?

This paper describes a method for elaborating a high level dependently typed
functional programming language to a low level core based on dependent
type theory.  The method involves building an elaboration monad which captures
the state of incomplete programs and a collection of \emph{tactics} for
refining programs, directed by high level syntax.  As we shall see,
this method allows higher level language constructs to be constructed in a
straightforward manner, without compromising the simplicity of the underlying
type theory.

\subsection{Contributions}

A dependently typed programming language relies on several components, many of
which are now well understood. For example, we rely on a type checker for
the core type theory \cite{Chapman2005epigram,loh2010tutorial}, a
unification algorithm \cite{Miller1992} and an evaluator. However, it is less
well understood how to combine these components effectively into a practical
programming language. 

The primary contribution of this paper is a tactic based method for translating
programs in a high level dependently typed programming language to a small core
type theory, \TT{}, based on UTT \cite{luo1994}. The paper describes the
structure of an elaboration monad capturing proof and system state, and
introduces a collection of \remph{tactics} for manipulating incomplete
programs.  Secondly, the paper gives a detailed description of the core type
theory used by \Idris{}, including a full description of the typing rules.
Finally, through describing the specific tactics, the paper shows how to extend
\Idris{} with higher level features.  While we apply these ideas to \Idris{}
specifically, the method for term construction is equally applicable to other
typed programming languages, and indeed the tactics themselves are applicable
to any high level language which can be explained in terms of \TT{}.

The paper is structured as follows: Section \ref{sect:hll} gives an overview of
programming in \Idris{}, introducing the high level constructs which will be
translated to the core language; Section \ref{sect:typechecking} describes the
core language \TT{} and its typing rules, covering expressions, data types
and pattern matching; Section \ref{sect:elaboration} describes the method
for translating \Idris{} to \TT{}, beginning with a restricted language
\IdrisM{} before extending to higher level features; Section
\ref{sect:delab} describes the process for translating \TT{} back to \Idris{}
and properties of the translation; finally, 
Section \ref{sect:related} discusses related work and Section \ref{sect:conclusion}
concludes.

\subsection{Typographical conventions}

This paper presents programs in two different but related languages: a high level
language \Idris{} intended for programmers, and a low level language \TT{} to
which \Idris{} is elaborated. We distinguish these languages typographically as
follows:

\begin{itemize}
\item \Idris{} programs are written in \texttt{typewriter} font, as they are written
in a conventional text editor. We use \texttt{e$_i$} to stand for non-terminal
expressions.
\item \TT{} programs are written in mathematical notation, with names arising
from \Idris{} expressions written in \texttt{typewriter} font. We use vector notation
$\te$ to stand for sequences of expressions.
\end{itemize}

Additionally, we describe the translation from \Idris{} to \TT{} in the form
of \emph{meta-operations}, which in practice are Haskell programs. Meta-operations
are operations on \Idris{} and \TT{} syntax, and are identified by their names being
in $\MO{SmallCaps}$.

\subsection{Elaboration Example}

\Idris{} is a Haskell-like pure functional programming language with dependent
types.  A simple example program is the following, which adds corresponding
elements of vectors of the same length:

\begin{SaveVerbatim}{vadd}

vAdd : Num a => Vect n a -> Vect n -> Vect n a
vAdd []        []        = []
vAdd (x :: xs) (y :: ys) = x + y :: vAdd xs ys

\end{SaveVerbatim}
\useverb{vadd}

\noindent
This illustrates some basic features of \Idris{}:

\begin{itemize}
\item A single colon is used for type signatures, and a double colon for the
cons operator, emphasising the importance of types.
\item Functions are defined by pattern matching, with a top level type signature.
Names which are free in the type signature (\texttt{a} and \texttt{n}) here
are implicitly bound.
\item The type system ensures that both input vectors are the same length (\texttt{n})
and have the same element type (\texttt{a}), and that the element type and length
are preserved in the output vector.
\item Functions can be overloaded using \remph{type classes}. Here, the element type
of the vector \texttt{a} must be numeric and therefore supports the \texttt{+} operator.
\end{itemize}

\noindent
\Idris{} programs elaborate into a small core language, \TT{}, which is a $\lambda$-calculus
with dependent types, augmented with algebraic data types and pattern matching.
\TT{} programs are fully explicitly typed, including the names \texttt{a} and \texttt{n}
in the type signature, and the names bound in each pattern match clause.
Type classes are also made explicit.
The \TT{} translation of \texttt{vAdd} is:

\DM{
\AR{
\FN{vAdd} \Hab(\va\Hab\Type)\to(\vn\Hab\Nat)\to\TC{Num}\;\va\to
  \Vect\;\vn\;\va\to\Vect\;\vn\;\va\to\Vect\;\vn\;\va\\
\RW{var}\;\va\Hab\Type,\;\vc\Hab\TC{Num}\;\va\SC\\
\hg\FN{vAdd} \; \va \; \Z \; \vc \; (\DC{Nil}\;\va) \; (\DC{Nil}\;\va) \; 
\cq\;\DC{Nil}\;\va \\
\RW{var}\;\AR{
\va\Hab\Type,\;\vk\Hab\Nat,\;\vc\Hab\TC{Num}\;\va,\\
\vx\Hab\va,\;\vxs\Hab\Vect\;\vk\;\va,\;
\vy\Hab\va,\;\vys\Hab\Vect\;\vk\;\va
\SC
\\
\FN{vAdd} \; \va \; (\suc\;\vk) \; \vc 
  \; ((\DC{::})\;\va\;\vk\;\vx\;\vxs) 
  \; ((\DC{::})\;\va\;\vk\;\vy\;\vys) 
   \\
   \hg\hg\cq\:((\DC{::})\;\va\;\vk\;((+)\;\vc\;\vx\;\vy)\; (\FN{vAdd}\;\va\;\vk\;\vc\;\vxs\;\vys))
}
}
}

\noindent
The rest of this paper describes \Idris{} and \TT{}, and systematically explains
how to translate from one to the other.

